% !TEX TS-program = pdflatex
% !TeX program = pdflatex
% !TEX encoding = UTF-8
% !TEX spellcheck = fr

\documentclass[11pt, a4paper]{article}
%\usepackage{fullpage}
\usepackage[left=1cm,right=1cm,top=1cm,bottom=2cm]{geometry}
\usepackage[fleqn]{amsmath}
\usepackage{amssymb}
%\usepackage{indentfirst}
\usepackage[T1]{fontenc}
\usepackage[utf8]{inputenc}
\usepackage[french,english]{babel}
\usepackage{txfonts} 
\usepackage[]{graphicx}
\usepackage{multirow}
\usepackage{hyperref}
\usepackage{parskip}
\usepackage{multicol}
\usepackage{wrapfig}

\usepackage{turnstile}%Induction symbole

\usepackage{tikz}
\usetikzlibrary{arrows, automata}
\usetikzlibrary{decorations.pathmorphing}

\renewcommand{\baselinestretch}{1}

\setlength{\parindent}{24pt}


\begin{document}

\selectlanguage {french}
%\pagestyle{empty} 

\noindent
\begin{tabular}{ll}
\multirow{3}{*}{\includegraphics[width=2cm]{../../../img/esi-logo.png}} & \'Ecole national Supérieure d'Informatique\\
& 2\textsuperscript{ème} année cycle supérieure (2CSSID)\\
& Traitement automatique du langage naturel (2021-2022)
\end{tabular}\\[.25cm]
\noindent\rule{\textwidth}{1pt}\\%[-0.25cm]
\begin{center}
{\LARGE \textbf{TP01 : Fouille des contacts}}
\begin{flushright}
	ARIES Abdelkrime
\end{flushright}
\end{center}
\noindent\rule{\textwidth}{1pt}

On veut récupérer les adresses émail, les réseaux sociaux et les numéros de téléphone qui se figurent dans les pages "contactez nous" des sites web en Algérie. 
Le problème est que ces pages représentent ces informations de plusieurs façons.
Pour récupérer et unifier la forme de ces informations, on va utiliser les expressions régulières. 

\section*{1. Informations recherchées}

Ici, on va décrire quelques variations existantes des informations qui existent dans les pages.
Ce n'est pas une liste complète ; donc, il faut ouvrir les pages où le système a échoué afin de localiser les formes non reconnues.

\subsection*{1.1. Réseaux sociaux}

On s'intéresse par les réseaux : facebook, twitter, youtube, linkedin et instagram. 
Lorsqu'on trouve une adresse, on doit retourner : "\textbf{<type\_reseau>:<nom>}".
Par exemple, "\textbf{facebook:ESI.Page}".
On veut seulement récupérer certaines adresses :
\begin{itemize}
	\item Pour Linkedin, on récupère seulement les réseaux sociaux de type "company". 
	
	Par exemple, "https://www.linkedin.com/company/airalgerie".
	
	\item Pour Youtube, on récupère seulement les adresses de type "channel".
	
	Par exemple, "https://www.youtube.com/channel/UC4pwuVraHCCElwQ1dIl1g8Q".
\end{itemize}

\subsection*{1.2. Téléphones}

Il existe plusieurs formes des numéros de téléphones. 
Par exemple : (023) 93 91 32 ; 023 93 91 32 ; 023 93.91.32 ; 023 93.91. 32 ; 023939132 ;
+213 23 93 91 32 ; + 213 23 93 91 32 ; 00 213 23 93 91 32 ; +213 (0) 23 93 91 32 ; +213 023 93 91 32 ; 023 93-91-32 ; ...
Cette liste n'est pas complète ; On doit examiner les fichier où le système a échoué.
La forme voulue est : "(0XX) XX XX XX" où X est un chiffre. 
Pour les numéros de téléphones portables, la forme sera : "(0XXX) XX XX XX".
Il y a quelques conditions : 
\begin{itemize}
	\item Pour les numéros de la forme (023) 93 91 32/41, on doit récupérer les numéros : 023) 93 91 32 et 023) 93 91 41. 
	Il peut y avoir des espaces entre les chiffres et le /.
	\item La même chose pour les numéros de la forme : (023) 93 91 32 à 41.
	\item Une autre forme est (023) 93 91 32 ou 41 ou 51. Dans cet exemple, on doit extraire 3 numéros où les deux derniers chiffres sont remplacés par les autres après "ou".
\end{itemize}

\subsection*{1.3. Adresses émail}

Il n'y a pas de conditions sur les adresses émail.

\section*{2. Spécifications techniques}

\begin{itemize}
	\item Langages de programmation : Java, Javascript (nodejs) ou Python (au choix).  
	Le programme qui fait la lecture et l'évaluation a été totalement implémenté.
	\item Il faut introduire les expressions régulières et la façon de remplacement.
	On peut introduire plusieurs, mais il faut utiliser un minimum des règles (sinon, on va avoir plusieurs passes et donc plus de temps de traitement). 
	Par exemple, 
	\begin{itemize}
		\item Python : (re.compile(u'(\textbackslash w+)@(\textbackslash w+)\textbackslash .(\textbackslash w+)'), '\$1@\$2.\$3')
		\item Javascript : {reg: /(\textbackslash w+)@(\textbackslash w+)\textbackslash.(\textbackslash w+)/g, pat: '\$1@\$2.\$3'}
		\item Java : mails\_remp.put(Pattern.compile("(\textbackslash\textbackslash w+)@(\textbackslash\textbackslash w+)\textbackslash\textbackslash.(\textbackslash\textbackslash w+)"), "\$1@\$2.\$3")
	\end{itemize}
	\item Fichiers : html ; Nombre : 20. Le fichier "ref.txt" contient les résultats attendus.
\end{itemize}

\section*{3. Procédure d'évaluation}

\begin{itemize}
	\item Durée : 1h (il faut rendre le TP à la fin de la séance)
	\item Note = F1\_score * 20 - (0.25 * (nombre\_règles - 10), si ce nombre dépasse 10).
\end{itemize}

\end{document}
