% !TEX TS-program = pdflatex
% !TeX program = pdflatex
% !TEX encoding = UTF-8
% !TEX spellcheck = fr

\documentclass[xcolor=table]{beamer}


%\usepackage{fullpage}
%\usepackage[left=2.8cm,right=2.2cm,top=2 cm,bottom=2 cm]{geometry}
\setbeamersize{text margin left=10pt,text margin right=10pt}
\usepackage{amsmath,amssymb} 
\usepackage[T1]{fontenc}
\usepackage{textcomp}
\usepackage[utf8]{inputenc}
\usepackage[french]{babel}
\usepackage{arabtex}
\usepackage{txfonts}
\usepackage[]{graphicx}
\usepackage{multirow}
\usepackage{hyperref}
\usepackage{colortbl}
\usepackage{listingsutf8}
\usepackage{wrapfig}
\usepackage{multicol}
\usepackage[export]{adjustbox} %for images in table, also for frame
\usepackage[many]{tcolorbox}

\hypersetup{
	colorlinks,
	urlcolor = blue
}

%\renewcommand{\baselinestretch}{1.5}

\def\supit#1{\raisebox{0.8ex}{\small\it #1}\hspace{0.05em}}

\AtBeginSection{%
	\begin{frame}
		\sectionpage
	\end{frame}
}

\newcommand{\rottext}[2]{%
	\rotatebox{90}{%
	\begin{minipage}{#1}%
		\raggedleft#2%
	\end{minipage}%
	}%
}

\usepackage{longtable}
\usepackage{tabu}


\institute{ %
École  nationale Supérieure d'Informatique (ESI, ex. INI), Algérie
}
\author[ \textbf{\footnotesize\insertframenumber/\inserttotalframenumber} \hspace*{1.5cm} ESI - ARIES Abdelkrime (2020/2021)] %
{ARIES Abdelkrime}
%\titlegraphic{\includegraphics[height=1cm]{../img/esi-logo.png}%\hspace*{4.75cm}~


\date{Année universitaire : 2020/2021} %\today

\usetheme{Warsaw} % Antibes Boadilla Warsaw

\beamertemplatenavigationsymbolsempty

%\setbeamertemplate{headline}{}

\definecolor{lightblue}{HTML}{D0D2FF}
\definecolor{lightyellow}{HTML}{FFFFAA}
\definecolor{darkblue}{HTML}{0000BB}
\definecolor{olivegreen}{HTML}{006600}
\definecolor{violet}{HTML}{6600CC}

\newcommand{\keyword}[1]{\textcolor{red}{\bfseries\itshape #1}}
\newcommand{\expword}[1]{\textcolor{olivegreen}{#1}}
\newcommand{\optword}[1]{\textcolor{violet}{\bfseries #1}}

\makeatletter
\newcommand\mysphere{%
	\parbox[t]{10pt}{\raisebox{0.2pt}{\beamer@usesphere{item projected}{bigsphere}}}}
\makeatother

%\let\oldtabular\tabular
%\let\endoldtabular\endtabular
%\renewenvironment{tabular}{\rowcolors{2}{white}{lightblue}\oldtabular\rowcolor{blue}}{\endoldtabular}


\NoAutoSpacing %french autospacing after ":"

\def\graphpath{}

\newcommand{\changegraphpath}[1]{\def\graphpath{#1}}


\newcommand{\vgraphpage}[2][.84\textheight]{%
%	\begin{center}%
		\includegraphics[height=#1]{\graphpath #2}%
%	\end{center}%
}

\newcommand{\hgraphpage}[2][\textwidth]{%
%	\begin{center}%
		\includegraphics[width=#1]{\graphpath #2}%
%	\end{center}%
}

\newcommand{\graphpage}[2][]{%
	\includegraphics[#1]{\graphpath #2}%
}

\bibliographystyle{acm}

\newcommand{\insertbibliography}[2]{
	\appendix
	\section*{Bibliographie}
	\nocite{#2}
%	\makeatletter % to change template
%	\setbeamertemplate{headline}[default] % not mandatory, but I though it was better to set it blank
%	\def\beamer@entrycode{\vspace*{-\headheight}} % here is the part we are interested in :)
%	\makeatother
	\begin{multicols*}{2}[\frametitle{\insertsection} \usebeamertemplate*{frametitle}]%\usebeamertemplate*{frametitle}\frametitle{Références}
		\tiny
		\bibliography{#1}
	\end{multicols*}
}

\definecolor{my-grey}{RGB}{233, 233, 233}

\newcommand{\insertlicence}{
	\begin{frame}[plain]
	\frametitle{Licence : CC-BY 4.0}
%	\framesubtitle{Licence: CC-BY-NC 4.0}

	\begin{tcolorbox}[colback=cyan,
		colframe=cyan,  
		arc=0pt,outer arc=0pt,
		valign=top, 
		halign=center,
		width=\textwidth]
		
		\includegraphics[width=.5cm]{../img/licence/cc_icon_white_x2.png}
		\includegraphics[width=.5cm]{../img/licence/attribution_icon_white_x2.png}
		
		\color{white}
		\bfseries Attribution 4.0 International (CC BY 4.0) \\
		\tiny \url{https://creativecommons.org/licenses/by/4.0/deed.fr}
		
	\end{tcolorbox}\vspace{-.5cm}
	\begin{tcolorbox}[colback=my-grey,
		colframe=my-grey,  
		center, arc=0pt,outer arc=0pt,
		valign=top, 
		halign=left,
		width=\textwidth]
		
		\tiny
		
		\begin{center}
			\bfseries\Large
			Vous êtes autorisé à :
		\end{center}
		
		\begin{minipage}{0.83\textwidth}
			\begin{itemize}
				\item[] \textbf{Partager} — copier, distribuer et communiquer le matériel par tous moyens et sous tous formats
				\item[] \textbf{Adapter} — remixer, transformer et créer à partir du matériel
				pour toute utilisation, y compris commerciale.
			\end{itemize}
		\end{minipage}
		\begin{minipage}{0.15\textwidth}
			\includegraphics[width=\textwidth]{../img/licence/FreeCulturalWorks_seal_x2.jpg}
		\end{minipage}
	
		
		\begin{center}
			\bfseries\Large
			Selon les conditions suivantes :
		\end{center}
		
		\begin{itemize}
			\item[] \textbf{Attribution} — Vous devez créditer l'Œuvre, intégrer un lien vers la licence et indiquer si des modifications ont été effectuées à l'Oeuvre. Vous devez indiquer ces informations par tous les moyens raisonnables, sans toutefois suggérer que l'Offrant vous soutient ou soutient la façon dont vous avez utilisé son Oeuvre. 
			\item[] \textbf{Pas de restrictions complémentaires} — Vous n'êtes pas autorisé à appliquer des conditions légales ou des mesures techniques qui restreindraient légalement autrui à utiliser l'Oeuvre dans les conditions décrites par la licence.
		\end{itemize}
		
	\end{tcolorbox}
	
%	\begin{center}
%		\bfseries Attribution 4.0 International (CC BY 4.0)
%		\url{https://creativecommons.org/licenses/by/4.0/deed.fr}
%	\end{center}

%	\tiny
%
%	Vous êtes autorisé à : 
%	\begin{itemize}
%		\item \textbf{Partager} — copier, distribuer et communiquer le matériel par tous moyens et sous tous formats
%		\item \textbf{Adapter} — remixer, transformer et créer à partir du matériel
%	\end{itemize}
%	
%	Selon les conditions suivantes : 
%	\begin{itemize}
%		\item \textbf{Attribution} — Vous devez créditer l'Œuvre, intégrer un lien vers la licence et indiquer si des modifications ont été effectuées à l'Oeuvre. Vous devez indiquer ces informations par tous les moyens raisonnables, sans toutefois suggérer que l'Offrant vous soutient ou soutient la façon dont vous avez utilisé son Oeuvre.
%		\item \textbf{Pas d'Utilisation Commerciale} — Vous n'êtes pas autorisé à faire un usage commercial de cette Oeuvre, tout ou partie du matériel la composant. 
%		\item \textbf{Pas de restrictions complémentaires} — Vous n'êtes pas autorisé à appliquer des conditions légales ou des mesures techniques qui restreindraient légalement autrui à utiliser l'Oeuvre dans les conditions décrites par la licence.
%	\end{itemize}

	\end{frame}
}

\settowidth{\leftmargini}{\usebeamertemplate{itemize item}}
\addtolength{\leftmargini}{\labelsep}

\AtBeginDocument{
	\newcolumntype{L}[2]{>{\vbox to #2\bgroup\vfill\flushleft}p{#1}<{\egroup}} 
	
	\begin{frame}[plain]
		\maketitle
	\end{frame}

	\insertlicence
}


% needs etoolbox; to break links after -
\appto\UrlBreaks{\do\-}


\lstdefinelanguage{CSS}{
	keywords={accelerator,azimuth,background,background-attachment,
		background-color,background-image,background-position,
		background-position-x,background-position-y,background-repeat,
		behavior,border,border-bottom,border-bottom-color,
		border-bottom-style,border-bottom-width,border-collapse,
		border-color,border-left,border-left-color,border-left-style,
		border-left-width,border-right,border-right-color,
		border-right-style,border-right-width,border-spacing,
		border-style,border-top,border-top-color,border-top-style,
		border-top-width,border-width,bottom,caption-side,clear,
		clip,color,content,counter-increment,counter-reset,cue,
		cue-after,cue-before,cursor,direction,display,elevation,
		empty-cells,filter,float,font,font-family,font-size,
		font-size-adjust,font-stretch,font-style,font-variant,
		font-weight,height,ime-mode,include-source,
		layer-background-color,layer-background-image,layout-flow,
		layout-grid,layout-grid-char,layout-grid-char-spacing,
		layout-grid-line,layout-grid-mode,layout-grid-type,left,
		letter-spacing,line-break,line-height,list-style,
		list-style-image,list-style-position,list-style-type,margin,
		margin-bottom,margin-left,margin-right,margin-top,
		marker-offset,marks,max-height,max-width,min-height,
		min-width,-moz-binding,-moz-border-radius,
		-moz-border-radius-topleft,-moz-border-radius-topright,
		-moz-border-radius-bottomright,-moz-border-radius-bottomleft,
		-moz-border-top-colors,-moz-border-right-colors,
		-moz-border-bottom-colors,-moz-border-left-colors,-moz-opacity,
		-moz-outline,-moz-outline-color,-moz-outline-style,
		-moz-outline-width,-moz-user-focus,-moz-user-input,
		-moz-user-modify,-moz-user-select,orphans,outline,
		outline-color,outline-style,outline-width,overflow,
		overflow-X,overflow-Y,padding,padding-bottom,padding-left,
		padding-right,padding-top,page,page-break-after,
		page-break-before,page-break-inside,pause,pause-after,
		pause-before,pitch,pitch-range,play-during,position,quotes,
		-replace,richness,right,ruby-align,ruby-overhang,
		ruby-position,-set-link-source,size,speak,speak-header,
		speak-numeral,speak-punctuation,speech-rate,stress,
		scrollbar-arrow-color,scrollbar-base-color,
		scrollbar-dark-shadow-color,scrollbar-face-color,
		scrollbar-highlight-color,scrollbar-shadow-color,
		scrollbar-3d-light-color,scrollbar-track-color,table-layout,
		text-align,text-align-last,text-decoration,text-indent,
		text-justify,text-overflow,text-shadow,text-transform,
		text-autospace,text-kashida-space,text-underline-position,top,
		unicode-bidi,-use-link-source,vertical-align,visibility,
		voice-family,volume,white-space,widows,width,word-break,
		word-spacing,word-wrap,writing-mode,z-index,zoom},  
	sensitive=true,
	morecomment=[l]{//},
	morecomment=[s]{/*}{*/},
	morestring=[b]',
	morestring=[b]",
	alsoletter={:},
	alsodigit={-}
}
\lstdefinelanguage{HTML5}{
	language=html,
	sensitive=true, 
	alsoletter={<>=-},
	otherkeywords={
		% HTML tags
		<, </, >,
		</a, <a, </a>,
		</abbr, <abbr, </abbr>,
		</address, <address, </address>,
		</area, <area, </area>,
		</area, <area, </area>,
		</article, <article, </article>,
		</aside, <aside, </aside>,
		</audio, <audio, </audio>,
		</audio, <audio, </audio>,
		</b, <b, </b>,
		</base, <base, </base>,
		</bdi, <bdi, </bdi>,
		</bdo, <bdo, </bdo>,
		</blockquote, <blockquote, </blockquote>,
		</body, <body, </body>,
		</br, <br, </br>,
		</button, <button, </button>,
		</canvas, <canvas, </canvas>,
		</caption, <caption, </caption>,
		</cite, <cite, </cite>,
		</code, <code, </code>,
		</col, <col, </col>,
		</colgroup, <colgroup, </colgroup>,
		</data, <data, </data>,
		</datalist, <datalist, </datalist>,
		</dd, <dd, </dd>,
		</del, <del, </del>,
		</details, <details, </details>,
		</dfn, <dfn, </dfn>,
		</div, <div, </div>,
		</dl, <dl, </dl>,
		</dt, <dt, </dt>,
		</em, <em, </em>,
		</embed, <embed, </embed>,
		</fieldset, <fieldset, </fieldset>,
		</figcaption, <figcaption, </figcaption>,
		</figure, <figure, </figure>,
		</footer, <footer, </footer>,
		</form, <form, </form>,
		</h1, <h1, </h1>,
		</h2, <h2, </h2>,
		</h3, <h3, </h3>,
		</h4, <h4, </h4>,
		</h5, <h5, </h5>,
		</h6, <h6, </h6>,
		</head, <head, </head>,
		</header, <header, </header>,
		</hr, <hr, </hr>,
		</html, <html, </html>,
		</i, <i, </i>,
		</iframe, <iframe, </iframe>,
		</img, <img, </img>,
		</input, <input, </input>,
		</ins, <ins, </ins>,
		</kbd, <kbd, </kbd>,
		</keygen, <keygen, </keygen>,
		</label, <label, </label>,
		</legend, <legend, </legend>,
		</li, <li, </li>,
		</link, <link, </link>,
		</main, <main, </main>,
		</map, <map, </map>,
		</mark, <mark, </mark>,
		</math, <math, </math>,
		</menu, <menu, </menu>,
		</menuitem, <menuitem, </menuitem>,
		</meta, <meta, </meta>,
		</meter, <meter, </meter>,
		</nav, <nav, </nav>,
		</noscript, <noscript, </noscript>,
		</object, <object, </object>,
		</ol, <ol, </ol>,
		</optgroup, <optgroup, </optgroup>,
		</option, <option, </option>,
		</output, <output, </output>,
		</p, <p, </p>,
		</param, <param, </param>,
		</pre, <pre, </pre>,
		</progress, <progress, </progress>,
		</q, <q, </q>,
		</rp, <rp, </rp>,
		</rt, <rt, </rt>,
		</ruby, <ruby, </ruby>,
		</s, <s, </s>,
		</samp, <samp, </samp>,
		</script, <script, </script>,
		</section, <section, </section>,
		</select, <select, </select>,
		</small, <small, </small>,
		</source, <source, </source>,
		</span, <span, </span>,
		</strong, <strong, </strong>,
		</style, <style, </style>,
		</summary, <summary, </summary>,
		</sup, <sup, </sup>,
		</svg, <svg, </svg>,
		</table, <table, </table>,
		</tbody, <tbody, </tbody>,
		</td, <td, </td>,
		</template, <template, </template>,
		</textarea, <textarea, </textarea>,
		</tfoot, <tfoot, </tfoot>,
		</th, <th, </th>,
		</thead, <thead, </thead>,
		</time, <time, </time>,
		</title, <title, </title>,
		</tr, <tr, </tr>,
		</track, <track, </track>,
		</u, <u, </u>,
		</ul, <ul, </ul>,
		</var, <var, </var>,
		</video, <video, </video>,
		</wbr, <wbr, </wbr>,
		/>, <!
	},  
	ndkeywords={
		% General
		=,
		% HTML attributes
		accept=, accept-charset=, accesskey=, action=, align=, alt=, async=, autocomplete=, autofocus=, autoplay=, autosave=, bgcolor=, border=, buffered=, challenge=, charset=, checked=, cite=, class=, code=, codebase=, color=, cols=, colspan=, content=, contenteditable=, contextmenu=, controls=, coords=, data=, datetime=, default=, defer=, dir=, dirname=, disabled=, download=, draggable=, dropzone=, enctype=, for=, form=, formaction=, headers=, height=, hidden=, high=, href=, hreflang=, http-equiv=, icon=, id=, ismap=, itemprop=, keytype=, kind=, label=, lang=, language=, list=, loop=, low=, manifest=, max=, maxlength=, media=, method=, min=, multiple=, name=, novalidate=, open=, optimum=, pattern=, ping=, placeholder=, poster=, preload=, pubdate=, radiogroup=, readonly=, rel=, required=, reversed=, rows=, rowspan=, sandbox=, scope=, scoped=, seamless=, selected=, shape=, size=, sizes=, span=, spellcheck=, src=, srcdoc=, srclang=, start=, step=, style=, summary=, tabindex=, target=, title=, type=, usemap=, value=, width=, wrap=,
		% CSS properties
		accelerator:,azimuth:,background:,background-attachment:,
		background-color:,background-image:,background-position:,
		background-position-x:,background-position-y:,background-repeat:,
		behavior:,border:,border-bottom:,border-bottom-color:,
		border-bottom-style:,border-bottom-width:,border-collapse:,
		border-color:,border-left:,border-left-color:,border-left-style:,
		border-left-width:,border-right:,border-right-color:,
		border-right-style:,border-right-width:,border-spacing:,
		border-style:,border-top:,border-top-color:,border-top-style:,
		border-top-width:,border-width:,bottom:,caption-side:,clear:,
		clip:,color:,content:,counter-increment:,counter-reset:,cue:,
		cue-after:,cue-before:,cursor:,direction:,display:,elevation:,
		empty-cells:,filter:,float:,font:,font-family:,font-size:,
		font-size-adjust:,font-stretch:,font-style:,font-variant:,
		font-weight:,height:,ime-mode:,include-source:,
		layer-background-color:,layer-background-image:,layout-flow:,
		layout-grid:,layout-grid-char:,layout-grid-char-spacing:,
		layout-grid-line:,layout-grid-mode:,layout-grid-type:,left:,
		letter-spacing:,line-break:,line-height:,list-style:,
		list-style-image:,list-style-position:,list-style-type:,margin:,
		margin-bottom:,margin-left:,margin-right:,margin-top:,
		marker-offset:,marks:,max-height:,max-width:,min-height:,
		min-width:,transition-duration:,transition-property:,
		transition-timing-function:,transform:,
		-moz-transform:,-moz-binding:,-moz-border-radius:,
		-moz-border-radius-topleft:,-moz-border-radius-topright:,
		-moz-border-radius-bottomright:,-moz-border-radius-bottomleft:,
		-moz-border-top-colors:,-moz-border-right-colors:,
		-moz-border-bottom-colors:,-moz-border-left-colors:,-moz-opacity:,
		-moz-outline:,-moz-outline-color:,-moz-outline-style:,
		-moz-outline-width:,-moz-user-focus:,-moz-user-input:,
		-moz-user-modify:,-moz-user-select:,orphans:,outline:,
		outline-color:,outline-style:,outline-width:,overflow:,
		overflow-X:,overflow-Y:,padding:,padding-bottom:,padding-left:,
		padding-right:,padding-top:,page:,page-break-after:,
		page-break-before:,page-break-inside:,pause:,pause-after:,
		pause-before:,pitch:,pitch-range:,play-during:,position:,quotes:,
		-replace:,richness:,right:,ruby-align:,ruby-overhang:,
		ruby-position:,-set-link-source:,size:,speak:,speak-header:,
		speak-numeral:,speak-punctuation:,speech-rate:,stress:,
		scrollbar-arrow-color:,scrollbar-base-color:,
		scrollbar-dark-shadow-color:,scrollbar-face-color:,
		scrollbar-highlight-color:,scrollbar-shadow-color:,
		scrollbar-3d-light-color:,scrollbar-track-color:,table-layout:,
		text-align:,text-align-last:,text-decoration:,text-indent:,
		text-justify:,text-overflow:,text-shadow:,text-transform:,
		text-autospace:,text-kashida-space:,text-underline-position:,top:,
		unicode-bidi:,-use-link-source:,vertical-align:,visibility:,
		voice-family:,volume:,white-space:,widows:,width:,word-break:,
		word-spacing:,word-wrap:,writing-mode:,z-index:,zoom:
	},  
	morecomment=[s]{<!--}{-->},
	tag=[s]
}

%\usepackage{color}
%\definecolor{editorGray}{rgb}{0.95, 0.95, 0.95}
%\definecolor{editorOcher}{rgb}{1, 0.5, 0} % #FF7F00 -> rgb(239, 169, 0)
%\definecolor{editorGreen}{rgb}{0, 0.5, 0} % #007C00 -> rgb(0, 124, 0)
%
%\lstset{%
%	% Basic design
%	backgroundcolor=\color{editorGray},
%	basicstyle={\small\ttfamily},   
%	frame=l,
%	% Line numbers
%	xleftmargin={0.75cm},
%	numbers=left,
%	stepnumber=1,
%	firstnumber=1,
%	numberfirstline=true,
%	% Code design   
%	keywordstyle=\color{blue}\bfseries,
%	commentstyle=\color{darkgray}\ttfamily,
%	ndkeywordstyle=\color{editorGreen}\bfseries,
%	stringstyle=\color{editorOcher},
%	% Code
%	language=HTML5,
%	alsodigit={.:;},
%	tabsize=2,
%	showtabs=false,
%	showspaces=false,
%	showstringspaces=false,
%	extendedchars=true,
%	breaklines=true,        
%}

\lstset{language=CSS,
	basicstyle=\ttfamily,
	keywordstyle=\color{blue}\ttfamily,
	stringstyle=\color{red}\ttfamily,
	commentstyle=\color{green}\ttfamily,
	morecomment=[l][\color{magenta}]{\#}
}

\title[TALN : 02- Traitements basiques]%
{Traitement automatique du langage naturel\\Chapitre 02 : Traitements basiques du texte} 

\changegraphpath{../img/basique/}

\begin{document}
	
\begin{frame}
\frametitle{Traitement automatique du langage naturel}
\framesubtitle{Traitements basiques du texte : Introduction}

\begin{itemize}
	\item 
\end{itemize}
%\begin{minipage}{0.78\textwidth}
%\begin{itemize}
%	\item Un domaine multidisciplinaire
%	\begin{itemize}
%		\item \optword{Linguistique} : étude du langage
%		\item \optword{Informatique} : Traitement automatique de l'information
%		\item \optword{Intelligence artificielle} : Ensemble de théories et de techniques mises en œuvre en vue de réaliser des machines capables de simuler l'intelligence humaine.
%	\end{itemize}
%\end{itemize}
%\end{minipage}
%\begin{minipage}{0.20\textwidth}
%\hgraphpage[\textwidth]{TALN.pdf}
%\end{minipage}
\end{frame}

\begin{frame}
\frametitle{Traitement automatique du langage naturel}
\framesubtitle{Traitements basiques du texte : Un peu d'humour}

%\begin{center}
%	\vgraphpage{humour.jpg}
%\end{center}

\end{frame}

\begin{frame}
\frametitle{Traitement automatique du langage naturel}
\framesubtitle{Traitements basiques du texte : Plan}

\begin{multicols}{2}
%	\small
\tableofcontents
\end{multicols}
\end{frame}

%===================================================================================
\section{Caractères}
%===================================================================================

\begin{frame}
\frametitle{Traitements basiques du texte}
\framesubtitle{Caractères}

\begin{itemize}
	\item \optword{Expressions régulières} : utilisées pour chercher des chaînes de caractères dans un texte
	\begin{itemize}
		\item reconnaissent des langages de types 3 (langages réguliers) dans la hiérarchie de Chomsky 
		\item utiles pour l'analyse lexicale (séparation des mots)
	\end{itemize}
	\item \optword{Distance d'édition} : utilisée pour mesurer la différence entre deux chaînes de caractères 
	\begin{itemize}
		\item utiles pour la recherche approximative
	\end{itemize}
\end{itemize}

\end{frame}

\subsection{Expressions régulières}

\begin{frame}
\frametitle{Traitements basiques du texte : Caractères}
\framesubtitle{Expressions régulières : Règles (1)}

\rowcolors{2}{lightblue}{lightyellow}
\begin{tabular}{p{.1\textwidth}p{.34\textwidth}p{.46\textwidth}}
	\rowcolor{darkblue}
	\textcolor{white}{ER} & \textcolor{white}{Sens} & \textcolor{white}{Exemple} \\
	
	. & n'importe quel caractère & \keyword{beg.n} : I \expword{begun} at the \expword{begin}ning. \\
	
	 \empty [aeuio] & caractères spécifiques & \keyword{[Ll][ae]} : \expword{Le} chat mange \expword{la} sourie. \\
	 
	\empty [a-e] & plage de caractères & \keyword{[A-Z]..} : \expword{J'a}i vu \expword{Kar}im. \\
	
	\empty [\textasciicircum aeuio] & exclure des caractères & \keyword{[\textasciicircum A-Z]a.} : J\expword{'ai} vu Karim. \\
	
	c? & un ou zéro & \keyword{colou?r} : It is \expword{colour} or \expword{color}. \\
	
	c* & zéro ou plus & \keyword{No*n} : \expword{Nn}! \expword{Non}! \expword{Nooooooon}! \\
	
	c+ & un ou plus & \keyword{No+n} : Nn! \expword{Non}! \expword{Nooooooon}! \\
	
	c\{n\} & n occurrences & \keyword{No\{3\}n} : Nn! Non! Noon! \expword{Nooon}! \\
	
	c\{n,m\} & de n à m occurrences & \keyword{No\{1,2\}n} : Nn! \expword{Non}! \expword{Noon}! Nooon! \\
	
	c\{n,\} & au moins n occurrences & \keyword{No\{2,\}n} : Nn! Non! \expword{Noon}! \expword{Nooon}! \\
	
	c\{,m\} & au plus m occurrences & \keyword{No\{,2\}n} : \expword{Nn}! \expword{Non}! \expword{Noon}! Nooon! \\
	
\end{tabular}

\end{frame}

\begin{frame}
\frametitle{Traitements basiques du texte : Caractères}
\framesubtitle{Expressions régulières : Règles (2)}

\begin{minipage}{.6\textwidth}
\begin{itemize}
	\item pour le groupement, utiliser \keyword{( )}
	\begin{itemize}
		\item Exemple, \expword{/(bla)+/} : Ceci est du \expword{blabla}.
	\end{itemize}
	\item pour la disjonction, utiliser \keyword{\textbar}
	\begin{itemize}
		\item Exemple, \expword{/continu(er\textbar ation\textbar el(le)?s?)/}
	\end{itemize}
	\item pour le début du texte, utiliser \keyword{\textasciicircum}
	\begin{itemize}
		\item Exemple, \expword{/\textasciicircum K/} :  \expword{K}ill Karim.
	\end{itemize}
	\item pour la fin du texte, utiliser \keyword{\$}
	\begin{itemize}
		\item Exemple, \expword{/\textbackslash .[\textasciicircum .]+\$/} :  fichier.tar\expword{.gz}
	\end{itemize}
	\item pour capturer un groupe, utiliser le regroupement avec \keyword{\$n} ou \keyword{\textbackslash n} où \expword{n} est la position du groupe
	\begin{itemize}
		\item Exemple, \expword{/(.*)(ement\textbar ation)\$/\textbackslash 1er/}
	\end{itemize}
\end{itemize}
\end{minipage}
\begin{minipage}{.38\textwidth}
\rowcolors{2}{lightblue}{lightyellow}
\begin{tabular}{p{.2\textwidth}p{.6\textwidth}}
	\rowcolor{darkblue}
	\textcolor{white}{ER} & \textcolor{white}{Équivalence} \\
	
	\textbackslash d & [0-9] \\
	\textbackslash D & [\textasciicircum 0-9] \\
	\textbackslash w & [a-zA-Z0-9\_] \\
	\textbackslash W & [\textasciicircum \textbackslash w] \\
	\textbackslash s & [ \textbackslash r\textbackslash t\textbackslash n\textbackslash f] \\
	\textbackslash S & [\textasciicircum \textbackslash s] \\
\end{tabular}

\end{minipage}

\end{frame}

\begin{frame}
\frametitle{Traitements basiques du texte : Caractères}
\framesubtitle{Expressions régulières : Cas d'utilisation}

\begin{itemize}
	\item Les éditeurs de texte utilisent les expressions régulières (ER) pour la recherche et le remplacement
	\item La plupart des langages de programmation introduisent des mécanismes pour utiliser les ER
	\item Extraction de données : par exemple, \expword{extraire les émails et les numéros de téléphones à partir des blogs et des réseaux sociaux}
	\item \url{https://github.com/kariminf/aruudy}
	\begin{itemize}
		\item \textit{One of my crazy projects}
		\item Détection du mètre d'une verse des poèmes arabes
		\item Avantage : les règles sont lisibles
		\item Inconvénient : plusieurs passes pour traiter une verse 
	\end{itemize}
\end{itemize}

\end{frame}

\begin{frame}
\frametitle{Traitements basiques du texte : Caractères}
\framesubtitle{Expressions régulières : Un peu d'humour}

\begin{center}
	\vgraphpage{humour-regex.jpeg}
\end{center}

\end{frame}

\subsection{Distance d'édition}

\begin{frame}
\frametitle{Traitements basiques du texte : Caractères}
\framesubtitle{Distance d'édition : Opérations d'édition}

\begin{itemize}
	\item \optword{Insertion} : insertion d'un caractère dans une chaîne\\
	$uv \rightarrow uxv \,/\, u, v \in X^*;\, uv \in X^+;\, x \in X$
	\begin{itemize}
		\item \expword{courir $ \rightarrow $ courrir, entraînement $ \rightarrow $ entraînnement }
	\end{itemize}
	
	\item \optword{Suppression} : suppression d'un caractère d'une chaîne\\
	$uxv \rightarrow uv \,/\, u, v \in X^*;\, uv \in X^+;\, x \in X$
	\begin{itemize}
		\item \expword{héros $ \rightarrow $ héro, meilleur $ \rightarrow $ meileur}
	\end{itemize}
	
	\item \optword{Substitution} : substitution d'un caractère par un autre\\
	$uxv \rightarrow uyv \,/\, u, v \in X^*;\, x, y \in X;\, x \ne y$
	\begin{itemize}
		\item \expword{cela $ \rightarrow $ celà, croient $ \rightarrow $ croyent }
	\end{itemize}
	
	\item \optword{Transposition} : changement de l'ordre de deux caractères\\
	$uxwyv \rightarrow uywxv \,/\, u, v, w \in X^*;\, x, y \in X;\, x \ne y$
	\begin{itemize}
		\item \expword{cueillir $ \rightarrow $ ceuillir}
	\end{itemize}
\end{itemize}

\end{frame}

\begin{frame}
\frametitle{Traitements basiques du texte : Caractères}
\framesubtitle{Distance d'édition : Algorithmes}

\begin{itemize}
	\item \optword{Distance de Hamming} : permet seulement la substitution. Les chaînes doivent être de même longueur. 
	\item \optword{Plus longue sous-séquence commune} : permet l'insertion et la suppression
	\item \optword{Distance de Levenshtein} : permet l'insertion, la suppression et la substitution
	\item \optword{Distance de Jaro} : permet la transposition
	\item \optword{Distance de Damerau–Levenshtein} : permet l'insertion, la suppression, la substitution et la transposition entre deux caractères adjacents
\end{itemize}

\end{frame}

\begin{frame}
\frametitle{Traitements basiques du texte : Caractères}
\framesubtitle{Distance d'édition : Distance de Levenshtein}

\begin{itemize}
	\item X est une chaîne source de longueur $n$
	\item Y une chaîne destinataire de longueur $m$
	\item D est une matrice où D[i, j] est la distance d'édition entre les sous-chaînes X[1..i] et Y[1..j]
	\item pour calculer $D[n, m]$, on utilise la programmation dynamique
	\item $D[0, 0] = 0$
\end{itemize}

\[
D[i, j] = \min 
\begin{cases}
D[i - 1, j] + 1 \text{ //Suppression}\\
D[i, j-1] + 1 \text{ //Insertion}\\
D[i-1, j-1] + \begin{cases}
2 & \text{si } X[i] \ne Y[j] \\
0 & \text{sinon}
\end{cases}
\end{cases}
\]

\end{frame}

\begin{frame}
\frametitle{Traitements basiques du texte : Caractères}
\framesubtitle{Distance d'édition : Distance de Levenshtein (exemple)}

\begin{figure}
	\centering
	\hgraphpage{levenshtein-exp.pdf}
	\caption{Exemple de calcul de distance de Levenshtein \cite{2019-jurafsky-martin}}
\end{figure}

\end{frame}

\begin{frame}
\frametitle{Traitements basiques du texte : Caractères}
\framesubtitle{Distance d'édition : Quelques applications}

\begin{itemize}
	\item Révision des fichiers : par exemple, la commande Unix \expword{diff} qui compare entre deux fichiers.
	\item Correction d'orthographe : suggérer des corrections possibles d'une faute (ex. \expword{Hunspell}).
	\item Détection du plagiat : ici, on utilise des mots à la place des caractères.
	\item Filtrage de spam : parfois, les spammeurs commettent des fautes d'orthographe intentionnellement pour tromper l'outil de détection de spam.
	\item Bio-informatique : quantification de la similarité entre deux séquences d'ADN.
\end{itemize}

\end{frame}

\begin{frame}
\frametitle{Traitements basiques du texte : Caractères}
\framesubtitle{Distance d'édition : Un peu d'humour}

\begin{center}
	\vgraphpage[.4\textheight]{humour-spell.jpg}
	\vgraphpage[.4\textheight]{humour-spell1.jpg}
	\vgraphpage[.4\textheight]{humour-spell2.jpg}
\end{center}

\end{frame}

%===================================================================================
\section{Segmentation du texte}
%===================================================================================

\begin{frame}
\frametitle{Traitements basiques du texte}
\framesubtitle{Segmentation du texte}

\begin{itemize}
	\item Un nouveau paragraphe est marqué par un retour à la ligne ou en utilisant la balise \keyword{\textless p\textgreater} en HTML.
	\begin{itemize}
		\item Lorsqu'on utilise un outil pour extraire du texte à partir des pdfs, on aura plusieurs retours à la ligne 
	\end{itemize}
	\item Une nouvelle phrase est marquée par un point (ou une autre marque)
	\begin{itemize}
		\item Le point n'est pas seulement utilisé pour marquer une phrase
	\end{itemize}
	\item Un mot est délimité par une espace 
	\begin{itemize}
		\item Pas toujours 
	\end{itemize}
\end{itemize}

\end{frame}

\subsection{Délimitation de la phrase}

\begin{frame}
\frametitle{Traitements basiques du texte : Segmentation du texte}
\framesubtitle{Délimitation de la phrase : Problématique} 

\begin{itemize}
	\item \expword{/[.?!]/} est une ER simple utilisée pour délimiter les phrases (français, anglais, etc.)
	\item Le point est utilisé dans les nombres : \expword{123,456.78 (style américain) 123.456,78 (style européen)}
	\item Les abréviations contiennent des points
	\item Des phrases longues sont difficiles à traiter ; il faut mieux les séparer 
	\item Les phrases entre guillemets
	\item Des langues, comme le thaï, n'utilisent pas des marqueurs pour séparer les phrases  
\end{itemize}

\end{frame}

\begin{frame}
\frametitle{Traitements basiques du texte : Segmentation du texte}
\framesubtitle{Délimitation de la phrase : Facteurs de segmentation} 

\begin{itemize}
	\item La casse : les phrases commencent par un majuscule
	\item Catégorie grammaticale
	\item Longueur du mot 
	\item Préfixes et suffixes 
	\item Classes des abréviations 
	\item Noms propres 
\end{itemize}

\end{frame}

\begin{frame}
\frametitle{Traitements basiques du texte : Segmentation du texte}
\framesubtitle{Délimitation de la phrase : Un peu d'humour} 

\hgraphpage{humour-segmentation.jpg}

\end{frame}

\subsection{Séparation des mots}

\begin{frame}
\frametitle{Traitements basiques du texte : Segmentation du texte}
\framesubtitle{Séparation des mots : Problématique}

\begin{itemize}
	\item \expword{/[ ]+/} est une ER simple utilisée pour séparer les mots (arabe, français, anglais, etc.)
	\item Des langues, comme le japonais, n'utilisent pas de marques pour séparer les mots (\expword{\begin{CJK}{UTF8}{min}今年は本当に忙しかったです。\end{CJK}})
	\item Il existe des mots composés : par attachement (allemand : ``\expword{Lebensversicherung}, assurance vie"; arabe : ``\expword{\RL{y_htbrwnhm}}, ils les testent") ou par trait d'union (\expword{va-t-il, c-à-dire})
	\item Confusion entre des caractères : l'apostrophe est utilisée pour la citation, la contraction (\expword{She's, J'ai}) 
	\item Les expressions avec des mots multiples : \expword{les expressions numériques comme les dates}
\end{itemize}

\end{frame}

\begin{frame}
\frametitle{Traitements basiques du texte : Segmentation du texte}
\framesubtitle{Séparation des mots : Approches}

\begin{itemize}
	\item Par règles : en utilisant des expressions régulières 
	\begin{itemize}
		\item \url{https://www.nltk.org/api/nltk.tokenize.html}
		\item \url{https://nlp.stanford.edu/software/tokenizer.shtml}
		\item \url{https://spacy.io/}
		\item \url{https://github.com/kariminf/jslingua}
		\item \url{https://github.com/linuxscout/pyarabic}
	\end{itemize}
	\item Statistique : en utilisant un modèle de langue pour calculer la probabilité qu'un caractère marque la  fin d'un mot 
	\begin{itemize}
		\item \url{https://nlp.stanford.edu/software/segmenter.html}
		\item \url{https://opennlp.apache.org/}
	\end{itemize}
\end{itemize}

\end{frame}

\begin{frame}
\frametitle{Traitements basiques du texte : Segmentation du texte}
\framesubtitle{Séparation des mots : Un peu d'humour} 

\begin{center}
	\vgraphpage{humour-punctuation.jpg}
\end{center}

\end{frame}


%===================================================================================
\section{Normalisation et filtrage du texte}
%===================================================================================

\begin{frame}
\frametitle{Traitements basiques du texte}
\framesubtitle{Normalisation et filtrage du texte}

\begin{itemize}
	\item \optword{Normalisation du texte}
	\begin{itemize}
		\item PROBLÉMATIQUE : Un texte peut contenir des variations du même terme. Aussi, dans des tâches comme la recherche d'information, on n'a pas besoin d'avoir le contenu exacte du texte.
		\item SOLUTION : transformer le texte à une forme canonique 
		\item \url{https://www.kaggle.com/c/text-normalization-challenge-english-language}
	\end{itemize}
	\item \optword{Filtrage du texte}
	\begin{itemize}
		\item PROBLÉMATIQUE : le texte peut contenir des caractères, des mots et des expressions qui peuvent entraver son traitement
		\item SOLUTION : suppression
	\end{itemize}
\end{itemize}

\end{frame}

\subsection{Normalisation du texte}

\begin{frame}
\frametitle{Traitements basiques du texte : Normalisation et filtrage}
\framesubtitle{Normalisation du texte (1)}

\begin{itemize}
%	\item \optword{Tâche} : transformer le texte à une forme canonique
	
	\item acronymes et les abréviations 
	\begin{itemize}
		\item forme standard : \expword{US \textrightarrow USA, U.S.A. \textrightarrow USA}
		\item version longue : \expword{M. \textrightarrow Monsieur}
	\end{itemize}
	
	\item formater les valeurs comme les dates et les nombres de la même façon
	\begin{itemize}
		\item conversion à la forme textuelle : \expword{1205 DZD \textrightarrow Mille deux cents cinq dinars algériens}
		\item format spécifique : \expword{12 Janvier 1986, 12.01.86 \textrightarrow 1986-01-12}
		\item remplacement par le type : \expword{12 Janvier 1986 \textrightarrow DATE, kariminfo0@gmail.com \textrightarrow EMAIL}
	\end{itemize}
	
	\item transformer les majuscules en minuscules. 
	\begin{itemize}
		\item \expword{Texte \textrightarrow texte}
		\item Des fois, il faut laisser la casse telle qu'elle est, comme dans les noms propres (\expword{Will})
	\end{itemize}
\end{itemize}

\end{frame}

\begin{frame}
\frametitle{Traitements basiques du texte : Normalisation et filtrage}
\framesubtitle{Normalisation du texte (2)}

\begin{itemize}
	
	\item contractions
	\begin{itemize}
		\item \expword{y'll \textrightarrow you all, s'il \textrightarrow si il}
	\end{itemize}
	
	\item diacritiques
	\begin{itemize}
		\item désaccentuation :  \expword{système \textrightarrow systeme}
		\item dé-vocalisation :  \expword{\RL{yadrusu} \textrightarrow \RL{ydrs}}. Sauf dans les cas où on a besoin des diacritiques (poèmes)
	\end{itemize}

	\item encodage 
	\begin{itemize}
		\item il faut utiliser le même encodage supporté dans le traitement
	\end{itemize}

\end{itemize}

\end{frame}

\subsection{Filtrage du texte}

\begin{frame}
\frametitle{Traitements basiques du texte : Normalisation et filtrage}
\framesubtitle{Filtrage du texte}

\begin{itemize}
	\item Les caractères spéciaux comme les caractères non imprimables
	\begin{itemize}
		\item ils peuvent mener à des traitements erronés 
	\end{itemize}
	\item Les mots clés dans les formats textuelles
	\begin{itemize}
		\item les balises HTML, XML, etc. 
	\end{itemize}
	\item Les \keyword{mots vides} : les mots non significatifs comme les prépositions, articles et les pronoms.
	\begin{itemize}
		\item en anglais : \keyword{stop words}
		\item dans la recherche d'information, il est inutile de les indexer
		\item dans le résumé automatique, ces mots peuvent dégrader les scores des phrases
	\end{itemize}
\end{itemize}

\end{frame}

\begin{frame}
\frametitle{Traitements basiques du texte : Normalisation et filtrage}
\framesubtitle{Filtrage du texte : Un peu d'humour}

\begin{center}
	\vgraphpage{humour-stopwords.jpg}
\end{center}

\end{frame}

%===================================================================================
\section{Morphologie}
%===================================================================================

\begin{frame}
\frametitle{Traitements basiques du texte}
\framesubtitle{Morphologie}


\end{frame}

\subsection{Formation des mots}

\begin{frame}
\frametitle{Traitements basiques du texte : Morphologie}
\framesubtitle{Formation des mots}

\begin{itemize}
	\item Flexion : variation morphologique d'un mot selon les traits grammaticaux (nombre, genre, etc.)
	\begin{itemize}
		\item Conjugaison des verbes 
		\item Déclinaison des noms, pronoms, adjectifs et déterminants. 
		Ex. \RL{q.t} \textrightarrow \RL{q.t.t}, 
	\end{itemize}
	\item Dérivation : variation morphologique d'un mot pour créer un nouveau lexème ou pour changer de catégorie
	\begin{itemize}
		\item Créer unn nouveau lexème : \expword{couper \textrightarrow\ découper, \RL{`ml} \textrightarrow \RL{ist`ml}}
		\item Changer de catégorie : nominalisation (\expword{classer \textrightarrow classement, classeur ; \RL{darasa} \textrightarrow \RL{darsuN, madrasaTuN, mudarrisuN, dArisuN}}), l'adjectif (\expword{fatiguer \textrightarrow fatigant})
	\end{itemize}
\end{itemize}

\end{frame}

\begin{frame}
\frametitle{Traitements basiques du texte : Morphologie}
\framesubtitle{Formation des mots : Exemple (conjugaison automatique)}

\begin{itemize}
	\item Base de données
	\begin{itemize}
		\item stocker la conjugaison des verbes dans une base de données
		\item Exemple, conjugaison des verbes en arabe : \url{https://github.com/linuxscout/qutrub}
	\end{itemize}
	\item Modèles (\textit{template})
	\begin{itemize}
		\item stocker la conjugaison de certains verbes modèles et utiliser une liste pour indexer les verbes similaires
		\item Exemple, \expword{la conjugaison des verbes en français}
	\end{itemize}
	\item Règles
	\begin{itemize}
		\item utilisation des règles SI-SINON
		\item Exemple, conjugaison des verbes en arabe, anglais, français et japonais : \url{https://github.com/kariminf/jslingua}
	\end{itemize}
\end{itemize}

\end{frame}

\begin{frame}
\frametitle{Traitements basiques du texte : Morphologie}
\framesubtitle{Formation des mots : Un peu d'humour}

\begin{center}
	\vgraphpage{humour-formation.jpg}
\end{center}

\end{frame}

\subsection{Réduction des formes}

\begin{frame}
\frametitle{Traitements basiques du texte : Morphologie}
\framesubtitle{Réduction des formes : Racinisation}

\begin{itemize}
	\item suppression des affixes (anglais : \keyword{stemming})
	\item le résultat \keyword{radical} (\keyword{racine}) (anglais : \keyword{stem})
	\item Exemple, \expword{chercher \textrightarrow cherch}
	\item \optword{par base de données} : stocker tous les termes et leurs racines dans une table
	\item \optword{par statistiques} : utiliser un modèle de langue (N-Gram) pour estimer la position de troncation
	\item \optword{par règles}
	\begin{itemize}
		\item algorithme de Porter \cite{1980-porter}
		\item \url{https://github.com/assem-ch/arabicstemmer}
		\item \url{https://snowballstem.org/algorithms/}
	\end{itemize}
\end{itemize}

\end{frame}

\begin{frame}
\frametitle{Traitements basiques du texte : Morphologie}
\framesubtitle{Réduction des formes : Racinisation (exemple, algorithme de Porter)}

\begin{itemize}
	\item un ensemble de règles condition/action
	\item un framework pour créer des racinateurs : \url{https://snowballstem.org/}
	\item \optword{condition sur la racine}
	\begin{itemize}
		\item la longueur, la fin, si elle contient des voyelles, etc.
		\item Exemple, \expword{(*v*) Y \textrightarrow I : happy \textrightarrow happi, sky \textrightarrow sky}
	\end{itemize}
	\item \optword{condition sur l'affixe}
	\begin{itemize}
		\item dans le cas de porter, il n'y a que le suffixe 
		\item Exemple, \expword{SSES \textrightarrow SS, ATIONAL \textrightarrow ATE}
	\end{itemize}
	\item \optword{condition sur la règle}
	\begin{itemize}
		\item si une règle est exécutée, d'autres sont désactivées
	\end{itemize}
\end{itemize}

\end{frame}

\begin{frame}
\frametitle{Traitements basiques du texte : Morphologie}
\framesubtitle{Réduction des formes : Lemmatisation}

\begin{itemize}
	\item 
\end{itemize}

\end{frame}

\insertbibliography{TALN02}{*}

\end{document}

