% !TEX TS-program = pdflatex
% !TeX program = pdflatex
% !TEX encoding = UTF-8
% !TEX spellcheck = fr

\documentclass[xcolor=table]{beamer}


%\usepackage{fullpage}
%\usepackage[left=2.8cm,right=2.2cm,top=2 cm,bottom=2 cm]{geometry}
\setbeamersize{text margin left=10pt,text margin right=10pt}
\usepackage{amsmath,amssymb} 
\usepackage[T1]{fontenc}
\usepackage{textcomp}
\usepackage[utf8]{inputenc}
\usepackage[french]{babel}
\usepackage{arabtex}
\usepackage{txfonts}
\usepackage[]{graphicx}
\usepackage{multirow}
\usepackage{hyperref}
\usepackage{colortbl}
\usepackage{listingsutf8}
\usepackage{wrapfig}
\usepackage{multicol}
\usepackage[export]{adjustbox} %for images in table, also for frame
\usepackage[many]{tcolorbox}

\hypersetup{
	colorlinks,
	urlcolor = blue
}

%\renewcommand{\baselinestretch}{1.5}

\def\supit#1{\raisebox{0.8ex}{\small\it #1}\hspace{0.05em}}

\AtBeginSection{%
	\begin{frame}
		\sectionpage
	\end{frame}
}

\newcommand{\rottext}[2]{%
	\rotatebox{90}{%
	\begin{minipage}{#1}%
		\raggedleft#2%
	\end{minipage}%
	}%
}

\usepackage{longtable}
\usepackage{tabu}


\institute{ %
École  nationale Supérieure d'Informatique (ESI, ex. INI), Algérie
}
\author[ \textbf{\footnotesize\insertframenumber/\inserttotalframenumber} \hspace*{1.5cm} ESI - ARIES Abdelkrime (2020/2021)] %
{ARIES Abdelkrime}
%\titlegraphic{\includegraphics[height=1cm]{../img/esi-logo.png}%\hspace*{4.75cm}~


\date{Année universitaire : 2020/2021} %\today

\usetheme{Warsaw} % Antibes Boadilla Warsaw

\beamertemplatenavigationsymbolsempty

%\setbeamertemplate{headline}{}

\definecolor{lightblue}{HTML}{D0D2FF}
\definecolor{lightyellow}{HTML}{FFFFAA}
\definecolor{darkblue}{HTML}{0000BB}
\definecolor{olivegreen}{HTML}{006600}
\definecolor{violet}{HTML}{6600CC}

\newcommand{\keyword}[1]{\textcolor{red}{\bfseries\itshape #1}}
\newcommand{\expword}[1]{\textcolor{olivegreen}{#1}}
\newcommand{\optword}[1]{\textcolor{violet}{\bfseries #1}}

\makeatletter
\newcommand\mysphere{%
	\parbox[t]{10pt}{\raisebox{0.2pt}{\beamer@usesphere{item projected}{bigsphere}}}}
\makeatother

%\let\oldtabular\tabular
%\let\endoldtabular\endtabular
%\renewenvironment{tabular}{\rowcolors{2}{white}{lightblue}\oldtabular\rowcolor{blue}}{\endoldtabular}


\NoAutoSpacing %french autospacing after ":"

\def\graphpath{}

\newcommand{\changegraphpath}[1]{\def\graphpath{#1}}


\newcommand{\vgraphpage}[2][.84\textheight]{%
%	\begin{center}%
		\includegraphics[height=#1]{\graphpath #2}%
%	\end{center}%
}

\newcommand{\hgraphpage}[2][\textwidth]{%
%	\begin{center}%
		\includegraphics[width=#1]{\graphpath #2}%
%	\end{center}%
}

\newcommand{\graphpage}[2][]{%
	\includegraphics[#1]{\graphpath #2}%
}

\bibliographystyle{acm}

\newcommand{\insertbibliography}[2]{
	\appendix
	\section*{Bibliographie}
	\nocite{#2}
%	\makeatletter % to change template
%	\setbeamertemplate{headline}[default] % not mandatory, but I though it was better to set it blank
%	\def\beamer@entrycode{\vspace*{-\headheight}} % here is the part we are interested in :)
%	\makeatother
	\begin{multicols*}{2}[\frametitle{\insertsection} \usebeamertemplate*{frametitle}]%\usebeamertemplate*{frametitle}\frametitle{Références}
		\tiny
		\bibliography{#1}
	\end{multicols*}
}

\definecolor{my-grey}{RGB}{233, 233, 233}

\newcommand{\insertlicence}{
	\begin{frame}[plain]
	\frametitle{Licence : CC-BY 4.0}
%	\framesubtitle{Licence: CC-BY-NC 4.0}

	\begin{tcolorbox}[colback=cyan,
		colframe=cyan,  
		arc=0pt,outer arc=0pt,
		valign=top, 
		halign=center,
		width=\textwidth]
		
		\includegraphics[width=.5cm]{../img/licence/cc_icon_white_x2.png}
		\includegraphics[width=.5cm]{../img/licence/attribution_icon_white_x2.png}
		
		\color{white}
		\bfseries Attribution 4.0 International (CC BY 4.0) \\
		\tiny \url{https://creativecommons.org/licenses/by/4.0/deed.fr}
		
	\end{tcolorbox}\vspace{-.5cm}
	\begin{tcolorbox}[colback=my-grey,
		colframe=my-grey,  
		center, arc=0pt,outer arc=0pt,
		valign=top, 
		halign=left,
		width=\textwidth]
		
		\tiny
		
		\begin{center}
			\bfseries\Large
			Vous êtes autorisé à :
		\end{center}
		
		\begin{minipage}{0.83\textwidth}
			\begin{itemize}
				\item[] \textbf{Partager} — copier, distribuer et communiquer le matériel par tous moyens et sous tous formats
				\item[] \textbf{Adapter} — remixer, transformer et créer à partir du matériel
				pour toute utilisation, y compris commerciale.
			\end{itemize}
		\end{minipage}
		\begin{minipage}{0.15\textwidth}
			\includegraphics[width=\textwidth]{../img/licence/FreeCulturalWorks_seal_x2.jpg}
		\end{minipage}
	
		
		\begin{center}
			\bfseries\Large
			Selon les conditions suivantes :
		\end{center}
		
		\begin{itemize}
			\item[] \textbf{Attribution} — Vous devez créditer l'Œuvre, intégrer un lien vers la licence et indiquer si des modifications ont été effectuées à l'Oeuvre. Vous devez indiquer ces informations par tous les moyens raisonnables, sans toutefois suggérer que l'Offrant vous soutient ou soutient la façon dont vous avez utilisé son Oeuvre. 
			\item[] \textbf{Pas de restrictions complémentaires} — Vous n'êtes pas autorisé à appliquer des conditions légales ou des mesures techniques qui restreindraient légalement autrui à utiliser l'Oeuvre dans les conditions décrites par la licence.
		\end{itemize}
		
	\end{tcolorbox}
	
%	\begin{center}
%		\bfseries Attribution 4.0 International (CC BY 4.0)
%		\url{https://creativecommons.org/licenses/by/4.0/deed.fr}
%	\end{center}

%	\tiny
%
%	Vous êtes autorisé à : 
%	\begin{itemize}
%		\item \textbf{Partager} — copier, distribuer et communiquer le matériel par tous moyens et sous tous formats
%		\item \textbf{Adapter} — remixer, transformer et créer à partir du matériel
%	\end{itemize}
%	
%	Selon les conditions suivantes : 
%	\begin{itemize}
%		\item \textbf{Attribution} — Vous devez créditer l'Œuvre, intégrer un lien vers la licence et indiquer si des modifications ont été effectuées à l'Oeuvre. Vous devez indiquer ces informations par tous les moyens raisonnables, sans toutefois suggérer que l'Offrant vous soutient ou soutient la façon dont vous avez utilisé son Oeuvre.
%		\item \textbf{Pas d'Utilisation Commerciale} — Vous n'êtes pas autorisé à faire un usage commercial de cette Oeuvre, tout ou partie du matériel la composant. 
%		\item \textbf{Pas de restrictions complémentaires} — Vous n'êtes pas autorisé à appliquer des conditions légales ou des mesures techniques qui restreindraient légalement autrui à utiliser l'Oeuvre dans les conditions décrites par la licence.
%	\end{itemize}

	\end{frame}
}

\settowidth{\leftmargini}{\usebeamertemplate{itemize item}}
\addtolength{\leftmargini}{\labelsep}

\AtBeginDocument{
	\newcolumntype{L}[2]{>{\vbox to #2\bgroup\vfill\flushleft}p{#1}<{\egroup}} 
	
	\begin{frame}[plain]
		\maketitle
	\end{frame}

	\insertlicence
}


% needs etoolbox; to break links after -
\appto\UrlBreaks{\do\-}


\lstdefinelanguage{CSS}{
	keywords={accelerator,azimuth,background,background-attachment,
		background-color,background-image,background-position,
		background-position-x,background-position-y,background-repeat,
		behavior,border,border-bottom,border-bottom-color,
		border-bottom-style,border-bottom-width,border-collapse,
		border-color,border-left,border-left-color,border-left-style,
		border-left-width,border-right,border-right-color,
		border-right-style,border-right-width,border-spacing,
		border-style,border-top,border-top-color,border-top-style,
		border-top-width,border-width,bottom,caption-side,clear,
		clip,color,content,counter-increment,counter-reset,cue,
		cue-after,cue-before,cursor,direction,display,elevation,
		empty-cells,filter,float,font,font-family,font-size,
		font-size-adjust,font-stretch,font-style,font-variant,
		font-weight,height,ime-mode,include-source,
		layer-background-color,layer-background-image,layout-flow,
		layout-grid,layout-grid-char,layout-grid-char-spacing,
		layout-grid-line,layout-grid-mode,layout-grid-type,left,
		letter-spacing,line-break,line-height,list-style,
		list-style-image,list-style-position,list-style-type,margin,
		margin-bottom,margin-left,margin-right,margin-top,
		marker-offset,marks,max-height,max-width,min-height,
		min-width,-moz-binding,-moz-border-radius,
		-moz-border-radius-topleft,-moz-border-radius-topright,
		-moz-border-radius-bottomright,-moz-border-radius-bottomleft,
		-moz-border-top-colors,-moz-border-right-colors,
		-moz-border-bottom-colors,-moz-border-left-colors,-moz-opacity,
		-moz-outline,-moz-outline-color,-moz-outline-style,
		-moz-outline-width,-moz-user-focus,-moz-user-input,
		-moz-user-modify,-moz-user-select,orphans,outline,
		outline-color,outline-style,outline-width,overflow,
		overflow-X,overflow-Y,padding,padding-bottom,padding-left,
		padding-right,padding-top,page,page-break-after,
		page-break-before,page-break-inside,pause,pause-after,
		pause-before,pitch,pitch-range,play-during,position,quotes,
		-replace,richness,right,ruby-align,ruby-overhang,
		ruby-position,-set-link-source,size,speak,speak-header,
		speak-numeral,speak-punctuation,speech-rate,stress,
		scrollbar-arrow-color,scrollbar-base-color,
		scrollbar-dark-shadow-color,scrollbar-face-color,
		scrollbar-highlight-color,scrollbar-shadow-color,
		scrollbar-3d-light-color,scrollbar-track-color,table-layout,
		text-align,text-align-last,text-decoration,text-indent,
		text-justify,text-overflow,text-shadow,text-transform,
		text-autospace,text-kashida-space,text-underline-position,top,
		unicode-bidi,-use-link-source,vertical-align,visibility,
		voice-family,volume,white-space,widows,width,word-break,
		word-spacing,word-wrap,writing-mode,z-index,zoom},  
	sensitive=true,
	morecomment=[l]{//},
	morecomment=[s]{/*}{*/},
	morestring=[b]',
	morestring=[b]",
	alsoletter={:},
	alsodigit={-}
}
\lstdefinelanguage{HTML5}{
	language=html,
	sensitive=true, 
	alsoletter={<>=-},
	otherkeywords={
		% HTML tags
		<, </, >,
		</a, <a, </a>,
		</abbr, <abbr, </abbr>,
		</address, <address, </address>,
		</area, <area, </area>,
		</area, <area, </area>,
		</article, <article, </article>,
		</aside, <aside, </aside>,
		</audio, <audio, </audio>,
		</audio, <audio, </audio>,
		</b, <b, </b>,
		</base, <base, </base>,
		</bdi, <bdi, </bdi>,
		</bdo, <bdo, </bdo>,
		</blockquote, <blockquote, </blockquote>,
		</body, <body, </body>,
		</br, <br, </br>,
		</button, <button, </button>,
		</canvas, <canvas, </canvas>,
		</caption, <caption, </caption>,
		</cite, <cite, </cite>,
		</code, <code, </code>,
		</col, <col, </col>,
		</colgroup, <colgroup, </colgroup>,
		</data, <data, </data>,
		</datalist, <datalist, </datalist>,
		</dd, <dd, </dd>,
		</del, <del, </del>,
		</details, <details, </details>,
		</dfn, <dfn, </dfn>,
		</div, <div, </div>,
		</dl, <dl, </dl>,
		</dt, <dt, </dt>,
		</em, <em, </em>,
		</embed, <embed, </embed>,
		</fieldset, <fieldset, </fieldset>,
		</figcaption, <figcaption, </figcaption>,
		</figure, <figure, </figure>,
		</footer, <footer, </footer>,
		</form, <form, </form>,
		</h1, <h1, </h1>,
		</h2, <h2, </h2>,
		</h3, <h3, </h3>,
		</h4, <h4, </h4>,
		</h5, <h5, </h5>,
		</h6, <h6, </h6>,
		</head, <head, </head>,
		</header, <header, </header>,
		</hr, <hr, </hr>,
		</html, <html, </html>,
		</i, <i, </i>,
		</iframe, <iframe, </iframe>,
		</img, <img, </img>,
		</input, <input, </input>,
		</ins, <ins, </ins>,
		</kbd, <kbd, </kbd>,
		</keygen, <keygen, </keygen>,
		</label, <label, </label>,
		</legend, <legend, </legend>,
		</li, <li, </li>,
		</link, <link, </link>,
		</main, <main, </main>,
		</map, <map, </map>,
		</mark, <mark, </mark>,
		</math, <math, </math>,
		</menu, <menu, </menu>,
		</menuitem, <menuitem, </menuitem>,
		</meta, <meta, </meta>,
		</meter, <meter, </meter>,
		</nav, <nav, </nav>,
		</noscript, <noscript, </noscript>,
		</object, <object, </object>,
		</ol, <ol, </ol>,
		</optgroup, <optgroup, </optgroup>,
		</option, <option, </option>,
		</output, <output, </output>,
		</p, <p, </p>,
		</param, <param, </param>,
		</pre, <pre, </pre>,
		</progress, <progress, </progress>,
		</q, <q, </q>,
		</rp, <rp, </rp>,
		</rt, <rt, </rt>,
		</ruby, <ruby, </ruby>,
		</s, <s, </s>,
		</samp, <samp, </samp>,
		</script, <script, </script>,
		</section, <section, </section>,
		</select, <select, </select>,
		</small, <small, </small>,
		</source, <source, </source>,
		</span, <span, </span>,
		</strong, <strong, </strong>,
		</style, <style, </style>,
		</summary, <summary, </summary>,
		</sup, <sup, </sup>,
		</svg, <svg, </svg>,
		</table, <table, </table>,
		</tbody, <tbody, </tbody>,
		</td, <td, </td>,
		</template, <template, </template>,
		</textarea, <textarea, </textarea>,
		</tfoot, <tfoot, </tfoot>,
		</th, <th, </th>,
		</thead, <thead, </thead>,
		</time, <time, </time>,
		</title, <title, </title>,
		</tr, <tr, </tr>,
		</track, <track, </track>,
		</u, <u, </u>,
		</ul, <ul, </ul>,
		</var, <var, </var>,
		</video, <video, </video>,
		</wbr, <wbr, </wbr>,
		/>, <!
	},  
	ndkeywords={
		% General
		=,
		% HTML attributes
		accept=, accept-charset=, accesskey=, action=, align=, alt=, async=, autocomplete=, autofocus=, autoplay=, autosave=, bgcolor=, border=, buffered=, challenge=, charset=, checked=, cite=, class=, code=, codebase=, color=, cols=, colspan=, content=, contenteditable=, contextmenu=, controls=, coords=, data=, datetime=, default=, defer=, dir=, dirname=, disabled=, download=, draggable=, dropzone=, enctype=, for=, form=, formaction=, headers=, height=, hidden=, high=, href=, hreflang=, http-equiv=, icon=, id=, ismap=, itemprop=, keytype=, kind=, label=, lang=, language=, list=, loop=, low=, manifest=, max=, maxlength=, media=, method=, min=, multiple=, name=, novalidate=, open=, optimum=, pattern=, ping=, placeholder=, poster=, preload=, pubdate=, radiogroup=, readonly=, rel=, required=, reversed=, rows=, rowspan=, sandbox=, scope=, scoped=, seamless=, selected=, shape=, size=, sizes=, span=, spellcheck=, src=, srcdoc=, srclang=, start=, step=, style=, summary=, tabindex=, target=, title=, type=, usemap=, value=, width=, wrap=,
		% CSS properties
		accelerator:,azimuth:,background:,background-attachment:,
		background-color:,background-image:,background-position:,
		background-position-x:,background-position-y:,background-repeat:,
		behavior:,border:,border-bottom:,border-bottom-color:,
		border-bottom-style:,border-bottom-width:,border-collapse:,
		border-color:,border-left:,border-left-color:,border-left-style:,
		border-left-width:,border-right:,border-right-color:,
		border-right-style:,border-right-width:,border-spacing:,
		border-style:,border-top:,border-top-color:,border-top-style:,
		border-top-width:,border-width:,bottom:,caption-side:,clear:,
		clip:,color:,content:,counter-increment:,counter-reset:,cue:,
		cue-after:,cue-before:,cursor:,direction:,display:,elevation:,
		empty-cells:,filter:,float:,font:,font-family:,font-size:,
		font-size-adjust:,font-stretch:,font-style:,font-variant:,
		font-weight:,height:,ime-mode:,include-source:,
		layer-background-color:,layer-background-image:,layout-flow:,
		layout-grid:,layout-grid-char:,layout-grid-char-spacing:,
		layout-grid-line:,layout-grid-mode:,layout-grid-type:,left:,
		letter-spacing:,line-break:,line-height:,list-style:,
		list-style-image:,list-style-position:,list-style-type:,margin:,
		margin-bottom:,margin-left:,margin-right:,margin-top:,
		marker-offset:,marks:,max-height:,max-width:,min-height:,
		min-width:,transition-duration:,transition-property:,
		transition-timing-function:,transform:,
		-moz-transform:,-moz-binding:,-moz-border-radius:,
		-moz-border-radius-topleft:,-moz-border-radius-topright:,
		-moz-border-radius-bottomright:,-moz-border-radius-bottomleft:,
		-moz-border-top-colors:,-moz-border-right-colors:,
		-moz-border-bottom-colors:,-moz-border-left-colors:,-moz-opacity:,
		-moz-outline:,-moz-outline-color:,-moz-outline-style:,
		-moz-outline-width:,-moz-user-focus:,-moz-user-input:,
		-moz-user-modify:,-moz-user-select:,orphans:,outline:,
		outline-color:,outline-style:,outline-width:,overflow:,
		overflow-X:,overflow-Y:,padding:,padding-bottom:,padding-left:,
		padding-right:,padding-top:,page:,page-break-after:,
		page-break-before:,page-break-inside:,pause:,pause-after:,
		pause-before:,pitch:,pitch-range:,play-during:,position:,quotes:,
		-replace:,richness:,right:,ruby-align:,ruby-overhang:,
		ruby-position:,-set-link-source:,size:,speak:,speak-header:,
		speak-numeral:,speak-punctuation:,speech-rate:,stress:,
		scrollbar-arrow-color:,scrollbar-base-color:,
		scrollbar-dark-shadow-color:,scrollbar-face-color:,
		scrollbar-highlight-color:,scrollbar-shadow-color:,
		scrollbar-3d-light-color:,scrollbar-track-color:,table-layout:,
		text-align:,text-align-last:,text-decoration:,text-indent:,
		text-justify:,text-overflow:,text-shadow:,text-transform:,
		text-autospace:,text-kashida-space:,text-underline-position:,top:,
		unicode-bidi:,-use-link-source:,vertical-align:,visibility:,
		voice-family:,volume:,white-space:,widows:,width:,word-break:,
		word-spacing:,word-wrap:,writing-mode:,z-index:,zoom:
	},  
	morecomment=[s]{<!--}{-->},
	tag=[s]
}

%\usepackage{color}
%\definecolor{editorGray}{rgb}{0.95, 0.95, 0.95}
%\definecolor{editorOcher}{rgb}{1, 0.5, 0} % #FF7F00 -> rgb(239, 169, 0)
%\definecolor{editorGreen}{rgb}{0, 0.5, 0} % #007C00 -> rgb(0, 124, 0)
%
%\lstset{%
%	% Basic design
%	backgroundcolor=\color{editorGray},
%	basicstyle={\small\ttfamily},   
%	frame=l,
%	% Line numbers
%	xleftmargin={0.75cm},
%	numbers=left,
%	stepnumber=1,
%	firstnumber=1,
%	numberfirstline=true,
%	% Code design   
%	keywordstyle=\color{blue}\bfseries,
%	commentstyle=\color{darkgray}\ttfamily,
%	ndkeywordstyle=\color{editorGreen}\bfseries,
%	stringstyle=\color{editorOcher},
%	% Code
%	language=HTML5,
%	alsodigit={.:;},
%	tabsize=2,
%	showtabs=false,
%	showspaces=false,
%	showstringspaces=false,
%	extendedchars=true,
%	breaklines=true,        
%}

\lstset{language=CSS,
	basicstyle=\ttfamily,
	keywordstyle=\color{blue}\ttfamily,
	stringstyle=\color{red}\ttfamily,
	commentstyle=\color{green}\ttfamily,
	morecomment=[l][\color{magenta}]{\#}
}

\title[TALN : 09- Cohérence du discours]%
{Traitement automatique du langage naturel\\Chapitre 09 : Cohérence du discours} 

\changegraphpath{../img/coherence/}

\begin{document}
	
\begin{frame}
\frametitle{Traitement automatique du langage naturel}
\framesubtitle{Cohérence du discours : Introduction}

\begin{exampleblock}{Exemple d'un texte en français}
	\begin{center}
		\Large\bfseries
		L'hiver est un des quatre saisons de l'année. 
		Le rapport était bien exposé. 
		La pluie est une des caractéristiques de cette saison.
		J'ai lu un rapport sur cette saison.
	\end{center}
\end{exampleblock}

\begin{itemize}
	\item Est-ce que vous pouvez comprendre de quoi s'agit-il ?
	\item S'il y a un problème, lequel ?
	\item Est-ce qu'on peut améliorer ce texte ?
\end{itemize}

\end{frame}

\begin{frame}
\frametitle{Traitement automatique du langage naturel}
\framesubtitle{Cohérence du discours :  Un peu d'humour}

\begin{center}
	\vgraphpage{humour/humour1.jpg}
\end{center}

\end{frame}

\begin{frame}
\frametitle{Traitement automatique du langage naturel}
\framesubtitle{Cohérence du discours : Plan}

\begin{multicols}{2}
%	\small
\tableofcontents
\end{multicols}
\end{frame}

%===================================================================================
\section{Relations de cohérence}
%===================================================================================

\begin{frame}
	\frametitle{Cohérence du discours}
	\framesubtitle{Relations de cohérence}
	
	Comment détecter la cohérence d'un discours ?
	\begin{itemize}
		\item \optword{Cohérence basée sur les relations} : les phrases d'un discours sont liées entre elles par des relations. 
		\begin{itemize}
			\item Rhetorical Structure Theory (RST) 
			\item Penn Discourse TreeBank (PDTB)
		\end{itemize}
		\item \optword{Cohérence basée sur l'entité} : les phrases d'un discours forment une seule entité ; elles discutent le même sujet
		\begin{itemize}
			\item Centering Theory 
		\end{itemize}
		\item \optword{Cohésion lexicale} : les phrases proches se partagent quelques mots ou sens 
		\begin{itemize}
			\item Similarité
		\end{itemize}
	\end{itemize}
	
\end{frame}

\subsection{Rhetorical Structure Theory (RST)}

\begin{frame}
\frametitle{Cohérence du discours : Relations de cohérence}
\framesubtitle{Rhetorical Structure Theory (RST)}
	
	\begin{itemize}
		\item Structure  
		\begin{itemize}
			\item \optword{Noyau (N)} :  unité indépendante, peut être interprétée indépendamment des autres unités du texte
			\item \optword{Satellite (S)} :  unité dépendante, ne peut être interprétée qu'en utilisant le noyau
		\end{itemize}
		\item Acteurs
		\begin{itemize}
			\item \optword{Lecteur (L)} :  qui a lit le script
			\item \optword{Scripteur (Sc)} :  qui a rédigé le script
		\end{itemize}
		\item Définition d'une relation
		\begin{itemize}
			\item Des restrictions sur le Noyau 
			\item Des restrictions sur le Satellite 
			\item Des restrictions sur la combinaison du Noyau et du Satellite 
			\item L'effet produit.
		\end{itemize}
	\end{itemize}
	
\end{frame}

\begin{frame}
	\frametitle{Cohérence du discours : Relations de cohérence}
	\framesubtitle{Rhetorical Structure Theory (RST) : Relations RST (Élaboration)}
	
	\begin{itemize}
		\item \optword{Élaboration} :  S donne des informations additionnelles sur la situation présentée dans N
		\begin{itemize}
			\item \textbf{Contraintes sur N + S :} S présente des informations supplémentaires vis-à-vis de la situation ou de quelqu'aspect du thème présenté(e) dans N 
			\item \textbf{Effet :}  L reconnaît la situation présentée dans S comme fournissant des informations supplémentaires au sujet de N. 
			\item \textbf{Lieu de l'effet :} N + S
			\item Ex. \expword{[\textsubscript{N} L'examen est facile.] [\textsubscript{S} Il ne prend qu'une heur.]}
		\end{itemize}
	\end{itemize}
	
\end{frame}

\begin{frame}
	\frametitle{Cohérence du discours : Relations de cohérence}
	\framesubtitle{Rhetorical Structure Theory (RST) : Relations RST (Évidence)}
	
	\begin{itemize}
		\item \optword{Évidence} :  S donne des informations additionnelles sur la situation présentée dans N dans le but de convincre L à accepter les informations de N
		\begin{itemize}
			\item \textbf{Contraintes sur N :} L pourra ne pas croire en N à un degré suffisant pour Sc
			\item \textbf{Contraintes sur S :} L croit S ou le trouve crédible
			\item \textbf{Contraintes sur N + S :} La compréhension de S par L augmente sa croyance en N
			\item \textbf{Effet :}  La croyance de L en N est effectivement augmentée
			\item \textbf{Lieu de l'effet :} N
			\item Ex. \expword{[\textsubscript{N} Kevin doit être ici.] [\textsubscript{S} Sa voiture est garée à l'extérieur.]}
		\end{itemize}
	\end{itemize}
	
\end{frame}

\begin{frame}
	\frametitle{Cohérence du discours : Relations de cohérence}
	\framesubtitle{Rhetorical Structure Theory (RST) : Relations RST (Contraste)}
	
	\begin{itemize}
		\item \optword{Contraste} :  Une relation d'opposition entre deux noyaux
		\begin{itemize}
			\item \textbf{Contraintes sur N :} multi-noyaux
			\item \textbf{Contraintes sur N + N :} pas plus de deux noyaux ; les situations présentées dans ces deux noyaux sont (a) comprises comme les mêmes à beaucoup d'égards, (b) comprises comme différant par quelques aspects et (c) comparées par rapport à l'une ou plus d'une de ces différences
			\item \textbf{Effet :}  L reconnaît la comparabilité et les différences fournies par la comparaison effectuée
			\item \textbf{Lieu de l'effet :} noyaux multiples
			\item Ex. \expword{[\textsubscript{N} Il a voté "Non" à la nouvelle constitution.] [\textsubscript{N} Son frère a voté "Oui".]}
		\end{itemize}
	\end{itemize}
	
\end{frame}

\begin{frame}
	\frametitle{Cohérence du discours : Relations de cohérence}
	\framesubtitle{Rhetorical Structure Theory (RST) : Relations RST (Antithèse)}
	
	\begin{itemize}
		\item \optword{Antithèse} :  Une relation d'opposition entre N et S dans le but que L ait une attitude positive envers N
		\begin{itemize}
			\item \textbf{Contraintes sur N :} Sc a une attitude positive par rapport à la situation présentée dans N
			\item \textbf{Contraintes sur N + N :} es situations présentées dans N et S sont en opposition (cf. CONTRASTE)
			\item \textbf{Effet :}  L'attitude positive de L vis-à-vis de N est augmentée
			\item \textbf{Lieu de l'effet :} N
			\item Ex. \expword{[\textsubscript{N} J'ai défendu la République jeune;] [\textsubscript{S} Je ne l'abandonnerai pas maintenant que je suis vieux.]}
		\end{itemize}
	\end{itemize}
	
\end{frame}

\begin{frame}
	\frametitle{Cohérence du discours : Relations de cohérence}
	\framesubtitle{Rhetorical Structure Theory (RST) : Relations thématiques vs. présentationnelles}
	
	\begin{center}
		\footnotesize
	\begin{tblr}{
			colspec = {p{.4\textwidth}lp{.4\textwidth}},
			row{odd} = {lightblue},
			row{even} = {lightyellow},
			row{1} = {darkblue},
			rowsep = 0pt,
		}

		\bfseries\textcolor{white}{thématique} && \bfseries\textcolor{white}{présentationnelle}\\
		
		Élaboration
		
		Circonstance
		
		Problème-Solution
		
		Cause intentionnelle
		
		Résultat intentionnel
		
		Cause non-intentionnelle
		
		Résultat non-intentionnel
		
		But
		
		Condition
		
		Interprétation
		
		Évaluation
		
		Reformulation
		
		Résumé
		
		Séquence
		
		Contraste
		
		&&
		Motivation
		
		Antithèse
		
		Arrière-plan
		
		Facilitation
		
		Évidence (« indice »)
		
		Justification
		
		Concession\\
	\end{tblr}
	\end{center}
	
\end{frame}

\begin{frame}
	\frametitle{Cohérence du discours : Relations de cohérence}
	\framesubtitle{Rhetorical Structure Theory (RST) : Relations sémantiques vs. pragmatiques vs. textuelles}
	
	\begin{center}
		\footnotesize
	\begin{tblr}{
			colspec = {p{.3\textwidth}lp{.25\textwidth}lp{.2\textwidth}},
			row{odd} = {lightblue},
			row{even} = {lightyellow},
			row{1} = {darkblue},
			rowsep = 0pt,
		}

		\bfseries\textcolor{white}{Sémantique} && \bfseries\textcolor{white}{Pragmatique} && \bfseries\textcolor{white}{Textuelle} \\
		
		Élaboration
		
		Circonstance
		
		Problème-Solution
		
		Cause intentionnelle
		
		Résultat intentionnel
		
		Cause non-intentionnelle
		
		Résultat non-intentionnel
		
		But
		
		Condition
		
		Interprétation
		
		Évaluation
		
		Séquence
		
		Contraste
		
		&&
		
		Motivation
		
		Antithèse
		
		Facilitation
		
		Évidence (« indice »)
		
		Justification
		
		Concession
		
		&&
		
		Arrière-plan
		
		Reformulation
		
		Résumé\\
	\end{tblr}
	\end{center}
	
\end{frame}

\subsection{Penn Discourse TreeBank (PDTB)}

\begin{frame}
	\frametitle{Cohérence du discours : Relations de cohérence}
	\framesubtitle{Penn Discourse TreeBank (PDTB)}
	
	\begin{itemize}
		\item Annotation des connectives de discours
		\begin{itemize}
			\item \optword{Conjonctions de subordination} :  
			temporelle (Ex., \expword{'when', 'as soon as'}), 
			causale (Ex., \expword{'because'}), 
			concessive (Ex., \expword{'although', 'even though'}), 
			objectif (Ex.,\expword{'so that', 'in order that'}) et 
			conditionnelle (Ex., \expword{'if', 'unless'}).
			
			\item \optword{Conjonctions de coordination} : \expword{'and', 'but', 'or'}
			
			\item \optword{Conjonctives adverbiales} : des adverbes qui expriment une relation de discours entre des évenements ou des états. Ex., \expword{'however', 'therefore', 'then', etc.}
			Des syntagmes prépositionnels sont inclus aussi dans cette classe. Ex. \expword{'as a result',
			'in addition', 'in fact', etc. }
			
			\item \optword{Connectives implicites} :  identifiées entre deux phrases adjacentes qui ne sont pas reliées par des connectives explicites.
		\end{itemize}
		\item Annotation des arguments 
		\begin{itemize}
			\item ARG2 : c'est la clause où la connective est liée (cas explicite)
			\item ARG1 : l'autre clause
		\end{itemize}
	\end{itemize}
	
\end{frame}

\begin{frame}
	\frametitle{Cohérence du discours : Relations de cohérence}
	\framesubtitle{Penn Discourse TreeBank (PDTB) : Sens des connectives}
	
	\begin{table}
	\tiny\bfseries
	\begin{tblr}{
			colspec = {p{.1\textwidth}lp{.8\textwidth}},
			row{odd} = {lightblue},
			row{even} = {lightyellow},
			row{1} = {darkblue},
			rowsep = 1pt,
		}

		\bfseries\textcolor{white}{Connective} && \bfseries\textcolor{white}{Sens}\\
		
		after && succession (523), succession-reason (50), other (4) \\
		since && reason (94), succession (78), succession-reason (10), other (2) \\
		when && Synchrony (477), succession (157), general (100), succession-reason (65), Synchrony-general (50),
		Synchrony-reason (39), hypothetical (11), implicit assertion (11), Synchrony-hypothetical (10), other
		(69) \\
		while && juxtaposition (182), Synchrony (154), Contrast (120), expectation (79), opposition (78), Conjunction
		(39), Synchrony-juxtaposition (26), Synchrony-Conjunction (21), Synchrony-Contrast(22), COMPARISON (18), Synchrony-opposition (11), other (31) \\
		meanwhile && Synchrony-Conjunction (92), Synchrony (26), Conjunction (25), Synchrony-juxtaposition (15),
		other(35)\\
		but && Contrast (1609), juxtaposition (636), contra-expectation (494), COMPARISTON (260), opposition
		(174), Conjunction (63), Conjunction-Pragmatic contrast (14), Pragmatic-contrast (14), other (32)
		however Contrast (254), juxtaposition (89), contra-expectation (70), COMPARISON (49), opposition (31),
		other (12)\\
		although && expectation (132), Contrast (114) juxtaposition (34), contra-expectation (21), COMPARISON (16),
		opposition (9), other (2)\\
		and && Conjunction (2543), List (210), result-Conjunction (138), result (38), precedence-Conjunction (30),
		juxtaposition (11), other(30)\\
		if && hypothetical (682), general (175), unreal present (122), factual present (73), unreal past (53), expectation (34), implicit assertion (29), relevance (20), other (31)\\
	\end{tblr}
	\caption{Quelques connectives et des statistiques sur leurs sens \cite{2008-prasad-al}}
	\end{table}
	
\end{frame}

\begin{frame}
	\frametitle{Cohérence du discours : Relations de cohérence}
	\framesubtitle{Penn Discourse TreeBank (PDTB) : Sens des connectives}
	
	\begin{figure}
		\centering\vspace{-6pt}
		\begin{tcolorbox}[colback=white, colframe=blue, boxrule=1pt, text width=.9\textwidth]
		\begin{minipage}{.3\textwidth}
			\tiny\bfseries\vspace{-6pt}
			\begin{itemize}
				\item CONTINGENCY
				\begin{itemize}\tiny\bfseries
					\item Cause
					\begin{itemize}\tiny\bfseries
						\item Reason
						\item Result
					\end{itemize}
					\item Pragmatic Cause
					\begin{itemize}\tiny\bfseries
						\item Justification
					\end{itemize}
					\item Condition
					\begin{itemize}\tiny\bfseries
						\item Hypothetical
						\item General
						\item Unreal Present
						\item Unreal Past
						\item Factual Present
						\item Factual Past
					\end{itemize}
					\item Pragmatic Condition
					\begin{itemize}\tiny\bfseries
						\item Relevance
						\item Implicit Assertion
					\end{itemize}
				\end{itemize}
			\end{itemize}
		\end{minipage}
		\begin{minipage}{.3\textwidth}
			\tiny\bfseries
			\begin{itemize}\tiny\bfseries
				\item TEMPORAL
				\begin{itemize}\tiny\bfseries
					\item Asynchronous
					\item Synchronous
					\begin{itemize}\tiny\bfseries
						\item Precedence
						\item Succession
					\end{itemize}
				\end{itemize}
				\item COMPARISON
				\begin{itemize}\tiny\bfseries
					\item Contrast
					\begin{itemize}\tiny\bfseries
						\item Juxtaposition 
						\item Opposition
					\end{itemize}
					\item Pragmatic Contrast
					\item Concession
					\begin{itemize}\tiny\bfseries
						\item Expectation
						\item Contra-expectation
					\end{itemize}
					\item Pragmatic Concession
				\end{itemize}
			\end{itemize}
		\end{minipage}
		\begin{minipage}{.3\textwidth}
			\tiny\bfseries
			\begin{itemize}\tiny\bfseries
				\item EXPANSION
				\begin{itemize}\tiny\bfseries
					\item Conjunction
					\item Instantiation
					\item Restatement
					\begin{itemize}\tiny\bfseries
						\item Specification
						\item Equivalence
						\item Generalization
					\end{itemize}
					\item Alternative
					\begin{itemize}\tiny\bfseries
						\item Conjunction
						\item Disjunction
						\item Chosen Alternative
					\end{itemize}
					\item Exception
					\item List
				\end{itemize}
			\end{itemize}
		\end{minipage}\vspace{-6pt}
		\end{tcolorbox}
		\caption{Hierarchie des sens dans PDTB \cite{2008-prasad-al}}
	\end{figure}


%	\begin{figure}
%		\vgraphpage[.7\textheight]{pdtb-sens_.pdf}
%		\caption{Hierarchie des sens dans PDTB \cite{2008-prasad-al}}
%	\end{figure}
	
\end{frame}

%===================================================================================
\section{Analyse basée structure de discours}
%===================================================================================

\begin{frame}
	\frametitle{Cohérence du discours}
	\framesubtitle{Analyse basée structure de discours}
	
	\begin{itemize}
		\item Un discours est cohérent s'il existe des relations entre ses parties
		\item La structure
		\begin{itemize}
			\item Un arbre de relations : RST
			\item Une relation binaire entre chaque deux parties : PDTB
		\end{itemize}
	\end{itemize}
	
\end{frame}

\subsection{Analyse RST}

\begin{frame}
	\frametitle{Cohérence du discours : Analyse basée structure}
	\framesubtitle{Analyse RST}
	
	\begin{itemize}
		\item Construire un arbre de relations de discours
		\item Deux étapes : 
		\begin{enumerate}
			\item Détection des unités élémentaires de discours
			\item Classification des relations
		\end{enumerate}
	\end{itemize}

	\begin{center}
		\hgraphpage[0.5\textwidth]{RST-arbre.pdf}
	\end{center}
	
\end{frame}

\begin{frame}
	\frametitle{Cohérence du discours : Analyse basée structure}
	\framesubtitle{Analyse RST : Détection des unités élémentaires de discours (UED)}
	
	\begin{itemize}
		\item Elementary discourse units (EDUs)
		\item Une phrase peut avoir plusieurs UEDs 
		\item Ex. \expword{[Mr. Rambo says]\textsubscript{e1} [that a 3.2-acre property]\textsubscript{e2} [overlooking the San Fernando Valley]\textsubscript{e3} [is priced at \$4 million]\textsubscript{e4} [because the late actor Erroll Flynn once lived there.]\textsubscript{e5}}
		\item Méthodes 
		\begin{itemize}
			\item \optword{Par règles}
			\begin{itemize}
				\item arbre syntaxique
				\item indices de surface : ponctuation et des marqueurs lexicaux comme les connectives
			\end{itemize}
			\item Statistiques 
			\begin{itemize}
				\item \optword{Par caractéristiques} : utilisation des informations syntaxiques et des indices de surface pour apprendre les limites des UEDs
				\item \optword{Par embeddings} : texte comme séquence. Ex., \url{https://github.com/PKU-TANGENT/NeuralEDUSeg}
			\end{itemize}
		\end{itemize}
	\end{itemize}
	
\end{frame}

\begin{frame}
	\frametitle{Cohérence du discours : Analyse basée structure}
	\framesubtitle{Analyse RST : Détection des UEDs par embeddings}
	
	\begin{figure}
		\centering
		\hgraphpage[.45\textwidth]{EDU_seg.pdf}
		\caption{Segmentation en UEDs proposée par \cite{2018-wang-al} ; figure inspirée de \cite{2019-jurafsky-martin}}
	\end{figure}
	
\end{frame}

\begin{frame}
	\frametitle{Cohérence du discours : Analyse basée structure}
	\framesubtitle{Analyse RST : Classification des relations}
	
	\begin{minipage}{.6\textwidth}
	\begin{itemize}
		\item Méthode plus utilisée : SHIFT-REDUCE
		\item Configuration : $C = (\sigma, \beta, A)$
		\begin{itemize}
			\item $\sigma$ est une pile
			\item $\beta$ est le tampon (buffer) d'entrée
			\item $A$ est la liste des arcs créés (relations)
			\item $C_{initiale} = (\varnothing, w, \emptyset)$
			\item $C_{finale} = (\varnothing, \varnothing, A)$
		\end{itemize}
	\end{itemize}
	\end{minipage}
	\begin{minipage}{.38\textwidth}
		\hgraphpage{RST-transitions.pdf}
	\end{minipage}
	\begin{itemize}
		\item Opérations 
		\begin{itemize}
			\item \optword{Shift} : mettre le premier élément du buffer dans la pile
			\item \optword{Reduce}\textbf{(l, d)} : fusionne les deux sous-arbres supérieurs de la pile, où \textbf{l} est l'étiquette de relation de cohérence, et \textbf{d} est la direction de nucléarité : \textbf{d $ \in $ \{NN, NS, SN\}}.
			\item \optword{Pop Root} : enlever l'arbre final de la pile.
		\end{itemize}
		\item Apprentissage 
		\begin{itemize}
			\item Par caractéristiques
			\item Par embeddings
		\end{itemize}
	\end{itemize}
	
\end{frame}

\begin{frame}
	\frametitle{Cohérence du discours : Analyse basée structure}
	\framesubtitle{Analyse RST : Classification des relations (exemple)}
	
	\begin{figure}
		\begin{minipage}{0.25\textwidth}
			\hgraphpage{RST_exp.pdf}
		\end{minipage}
		\begin{minipage}{0.70\textwidth}
			\scriptsize
			\begin{itemize}
				\item e\textsubscript{1} : American Telephone \& Telegraph Co. said it
				\item e\textsubscript{2} : will lay off 75 to 85 technicians here , effective Nov. 1.
				\item e\textsubscript{3} : The workers install , maintain and repair its private branch exchanges,
				\item e\textsubscript{4} : which are large intracompany telephone networks.
			\end{itemize}
		\end{minipage}
		
		\small
		\begin{tabular}{lllll}
			\hline\hline 
			étape & Pile & Tampon & Action & Relation ajoutée \\
			\hline
			01 & $\emptyset$ & $e_1, e_2, e_3, e_4$ & SH & / \\
			02 & $e_1$ & $e_2, e_3, e_4$ & SH & / \\
			03 & $e_1, e_2$ & $e_3, e_4$ & RD(attr, SN) & $\widehat{e_1 \mathbf{e_2}}$ \\
			04 & $e_{1:2}$ & $e_4$ & SH & / \\
			05 & $e_{1:2}, e_3$ & $e_4$ & SH & / \\
			06 & $e_{1:2}, e_3, e_4$ & $\emptyset$ & RD(elab, NS) & $\widehat{\mathbf{e_3} e_4}$ \\
			07 & $e_{1:2}, e_{3:4}$ & $\emptyset$ & RD(elab, SN) & $\widehat{e_{1:2}\mathbf{e_{3:4}}}$ \\
			08 & $e_{1:4}$ & $\emptyset$ & PR & / \\
			\hline\hline
		\end{tabular}
		\caption{Exemple d'analyse RST en utilisant Shif-Reduce \cite{2018-yu-al}}
	\end{figure}
	
\end{frame}

\begin{frame}
	\frametitle{Cohérence du discours : Analyse basée structure}
	\framesubtitle{Analyse RST : Classification des relations avec embeddings (1)}
	
	\begin{itemize}
		\item Cas d'étude : \cite{2018-yu-al} (Encodeur-décodeur)
		\item Code : \url{https://github.com/yunan4nlp/NNDisParser}
		\item Encodeur 
		\begin{itemize}
			\item Encoder les mots : $ w_i $ un mot avec une catégorie grammaticale $t_i$
			\[x_i^w = embedding(w_i) \oplus embedding(t_i)\]
			\item Représenter les mots séquentiels
			\[ \{h_1^w, h_2^w, \ldots, h_m^w \} = biLSTM(\{x_1^w, x_2^w, \ldots, x_m^w \})\]
			\item Représenter un UED $\{w_s, w_{s+1}, \ldots, w_t \}$
			\[ x^e = \frac{1}{t-s+1} \sum_{k=s}^{t} h_k^w\]
			\item Représenter les UEDs séquentiels
			\[ \{h_1^e, h_2^e, \ldots, h_n^e \} = biLSTM(\{x_1^e, x_2^e, \ldots, x_n^e \})\]
		\end{itemize}
	\end{itemize}
	
\end{frame}

\begin{frame}
	\frametitle{Cohérence du discours : Analyse basée structure}
	\framesubtitle{Analyse RST : Classification des relations avec embeddings (2)}
	
	\begin{itemize}
		\item Décodeur 
		\begin{itemize}
			\item Une couche neuronale feed-forward pour inférer l'action $o$
			\[o = W(h_{s0}^{sbt} \oplus h_{s1}^{sbt} \oplus h_{s2}^{sbt} \oplus h_{q0}^{e})\]
			\item $ h_{e}^{q0} $ : la représentation séquentielle du premier UED dans le buffer
			\item $h_{si}^{sbt}$ : la représentation séquentielle du sous-arbre $i$ dans la pile
			\item la représentation d'un sous arbre entre deux UEDs $ s= \{e_i, \ldots, e_j\}$ est la moyenne des représentations des UEDs couverts
			\[ h_{s}^{sbt} = \frac{1}{j-i+1} \sum_{k=i}^{j} h_k^e\]
		\end{itemize}
	\end{itemize}
	
\end{frame}

\subsection{Analyse PDTB}

\begin{frame}
	\frametitle{Cohérence du discours : Analyse basée structure}
	\framesubtitle{Analyse PDTB}
	
	\begin{itemize}
		\item Trouver la relation entre deux parties du texte
		\item Quatre étapes : 
		\begin{enumerate}
			\item Trouvez les connecteurs du discours
			\begin{itemize}
				\item En utilisant la désambigüisation du sens (délimiteur ou non)
			\end{itemize}
			\item Trouvez les deux parties pour chaque connective
			\begin{itemize}
				\item Pour la partie avec le connecteur, on peut classer les parties adjacente comme ayant une relation ou non
			\end{itemize}
			\item Étiquetez la relation entre ces parties (connectives explicites) : 
			\begin{itemize}
				\item Les étiquettes sont des connectives
			\end{itemize}
			\item Attribuer une relation entre chaque paire de phrases adjacentes (connectives implicites)
			\begin{itemize}
				\item Par exemple, en utilisant BERT avec les deux parties en entrée et en ajoutant une couche neuronale à sa sortie $ <CLS> $ pour inférer la relation
			\end{itemize}
		\end{enumerate}
	\end{itemize}
	
\end{frame}

\begin{frame}
	\frametitle{Traitement automatique du langage naturel}
	\framesubtitle{Analyse basée structure :  Un peu d'humour}
	
	\begin{center}
		\vgraphpage{humour/humour2.jpg}
	\end{center}
	
\end{frame}

%===================================================================================
\section{Analyse basée sur l'entité de discours}
%===================================================================================

\begin{frame}
	\frametitle{Cohérence du discours}
	\framesubtitle{Analyse basée sur l'entité de discours}
	
	\begin{itemize}
		\item Un discours est cohérent s'il discute une certaine entité
		\item Quelque soit la position dans le discours, cette entité doit rester la plus importante
	\end{itemize}

	\begin{exampleblock}{Exemple de cohérence \cite{2019-jurafsky-martin}}
		\small
		\begin{minipage}{.48\textwidth}
			\begin{enumerate}
				\item John went to his favorite music store to buy a piano. [John]
				\item He had frequented the store for many years. [John]
				\item He was excited that he could finally buy a piano. [John]
				\item He arrived just as the store was closing for the day. [John]
			\end{enumerate}
		\end{minipage}
		\begin{minipage}{.48\textwidth}
			\begin{enumerate}
				\item John went to his favorite music store to buy a piano. [John]
				\item It was a store John had frequented for many years. [The store]
				\item He was excited that he could finally buy a piano. [John]
				\item It was closing just as John arrived. [The store]
			\end{enumerate}
		\end{minipage}
	\end{exampleblock}
	
\end{frame}

\subsection{Centering theory}

\begin{frame}
	\frametitle{Cohérence du discours : Analyse basée entité}
	\framesubtitle{Centering theory : Centres de discours}
	
	\begin{itemize}
		\item $\mathbf{U_n}$ : une énonciation
		\item $\mathbf{C_b(U_n)}$ : \optword{Backward-looking center} (centre rétrospectif)
		\begin{itemize}
			\item l'entité saillante actuelle
			\item celle sur laquelle se concentre le discours dans l'énonciation $ U_{n-1} $.
		\end{itemize}
		\item $\mathbf{C_f(U_n)}$ : \optword{Forward-looking center} (centres prospectifs)
		\begin{itemize}
			\item les entités potentiellement saillantes dans le futur
			\item celles candidates pour être $C_b(U_{n+1})$.
		\end{itemize}
		\item $\mathbf{C_p(U_n) \in C_f(U_n)}$ : \optword{Prefered center} (centre préféré)
		\begin{itemize}
			\item centre candidat pour être $C_b(U_{n+1})$
			\item score sur le rôle grammatical (sujet plus important que l'objet qui est plus important que le reste), l'ordre (Ex. \expword{En Arabe, ce qui est en premier est plus important}), etc.
		\end{itemize}
		\item $C_b(U_n)$ doit être représenté par un pronom dans $U_{n+1}$
	\end{itemize}
	
\end{frame}

\begin{frame}
	\frametitle{Cohérence du discours : Analyse basée entité}
	\framesubtitle{Centering theory : Transitions}
	
	\begin{itemize}
		\item Ordre des transitions selon la plus cohérente
		\item \optword{CONTINUE} : le locuteur parle d'une entité et a l'intention d'en parler en futur
		\item \optword{RETAIN} : le locuteur parle d'une entité et a l'intention d'en changer en futur
		\item \optword{SHIFT} : le locuteur a changé l'entité centrale
		\begin{itemize}
			\item \textbf{Smooth-SHIFT} : après le changement, il a l'intention d'en parler en futur
			\item \textbf{Rough-SHIFT} : après le changement, il a l'intention d'en changer en futur
		\end{itemize}
	\end{itemize}

	\begin{center}
		\tiny\bfseries
		\begin{tblr}{
				colspec = {p{.2\textwidth}p{.2\textwidth}p{.2\textwidth}},
				row{odd} = {lightblue},
				row{even} = {lightyellow},
				row{1} = {darkblue},
				rowsep = 2pt,
			}
			
			& \bfseries\color{white}$\mathbf{C_b(U_n) = C_b(U_{n-1})}$
			
			OU $\mathbf{C_b(U_n) = NULL}$
			& \bfseries\color{white}$\mathbf{C_b(U_n) \ne C_b(U_{n-1})}$\\
			
			$\mathbf{C_b(U_n) = C_p(U_n)}$ &
			CONTINUE & Smooth-SHIFT\\
			
			$\mathbf{C_b(U_n) \ne C_p(U_n)}$ &
			RETAIN & Rough-SHIFT\\
		\end{tblr}
	\end{center}
	
\end{frame}

\subsection{Entity Grid model}

\begin{frame}
	\frametitle{Cohérence du discours : Analyse basée entité}
	\framesubtitle{Entity Grid model}
	
	\begin{itemize}
		\item Un document est représenté par une matrice
		\begin{itemize}
			\item Lignes : les phrases 
			\item Colonnes : les entités
			\item Valeurs : fonction grammaticale de l'entité dans la phrase ; sujet (S), Objet (O), Autre (X) ou l'entité n'existe pas (\_).
		\end{itemize}
		\item Un document est cohérent, s'il suit un pattern
		\begin{itemize}
			\item Ce pattern peut-être représenté par les fréquences de transitions grammaticales. Ex. \expword{SS, SO, SX, S\_, OS, OO, OX, O\_, ...}
			\item On utilise l'apprentissage automatique pour juger la cohérence d'un document
		\end{itemize}
		
	\end{itemize}
	
\end{frame}

\begin{frame}
	\frametitle{Cohérence du discours : Analyse basée entité}
	\framesubtitle{Entity Grid model : Exemple d'une représentation de document}
	
	\begin{figure}
		\centering\tiny\bfseries
		\begin{minipage}{.8\textwidth}
			\begin{enumerate}
				\item\ [The Justice Department]\textsubscript{s} is conducting an [anti-trust trial]\textsubscript{o} against [Microsoft Corp.]\textsubscript{x} with [evidence]\textsubscript{x} that [the company]\textsubscript{s} is increasingly attempting to crush [competitors]\textsubscript{o}.
				\item\ [Microsoft]\textsubscript{o} is accused of trying to forcefully buy into [markets]\textsubscript{x} where [its own products]\textsubscript{s} are not competitive enough to unseat [established brands]\textsubscript{o}.
				\item\ [The case]\textsubscript{s} revolves around [evidence]\textsubscript{o} of [Microsoft]\textsubscript{s} aggressively pressuring [Netscape]\textsubscript{o} into merging [browser software]\textsubscript{o}.
				\item\ [Microsoft]\textsubscript{s} claims [its tactics]\textsubscript{s} are commonplace and good economically.
				\item\ [The government]\textsubscript{s} may file [a civil suit]\textsubscript{o} ruling that [conspiracy]\textsubscript{s} to curb [competition]\textsubscript{o} through [collision]\textsubscript{x} is [a violation of the Sherman Act]\textsubscript{o}.
				\item\ [Microsoft]\textsubscript{s} continues to show [inscreased earnings]\textsubscript{o} despite [the trial]\textsubscript{x}.
			\end{enumerate}
		\end{minipage}
		
		\color{blue}
		\begin{tabular}{ccccccccccccccccc}
			& \rotatebox[origin=c]{90}{Department} & \rotatebox[origin=c]{90}{Trial} & \rotatebox[origin=c]{90}{Microsoft} &
			\rotatebox[origin=c]{90}{Evidence} & \rotatebox[origin=c]{90}{Competitors} & \rotatebox[origin=c]{90}{Markets} &
			\rotatebox[origin=c]{90}{Products} & \rotatebox[origin=c]{90}{Brands} & \rotatebox[origin=c]{90}{Case} &
			\rotatebox[origin=c]{90}{Netscape} & \rotatebox[origin=c]{90}{Software} & \rotatebox[origin=c]{90}{Tactics} &
			\rotatebox[origin=c]{90}{Government} & \rotatebox[origin=c]{90}{Suit} & \rotatebox[origin=c]{90}{Earnings} & \\
			
			1. & s & o & s & x & o & - & - & - & - & - & - & - & - & - & - & 1. \\
			2. & - & - & o & - & - & x & s & o & - & - & - & - & - & - & - & 2. \\
			3. & - & - & s & o & - & - & - & - & s & o & o & - & - & - & - & 3. \\
			4. & - & - & s & - & - & - & - & - & - & - & - & s & - & - & - & 4. \\
			5. & - & - & - & - & - & - & - & - & - & - & - & - & s & o & - & 5. \\
			6. & - & x & s & - & - & - & - & - & - & - & - & - & - & - & o & 6. \\
			
		\end{tabular}
		\caption{Exemple de la représentation d'un document par entités \cite{2008-barzilay-lapata}}
	\end{figure}


%	\begin{figure}
%		\hgraphpage[.7\textwidth]{EGM_doc_exp_.pdf}
%		
%		\hgraphpage[.4\textwidth]{EGM_doc_rep_exp_.pdf}
%		\caption{Un exemple de la représentation d'un document par entités \cite{2008-barzilay-lapata}}
%	\end{figure}
	
\end{frame}

\begin{frame}
	\frametitle{Cohérence du discours : Analyse basée entité}
	\framesubtitle{Entity Grid model : Représentation par transitions}
	
	\begin{itemize}
		\item Un document est représenté par les probabilités des transitions grammaticales
		\item Une probabilité est le ration entre le nombre d'une transition et le nombre de toutes les transitions
		\item Ex. Dans l'exemple précédent : \expword{P(S\_) = 2/75}
	\end{itemize}
	\begin{figure}
		\tiny\bfseries
		\begin{tabular}{lllllllllllllllll}
			\hline
			& s s & s o & s x & s - & o s & o o & o x & o - & x s & x o & x x & x - & - s & - o & - x & - - \\
			\hline
			d1 & .01 & .01 & .00 & .08 & .01 & .00 & .00 & .09 & .00 & .00 & .00 & .03 & .05 & .07 & .03 & .59 \\
			d2 & .02 & .01 & .01 & .02 & .00 & .07 & .00 & .02 & .14 & .14 & .06 & .04 & .03 & .07 & .01 & .36 \\
			d3 & .02 & .00 & .00 & .03 & .09 & .00 & .09 & .06 & .00 & .00 & .00 & .05 & .03 & .07 & .17 & .39 \\
			\hline
		\end{tabular}
		\caption{Un exemple de la représentation des documents par transitions grammaticales \cite{2008-barzilay-lapata}}
	\end{figure}
	
\end{frame}

\begin{frame}
	\frametitle{Cohérence du discours : Analyse basée entité}
	\framesubtitle{Entity Grid model : Apprentissage}
	
	\begin{itemize}
		\item L'algorithme apprend à noter un texte selon sa cohérence
		\item Un texte non cohérent doit avoir un score inférieur à un autre cohérent
		\item Utilisation des textes annotés selon leurs cohérences par des experts
		\item Utilisation d'une méthode auto-supervisée : créer un texte non cohérent à partir d'un texte cohérent
		\begin{itemize}
			\item Discrimination de l'ordre des phrases : ordre aléatoire des phrases et comparaison du score de cohérence avec celui de l'ordre original
			\item Insertion de la phrase : modifier l'ordre d'une seule phrase et comparaison du score de cohérence avec celui de l'ordre original
			\item Reconstruction de l'ordre des phrases : apprendre à ordonner les phrases
		\end{itemize}
	\end{itemize}
	
\end{frame}

\begin{frame}
	\frametitle{Traitement automatique du langage naturel}
	\framesubtitle{Analyse basée entité :  Un peu d'humour}
	
	\begin{center}
		\vgraphpage{humour/humour3.png}
	\end{center}
	
\end{frame}


\insertbibliography{TALN09}{*}

\end{document}

