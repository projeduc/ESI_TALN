% !TEX TS-program = pdflatex
% !TeX program = pdflatex
% !TEX encoding = UTF-8
% !TEX spellcheck = fr

\documentclass{beamer}


%\usepackage{fullpage}
%\usepackage[left=2.8cm,right=2.2cm,top=2 cm,bottom=2 cm]{geometry}
\setbeamersize{text margin left=10pt,text margin right=10pt}
\usepackage{amsmath,amssymb} 
\usepackage[T1]{fontenc}
\usepackage{textcomp}
\usepackage[utf8]{inputenc}
\usepackage[french]{babel}
\usepackage{arabtex}
\usepackage{txfonts}
\usepackage[]{graphicx}
\usepackage{multirow}
\usepackage{hyperref}
\usepackage{colortbl}
\usepackage{listingsutf8}
\usepackage{wrapfig}
\usepackage{multicol}
\usepackage[export]{adjustbox} %for images in table, also for frame
\usepackage[many]{tcolorbox}

\hypersetup{
	colorlinks,
	urlcolor = blue
}

%\renewcommand{\baselinestretch}{1.5}

\def\supit#1{\raisebox{0.8ex}{\small\it #1}\hspace{0.05em}}

\AtBeginSection{%
	\begin{frame}
		\sectionpage
	\end{frame}
}

\newcommand{\rottext}[2]{%
	\rotatebox{90}{%
	\begin{minipage}{#1}%
		\raggedleft#2%
	\end{minipage}%
	}%
}

\usepackage{longtable}
\usepackage{tabu}


\institute{ %
École  nationale Supérieure d'Informatique (ESI, ex. INI), Algérie
}
\author[ \textbf{\footnotesize\insertframenumber/\inserttotalframenumber} \hspace*{1.5cm} ESI - ARIES Abdelkrime (2020/2021)] %
{ARIES Abdelkrime}
%\titlegraphic{\includegraphics[height=1cm]{../img/esi-logo.png}%\hspace*{4.75cm}~


\date{Année universitaire : 2020/2021} %\today

\usetheme{Warsaw} % Antibes Boadilla Warsaw

\beamertemplatenavigationsymbolsempty

%\setbeamertemplate{headline}{}

\definecolor{lightblue}{HTML}{D0D2FF}
\definecolor{lightyellow}{HTML}{FFFFAA}
\definecolor{darkblue}{HTML}{0000BB}
\definecolor{olivegreen}{HTML}{006600}
\definecolor{violet}{HTML}{6600CC}

\newcommand{\keyword}[1]{\textcolor{red}{\bfseries\itshape #1}}
\newcommand{\expword}[1]{\textcolor{olivegreen}{#1}}
\newcommand{\optword}[1]{\textcolor{violet}{\bfseries #1}}

\makeatletter
\newcommand\mysphere{%
	\parbox[t]{10pt}{\raisebox{0.2pt}{\beamer@usesphere{item projected}{bigsphere}}}}
\makeatother

%\let\oldtabular\tabular
%\let\endoldtabular\endtabular
%\renewenvironment{tabular}{\rowcolors{2}{white}{lightblue}\oldtabular\rowcolor{blue}}{\endoldtabular}


\NoAutoSpacing %french autospacing after ":"

\def\graphpath{}

\newcommand{\changegraphpath}[1]{\def\graphpath{#1}}


\newcommand{\vgraphpage}[2][.84\textheight]{%
%	\begin{center}%
		\includegraphics[height=#1]{\graphpath #2}%
%	\end{center}%
}

\newcommand{\hgraphpage}[2][\textwidth]{%
%	\begin{center}%
		\includegraphics[width=#1]{\graphpath #2}%
%	\end{center}%
}

\newcommand{\graphpage}[2][]{%
	\includegraphics[#1]{\graphpath #2}%
}

\bibliographystyle{acm}

\newcommand{\insertbibliography}[2]{
	\appendix
	\section*{Bibliographie}
	\nocite{#2}
%	\makeatletter % to change template
%	\setbeamertemplate{headline}[default] % not mandatory, but I though it was better to set it blank
%	\def\beamer@entrycode{\vspace*{-\headheight}} % here is the part we are interested in :)
%	\makeatother
	\begin{multicols*}{2}[\frametitle{\insertsection} \usebeamertemplate*{frametitle}]%\usebeamertemplate*{frametitle}\frametitle{Références}
		\tiny
		\bibliography{#1}
	\end{multicols*}
}

\definecolor{my-grey}{RGB}{233, 233, 233}

\newcommand{\insertlicence}{
	\begin{frame}[plain]
	\frametitle{Licence : CC-BY 4.0}
%	\framesubtitle{Licence: CC-BY-NC 4.0}

	\begin{tcolorbox}[colback=cyan,
		colframe=cyan,  
		arc=0pt,outer arc=0pt,
		valign=top, 
		halign=center,
		width=\textwidth]
		
		\includegraphics[width=.5cm]{../img/licence/cc_icon_white_x2.png}
		\includegraphics[width=.5cm]{../img/licence/attribution_icon_white_x2.png}
		
		\color{white}
		\bfseries Attribution 4.0 International (CC BY 4.0) \\
		\tiny \url{https://creativecommons.org/licenses/by/4.0/deed.fr}
		
	\end{tcolorbox}\vspace{-.5cm}
	\begin{tcolorbox}[colback=my-grey,
		colframe=my-grey,  
		center, arc=0pt,outer arc=0pt,
		valign=top, 
		halign=left,
		width=\textwidth]
		
		\tiny
		
		\begin{center}
			\bfseries\Large
			Vous êtes autorisé à :
		\end{center}
		
		\begin{minipage}{0.83\textwidth}
			\begin{itemize}
				\item[] \textbf{Partager} — copier, distribuer et communiquer le matériel par tous moyens et sous tous formats
				\item[] \textbf{Adapter} — remixer, transformer et créer à partir du matériel
				pour toute utilisation, y compris commerciale.
			\end{itemize}
		\end{minipage}
		\begin{minipage}{0.15\textwidth}
			\includegraphics[width=\textwidth]{../img/licence/FreeCulturalWorks_seal_x2.jpg}
		\end{minipage}
	
		
		\begin{center}
			\bfseries\Large
			Selon les conditions suivantes :
		\end{center}
		
		\begin{itemize}
			\item[] \textbf{Attribution} — Vous devez créditer l'Œuvre, intégrer un lien vers la licence et indiquer si des modifications ont été effectuées à l'Oeuvre. Vous devez indiquer ces informations par tous les moyens raisonnables, sans toutefois suggérer que l'Offrant vous soutient ou soutient la façon dont vous avez utilisé son Oeuvre. 
			\item[] \textbf{Pas de restrictions complémentaires} — Vous n'êtes pas autorisé à appliquer des conditions légales ou des mesures techniques qui restreindraient légalement autrui à utiliser l'Oeuvre dans les conditions décrites par la licence.
		\end{itemize}
		
	\end{tcolorbox}
	
%	\begin{center}
%		\bfseries Attribution 4.0 International (CC BY 4.0)
%		\url{https://creativecommons.org/licenses/by/4.0/deed.fr}
%	\end{center}

%	\tiny
%
%	Vous êtes autorisé à : 
%	\begin{itemize}
%		\item \textbf{Partager} — copier, distribuer et communiquer le matériel par tous moyens et sous tous formats
%		\item \textbf{Adapter} — remixer, transformer et créer à partir du matériel
%	\end{itemize}
%	
%	Selon les conditions suivantes : 
%	\begin{itemize}
%		\item \textbf{Attribution} — Vous devez créditer l'Œuvre, intégrer un lien vers la licence et indiquer si des modifications ont été effectuées à l'Oeuvre. Vous devez indiquer ces informations par tous les moyens raisonnables, sans toutefois suggérer que l'Offrant vous soutient ou soutient la façon dont vous avez utilisé son Oeuvre.
%		\item \textbf{Pas d'Utilisation Commerciale} — Vous n'êtes pas autorisé à faire un usage commercial de cette Oeuvre, tout ou partie du matériel la composant. 
%		\item \textbf{Pas de restrictions complémentaires} — Vous n'êtes pas autorisé à appliquer des conditions légales ou des mesures techniques qui restreindraient légalement autrui à utiliser l'Oeuvre dans les conditions décrites par la licence.
%	\end{itemize}

	\end{frame}
}

\settowidth{\leftmargini}{\usebeamertemplate{itemize item}}
\addtolength{\leftmargini}{\labelsep}

\AtBeginDocument{
	\newcolumntype{L}[2]{>{\vbox to #2\bgroup\vfill\flushleft}p{#1}<{\egroup}} 
	
	\begin{frame}[plain]
		\maketitle
	\end{frame}

	\insertlicence
}


% needs etoolbox; to break links after -
\appto\UrlBreaks{\do\-}


\lstdefinelanguage{CSS}{
	keywords={accelerator,azimuth,background,background-attachment,
		background-color,background-image,background-position,
		background-position-x,background-position-y,background-repeat,
		behavior,border,border-bottom,border-bottom-color,
		border-bottom-style,border-bottom-width,border-collapse,
		border-color,border-left,border-left-color,border-left-style,
		border-left-width,border-right,border-right-color,
		border-right-style,border-right-width,border-spacing,
		border-style,border-top,border-top-color,border-top-style,
		border-top-width,border-width,bottom,caption-side,clear,
		clip,color,content,counter-increment,counter-reset,cue,
		cue-after,cue-before,cursor,direction,display,elevation,
		empty-cells,filter,float,font,font-family,font-size,
		font-size-adjust,font-stretch,font-style,font-variant,
		font-weight,height,ime-mode,include-source,
		layer-background-color,layer-background-image,layout-flow,
		layout-grid,layout-grid-char,layout-grid-char-spacing,
		layout-grid-line,layout-grid-mode,layout-grid-type,left,
		letter-spacing,line-break,line-height,list-style,
		list-style-image,list-style-position,list-style-type,margin,
		margin-bottom,margin-left,margin-right,margin-top,
		marker-offset,marks,max-height,max-width,min-height,
		min-width,-moz-binding,-moz-border-radius,
		-moz-border-radius-topleft,-moz-border-radius-topright,
		-moz-border-radius-bottomright,-moz-border-radius-bottomleft,
		-moz-border-top-colors,-moz-border-right-colors,
		-moz-border-bottom-colors,-moz-border-left-colors,-moz-opacity,
		-moz-outline,-moz-outline-color,-moz-outline-style,
		-moz-outline-width,-moz-user-focus,-moz-user-input,
		-moz-user-modify,-moz-user-select,orphans,outline,
		outline-color,outline-style,outline-width,overflow,
		overflow-X,overflow-Y,padding,padding-bottom,padding-left,
		padding-right,padding-top,page,page-break-after,
		page-break-before,page-break-inside,pause,pause-after,
		pause-before,pitch,pitch-range,play-during,position,quotes,
		-replace,richness,right,ruby-align,ruby-overhang,
		ruby-position,-set-link-source,size,speak,speak-header,
		speak-numeral,speak-punctuation,speech-rate,stress,
		scrollbar-arrow-color,scrollbar-base-color,
		scrollbar-dark-shadow-color,scrollbar-face-color,
		scrollbar-highlight-color,scrollbar-shadow-color,
		scrollbar-3d-light-color,scrollbar-track-color,table-layout,
		text-align,text-align-last,text-decoration,text-indent,
		text-justify,text-overflow,text-shadow,text-transform,
		text-autospace,text-kashida-space,text-underline-position,top,
		unicode-bidi,-use-link-source,vertical-align,visibility,
		voice-family,volume,white-space,widows,width,word-break,
		word-spacing,word-wrap,writing-mode,z-index,zoom},  
	sensitive=true,
	morecomment=[l]{//},
	morecomment=[s]{/*}{*/},
	morestring=[b]',
	morestring=[b]",
	alsoletter={:},
	alsodigit={-}
}
\lstdefinelanguage{HTML5}{
	language=html,
	sensitive=true, 
	alsoletter={<>=-},
	otherkeywords={
		% HTML tags
		<, </, >,
		</a, <a, </a>,
		</abbr, <abbr, </abbr>,
		</address, <address, </address>,
		</area, <area, </area>,
		</area, <area, </area>,
		</article, <article, </article>,
		</aside, <aside, </aside>,
		</audio, <audio, </audio>,
		</audio, <audio, </audio>,
		</b, <b, </b>,
		</base, <base, </base>,
		</bdi, <bdi, </bdi>,
		</bdo, <bdo, </bdo>,
		</blockquote, <blockquote, </blockquote>,
		</body, <body, </body>,
		</br, <br, </br>,
		</button, <button, </button>,
		</canvas, <canvas, </canvas>,
		</caption, <caption, </caption>,
		</cite, <cite, </cite>,
		</code, <code, </code>,
		</col, <col, </col>,
		</colgroup, <colgroup, </colgroup>,
		</data, <data, </data>,
		</datalist, <datalist, </datalist>,
		</dd, <dd, </dd>,
		</del, <del, </del>,
		</details, <details, </details>,
		</dfn, <dfn, </dfn>,
		</div, <div, </div>,
		</dl, <dl, </dl>,
		</dt, <dt, </dt>,
		</em, <em, </em>,
		</embed, <embed, </embed>,
		</fieldset, <fieldset, </fieldset>,
		</figcaption, <figcaption, </figcaption>,
		</figure, <figure, </figure>,
		</footer, <footer, </footer>,
		</form, <form, </form>,
		</h1, <h1, </h1>,
		</h2, <h2, </h2>,
		</h3, <h3, </h3>,
		</h4, <h4, </h4>,
		</h5, <h5, </h5>,
		</h6, <h6, </h6>,
		</head, <head, </head>,
		</header, <header, </header>,
		</hr, <hr, </hr>,
		</html, <html, </html>,
		</i, <i, </i>,
		</iframe, <iframe, </iframe>,
		</img, <img, </img>,
		</input, <input, </input>,
		</ins, <ins, </ins>,
		</kbd, <kbd, </kbd>,
		</keygen, <keygen, </keygen>,
		</label, <label, </label>,
		</legend, <legend, </legend>,
		</li, <li, </li>,
		</link, <link, </link>,
		</main, <main, </main>,
		</map, <map, </map>,
		</mark, <mark, </mark>,
		</math, <math, </math>,
		</menu, <menu, </menu>,
		</menuitem, <menuitem, </menuitem>,
		</meta, <meta, </meta>,
		</meter, <meter, </meter>,
		</nav, <nav, </nav>,
		</noscript, <noscript, </noscript>,
		</object, <object, </object>,
		</ol, <ol, </ol>,
		</optgroup, <optgroup, </optgroup>,
		</option, <option, </option>,
		</output, <output, </output>,
		</p, <p, </p>,
		</param, <param, </param>,
		</pre, <pre, </pre>,
		</progress, <progress, </progress>,
		</q, <q, </q>,
		</rp, <rp, </rp>,
		</rt, <rt, </rt>,
		</ruby, <ruby, </ruby>,
		</s, <s, </s>,
		</samp, <samp, </samp>,
		</script, <script, </script>,
		</section, <section, </section>,
		</select, <select, </select>,
		</small, <small, </small>,
		</source, <source, </source>,
		</span, <span, </span>,
		</strong, <strong, </strong>,
		</style, <style, </style>,
		</summary, <summary, </summary>,
		</sup, <sup, </sup>,
		</svg, <svg, </svg>,
		</table, <table, </table>,
		</tbody, <tbody, </tbody>,
		</td, <td, </td>,
		</template, <template, </template>,
		</textarea, <textarea, </textarea>,
		</tfoot, <tfoot, </tfoot>,
		</th, <th, </th>,
		</thead, <thead, </thead>,
		</time, <time, </time>,
		</title, <title, </title>,
		</tr, <tr, </tr>,
		</track, <track, </track>,
		</u, <u, </u>,
		</ul, <ul, </ul>,
		</var, <var, </var>,
		</video, <video, </video>,
		</wbr, <wbr, </wbr>,
		/>, <!
	},  
	ndkeywords={
		% General
		=,
		% HTML attributes
		accept=, accept-charset=, accesskey=, action=, align=, alt=, async=, autocomplete=, autofocus=, autoplay=, autosave=, bgcolor=, border=, buffered=, challenge=, charset=, checked=, cite=, class=, code=, codebase=, color=, cols=, colspan=, content=, contenteditable=, contextmenu=, controls=, coords=, data=, datetime=, default=, defer=, dir=, dirname=, disabled=, download=, draggable=, dropzone=, enctype=, for=, form=, formaction=, headers=, height=, hidden=, high=, href=, hreflang=, http-equiv=, icon=, id=, ismap=, itemprop=, keytype=, kind=, label=, lang=, language=, list=, loop=, low=, manifest=, max=, maxlength=, media=, method=, min=, multiple=, name=, novalidate=, open=, optimum=, pattern=, ping=, placeholder=, poster=, preload=, pubdate=, radiogroup=, readonly=, rel=, required=, reversed=, rows=, rowspan=, sandbox=, scope=, scoped=, seamless=, selected=, shape=, size=, sizes=, span=, spellcheck=, src=, srcdoc=, srclang=, start=, step=, style=, summary=, tabindex=, target=, title=, type=, usemap=, value=, width=, wrap=,
		% CSS properties
		accelerator:,azimuth:,background:,background-attachment:,
		background-color:,background-image:,background-position:,
		background-position-x:,background-position-y:,background-repeat:,
		behavior:,border:,border-bottom:,border-bottom-color:,
		border-bottom-style:,border-bottom-width:,border-collapse:,
		border-color:,border-left:,border-left-color:,border-left-style:,
		border-left-width:,border-right:,border-right-color:,
		border-right-style:,border-right-width:,border-spacing:,
		border-style:,border-top:,border-top-color:,border-top-style:,
		border-top-width:,border-width:,bottom:,caption-side:,clear:,
		clip:,color:,content:,counter-increment:,counter-reset:,cue:,
		cue-after:,cue-before:,cursor:,direction:,display:,elevation:,
		empty-cells:,filter:,float:,font:,font-family:,font-size:,
		font-size-adjust:,font-stretch:,font-style:,font-variant:,
		font-weight:,height:,ime-mode:,include-source:,
		layer-background-color:,layer-background-image:,layout-flow:,
		layout-grid:,layout-grid-char:,layout-grid-char-spacing:,
		layout-grid-line:,layout-grid-mode:,layout-grid-type:,left:,
		letter-spacing:,line-break:,line-height:,list-style:,
		list-style-image:,list-style-position:,list-style-type:,margin:,
		margin-bottom:,margin-left:,margin-right:,margin-top:,
		marker-offset:,marks:,max-height:,max-width:,min-height:,
		min-width:,transition-duration:,transition-property:,
		transition-timing-function:,transform:,
		-moz-transform:,-moz-binding:,-moz-border-radius:,
		-moz-border-radius-topleft:,-moz-border-radius-topright:,
		-moz-border-radius-bottomright:,-moz-border-radius-bottomleft:,
		-moz-border-top-colors:,-moz-border-right-colors:,
		-moz-border-bottom-colors:,-moz-border-left-colors:,-moz-opacity:,
		-moz-outline:,-moz-outline-color:,-moz-outline-style:,
		-moz-outline-width:,-moz-user-focus:,-moz-user-input:,
		-moz-user-modify:,-moz-user-select:,orphans:,outline:,
		outline-color:,outline-style:,outline-width:,overflow:,
		overflow-X:,overflow-Y:,padding:,padding-bottom:,padding-left:,
		padding-right:,padding-top:,page:,page-break-after:,
		page-break-before:,page-break-inside:,pause:,pause-after:,
		pause-before:,pitch:,pitch-range:,play-during:,position:,quotes:,
		-replace:,richness:,right:,ruby-align:,ruby-overhang:,
		ruby-position:,-set-link-source:,size:,speak:,speak-header:,
		speak-numeral:,speak-punctuation:,speech-rate:,stress:,
		scrollbar-arrow-color:,scrollbar-base-color:,
		scrollbar-dark-shadow-color:,scrollbar-face-color:,
		scrollbar-highlight-color:,scrollbar-shadow-color:,
		scrollbar-3d-light-color:,scrollbar-track-color:,table-layout:,
		text-align:,text-align-last:,text-decoration:,text-indent:,
		text-justify:,text-overflow:,text-shadow:,text-transform:,
		text-autospace:,text-kashida-space:,text-underline-position:,top:,
		unicode-bidi:,-use-link-source:,vertical-align:,visibility:,
		voice-family:,volume:,white-space:,widows:,width:,word-break:,
		word-spacing:,word-wrap:,writing-mode:,z-index:,zoom:
	},  
	morecomment=[s]{<!--}{-->},
	tag=[s]
}

%\usepackage{color}
%\definecolor{editorGray}{rgb}{0.95, 0.95, 0.95}
%\definecolor{editorOcher}{rgb}{1, 0.5, 0} % #FF7F00 -> rgb(239, 169, 0)
%\definecolor{editorGreen}{rgb}{0, 0.5, 0} % #007C00 -> rgb(0, 124, 0)
%
%\lstset{%
%	% Basic design
%	backgroundcolor=\color{editorGray},
%	basicstyle={\small\ttfamily},   
%	frame=l,
%	% Line numbers
%	xleftmargin={0.75cm},
%	numbers=left,
%	stepnumber=1,
%	firstnumber=1,
%	numberfirstline=true,
%	% Code design   
%	keywordstyle=\color{blue}\bfseries,
%	commentstyle=\color{darkgray}\ttfamily,
%	ndkeywordstyle=\color{editorGreen}\bfseries,
%	stringstyle=\color{editorOcher},
%	% Code
%	language=HTML5,
%	alsodigit={.:;},
%	tabsize=2,
%	showtabs=false,
%	showspaces=false,
%	showstringspaces=false,
%	extendedchars=true,
%	breaklines=true,        
%}

\lstset{language=CSS,
	basicstyle=\ttfamily,
	keywordstyle=\color{blue}\ttfamily,
	stringstyle=\color{red}\ttfamily,
	commentstyle=\color{green}\ttfamily,
	morecomment=[l][\color{magenta}]{\#}
}

\title[TALN : 01- Introduction]%
{Traitement automatique du langage naturel\\Chapitre 01 : Introduction} 

\changegraphpath{../img/intro/}

\begin{document}
	
\begin{frame}
\frametitle{Traitement automatique du langage naturel}
\framesubtitle{Introduction : Définition}

\begin{itemize}
	\item \keyword{TALN} : Traitement automatique du langage naturel
	\item \keyword{TAL} : Traitement automatique des langues
	\item l'ensemble des méthodes permettant de rendre le langage humain accessible aux ordinateurs.
\end{itemize}
\begin{minipage}{0.78\textwidth}
\begin{itemize}
	\item Un domaine multidisciplinaire
	\begin{itemize}
		\item \optword{Linguistique} : étude du langage
		\item \optword{Informatique} : Traitement automatique de l'information
		\item \optword{Intelligence artificielle} : Ensemble de théories et de techniques mises en œuvre en vue de réaliser des machines capables de simuler l'intelligence humaine.
	\end{itemize}
\end{itemize}
\end{minipage}
\begin{minipage}{0.20\textwidth}
\hgraphpage[\textwidth]{TALN.pdf}
\end{minipage}
\end{frame}

\begin{frame}
\frametitle{Traitement automatique du langage naturel}
\framesubtitle{Introduction : Motivation}

\begin{itemize}
	\item Augmenter la productivité en utilisant des applications comme la traduction automatique et le résumé automatique (pourtant ces deux applications sont loin d'être parfaites)
	
	\item Service Clientèle : la réponse automatique aux questions des clients en utilisant les chatbots (question-réponse et reconnaissance de voix). 
	
	\item Surveillance de la réputation : on utilise l'analyse des sentiments pour savoir si les clients sont heureux avec ses produits ou non. 
	
	\item La publicité : on scannant les réseaux sociaux et les courriels, on peut savoir qui est intéressé par ses produits. Ceci permet aux entreprises de viser l'audience de la publicité. 
	
	\item Connaissance du marché (Market intelligence) : surveiller les compétiteurs afin de se tenir au courant des évènements liés à l'industrie.
\end{itemize}

\end{frame}

\begin{frame}
\frametitle{Rédaction d'un document numérique}
\framesubtitle{Introduction : Plan}

\begin{multicols}{2}
%	\small
\tableofcontents
\end{multicols}
\end{frame}

%===================================================================================
\section{Histoire}
%===================================================================================

\begin{frame}
\frametitle{Introduction au TALN}
\framesubtitle{Histoire}

\hgraphpage{histoire.pdf}

\end{frame}

\subsection{Naissance de l'IA et âge d'or}

\begin{frame}
\frametitle{Introduction au TALN : Histoire}
\framesubtitle{Naissance de l'IA et âge d'or : Les années 195x}

\begin{itemize}
	\item \optword{1951} Shannon a exploité les modèles probabilistes des langages naturels \cite{1951-shannon}.
	\item \optword{1954} expérimentation Georgetown-IBM pour traduire automatiquement 60 phrases du russe vers l'anglais.
	\item \optword{1956} Chomsky a développé les modèles formels de syntaxe.
	\item \optword{1958} Luhn (IBM) a expérimenté sur le résumé automatiquement du texte par extraction \cite{1958-luhn}
\end{itemize}

\end{frame}

\begin{frame}
\frametitle{Introduction au TALN : Histoire}
\framesubtitle{Naissance de l'IA et âge d'or : Les années 196x}

\begin{itemize}
	\item \optword{1961} Développement du premier analyseur syntaxique automatique à U. Penn. \cite{1961-joshi,1962-harris} 
	\item \optword{1964} Weizenbaum a mis au point \keyword{ELIZA}, une simulation d'un psychothérapeute au sein du laboratoire MIT AI.
	\item \optword{1964} Bobrow a mis au point \keyword{STUDENT}, conçu pour lire et résoudre des problèmes de mots trouvés dans les livres d'algèbre de lycée \cite{1964-bobrow}.
	\item \optword{1967} Brown corpus, le premier corpus électronique.
\end{itemize}

\end{frame}

\subsection{Hiver de l'IA}

\begin{frame}
\frametitle{Introduction au TALN : Histoire}
\framesubtitle{Hiver de l'IA : Les années 197x}

\begin{itemize}
	\item \optword{1971} Winograd (MIT) a développé \keyword{SHRDLU}, un programme de compréhension du langage naturel \cite{1971-winograd}.
	\item \optword{1972} Colby (Stanford) a créé \keyword{PARRY} un chatbot qui simule une personne avec la schizophrénie paranoïde.
	\item \optword{1975} \keyword{MARGIE} un système qui fait des inférences et des paraphrases à partir des phrases en utilisant la représentation conceptuelle du langage. 
	\item \optword{1975} \keyword{DRAGON}, un système pour la reconnaissance automatique de la parole en utilisant les modèles de Markov cachés \cite{1975-baker}.
\end{itemize}

\end{frame}

\begin{frame}
\frametitle{Introduction au TALN : Histoire}
\framesubtitle{Hiver de l'IA : Les années 198x}

\begin{itemize}
	\item \optword{1980} \keyword{KL-One}, représentation de connaissance pour le traitement de la syntaxe et la sémantique \cite{1980-bobrow}
	\item \optword{1986} \keyword{TRUMP}, analyseur de langage en utilisant une base lexicale \cite{1986-jacobs}
	\item \optword{1987} \keyword{HPSG} (head-driven phrase structure grammar, traduction française : grammaire syntagmatique guidée par les têtes) \cite{1987-sag-pollard}
	\item \optword{1987} \keyword{MUC} conférence sur l'extraction des données financée par \keyword{DARPA}
	\item \optword{1988} Utilisation des models de markov cachés dans  l'étiquetage morpho-syntaxique \cite{1988-church}
	\item Solutions symboliques sur le traitement du discours et la génération du langage naturel.
\end{itemize}

\end{frame}

\subsection{Printemps de l'IA}

\begin{frame}
\frametitle{Introduction au TALN : Histoire}
\framesubtitle{Printemps de l'IA : Les années 199x}

\begin{itemize}
	\item \optword{1990} Une approche statistique pour la traduction automatique \cite{1990-brown-al}
	\item \optword{1993} \keyword{Pen Treebank}, un corpus annoté de l'anglais \cite{1993-marcus-al}
	\item \optword{1995} \keyword{Wordnet}, une base lexicale pour l'anglais \cite{1995-miller}
	\item \optword{1996} \keyword{SPATTER}, un analyseur lexical statistique basé sur les arbres de décision \cite{1996-magerman}
	\item Popularité des méthodes statistiques et de l'évaluation empirique
\end{itemize}

\end{frame}

\begin{frame}
\frametitle{Introduction au TALN : Histoire}
\framesubtitle{Printemps de l'IA : Les années 200x}

\begin{itemize}
	\item \optword{2003} Les modèles probabilistes de langues en utilisant les réseaux de neurones \cite{2003-bengio-al}
	\item \optword{2006} \keyword{Watson} (IBM), un système de question/réponse
	\item Utilisation de l'apprentissage non supervisé et semi-supervisé comme alternatives à l'apprentissage purement supervisé.
	\item Déplacer le focus sur les tâches sémantiques.
\end{itemize}

\end{frame}

\begin{frame}
\frametitle{Introduction au TALN : Histoire}
\framesubtitle{Printemps de l'IA : Les années 201x}

\begin{itemize}
	\item \optword{2011} \keyword{Siri} (Apple)  un assistant numérique personnel. Il a été suivi par \keyword{Alexa} (Amazon, \optword{2014}) et \optword{Google Assistant} (\optword{2016})
	\item \optword{2014} Word embedding \cite{2014-lebret-collobert}
	\item \optword{2018} L'apparition des représentations contextuelles (des modèles de langue pré-entraînés) : \keyword{ULMfit} (fast.ai) \cite{2018-howard-ruder}, \keyword{ELMO} (AllenNLP) \cite{2018-peters-al}, \keyword{GPT} (OpenAI) \cite{2018-radford-al}, \keyword{BERT} (Google) \cite{2018-devlin-al}, \keyword{XLM} (Facebook) \cite{2019-lample-conneau}
\end{itemize}

\end{frame}

%===================================================================================
\section{Les niveaux de traitement d'une langue}
%===================================================================================

\begin{frame}
\frametitle{Introduction au TALN}
\framesubtitle{Les niveaux de traitement d'une langue}

\hgraphpage{niveaux.pdf}

\end{frame}

\subsection{Phonétique, phonologie et orthographe}

\begin{frame}
\frametitle{Introduction au TALN : Les niveaux}
\framesubtitle{Phonétique, phonologie et orthographe : Phonétique}

\begin{itemize}
	\item étude des sons ou phones produits par l'appareil phonatoire humain
	\item Les branches
	\begin{itemize}
		\item \optword{Phonétique articulatoire} (la division la plus anatomique et physiologique) décrit comment les voyelles et les consonnes sont produites ou «articulées» dans diverses parties de la bouche et de la gorge.
		\item \optword{Phonétique acoustique} (la branche qui a les affinités les plus étroites avec la physique) étudie les ondes sonores qui transmettent les voyelles et les consonnes dans l'air du locuteur à l'auditeur.
		\item \optword{Phonétique auditive} (la branche qui intéresse le plus les psychologues) examine la manière dont le cerveau de l'auditeur décode les ondes sonores en voyelles et consonnes initialement prévues par le locuteur.
	\end{itemize}
\end{itemize}

\end{frame}

\begin{frame}
\frametitle{Introduction au TALN : Les niveaux}
\framesubtitle{Phonétique, phonologie et orthographe : Phonétique (Points d'articulation des consonnes)}

\begin{minipage}{0.5\textwidth}
\begin{itemize}
	\item \optword{labial} à l'aide des lèvres. Exemple, \expword{\textipa{[b], [p], [m], [f], [v]}}
	\item \optword{apicale} avec la pointe de la langue ou sa partie antérieure. 
	Exemple, \expword{%
		\textipa{[t], [d], [n], [r], }
		\RL{^s} \textipa{[S],} 
		\RL{_t} \textipa{[T],} 
		\RL{_d} \textipa{[D]}
	}
	\item \optword{dorsal} avec la partie postérieure de la langue. Exemple, \expword{\textipa{[c], [k], [g], [q],} \RL{.g} \textipa{[G]}}
	\item \optword{pharyngale} au niveau du pharynx. 
	Exemple, \expword{\RL{.h} \textipa{[\*h],} \RL{`} \textipa{[Q]}}
	\item \optword{glottale} au niveau de la glotte. 
	Exemple, \expword{\textipa{[h],} \RL{'} \textipa{[P]}}
\end{itemize}
\end{minipage}
\begin{minipage}{0.48\textwidth}
	\begin{figure}
		\hgraphpage{oraltract.pdf}\caption{Voie orale \cite{2009-ball}}
	\end{figure}
\end{minipage}

\end{frame}

\begin{frame}
\frametitle{Introduction au TALN : Les niveaux}
\framesubtitle{Phonétique, phonologie et orthographe : Phonétique (Modes d'articulation des consonnes)}

\begin{itemize}
	\item \optword{occlusive} blocage complet de l'écoulement de l'air.
	Exemple, \expword{\textipa{[p], [k], [b], [m], [n]}}
	
	\item \optword{fricative} resserrement de la bouche, du pharynx ou de la glotte sans qu'il y ait fermeture complète de ceux-ci.
	Exemple, \expword{\textipa{[f], [v], [s]}}
	
	\item \optword{affriquée} le flux d'air est bloqué ensuite relâché.
	Exemple, \expword{\textipa{[\t{\textteshlig}]}}
	
	\item \optword{latérale} écoulement de l'air par un canal latéral (parfois bilatéral).
	Exemple, \expword{\textipa{[l]}}
	
	\item \optword{nasale} en abaissant le voile du palais.
	Exemple, \expword{\textipa{[m], [n]}}
	
	\item \optword{clic} avec la langue ou les lèvres sans l'aide des poumons.
	Exemple, \expword{\textipa{[!]} (claquement de langue)}
	
\end{itemize}

\end{frame}

\begin{frame}
\frametitle{Introduction au TALN : Les niveaux}
\framesubtitle{Phonétique, phonologie et orthographe : Phonétique (IPA)}

\begin{center}
	\vgraphpage{IPA1.pdf}
\end{center}

\end{frame}

\begin{frame}
\frametitle{Introduction au TALN : Les niveaux}
\framesubtitle{Phonétique, phonologie et orthographe : Phonologie}

\begin{itemize}
	\item étude des sons ou phonèmes d'une langue donnée
	\item s'intéresse aux sons en tant qu'éléments d'un système
\end{itemize}

\begin{exampleblock}{Exemple : le phonème \textit{/r/}}
	\begin{itemize}
		\item En français, le \textit{r} peut se prononcer (en phonétique) : roulé \expword{\textipa{[r]}}, grasseyé \expword{\textipa{[\;R]}}, ou normal (parisien) \expword{\textipa{[K]}}
		\item Il est transcrit toujours de la même façon, exemple \expword{rat /rat/}
		\item En arabe, il y a les consonnes \expword{\RL{r} \textipa{[r]}} et \expword{\RL{.g} \textipa{[G]}} qui ont deux phonèmes différents : \expword{\textipa{/r/}} et \expword{\textipa{/G/}} respectivement. 
		Exemple, \expword{\RL{.gryb} \textipa{/G\ae ri:b/} (étranger)}
	\end{itemize}
\end{exampleblock}

\end{frame}

\begin{frame}
\frametitle{Introduction au TALN : Les niveaux}
\framesubtitle{Phonétique, phonologie et orthographe : Orthographe}

\begin{itemize}
	\item étude des types et de la forme des lemmes/monèmes 
	\item Système d'écriture
	\begin{itemize}
		\item \optword{logographique} : logogrammes, chacun est un unique graphème notant un lemme (mot).
		Exemple, \expword{
			Kanji (Japonais) : 
			\begin{CJK}{UTF8}{min}日, 本, 語\end{CJK}
		}
		\item \optword{syllabique} : symboles, chacun représente un syllabe (son vocalisé). 
		Exemple, \expword{
			Hiragana (Japonais) : 
			\begin{CJK}{UTF8}{min}る, た, め, の\end{CJK} ; 
			Katakana (Japonais) : 
			\begin{CJK}{UTF8}{min}セ, ク\end{CJK} ; 
		}
		\item \optword{alphabétique} : lettres, chacune d'elles représente un phonème. 
		Exemple, \expword{Le latin : A, B, C, etc. L'arabe : \RL{b}, \RL{t}, \RL{h}}
	\end{itemize}
	\item Ponctuation 
	\item Les règles d'écriture 
\end{itemize}

\end{frame}

\subsection{Morphologie et syntaxe}

\begin{frame}
\frametitle{Introduction au TALN : Les niveaux}
\framesubtitle{Morphologie et syntaxe : Morphologie}

\begin{itemize}
	\item l'étude de la formation des mots, y compris la façon dont les nouveaux mots sont inventés dans les langues du monde
	\item l'étude de la façon dont les formes des mots varient en fonction de leurs utilisations dans les phrases.
	\item \keyword{Morphème} la plus petite unité de langage qui a sa propre signification. Exemple, \expword{Les noms propres, les suffixes, etc.}
	\item \keyword{Lexème} un ensemble de toutes les formes grammaticales qui ont le même sens
	\item \keyword{Lemme} un mot choisi parmi ces formes pour représenter le lexème
	\item Les catégories grammaticales : classe ouverte (\optword{adjectif}, \optword{nom}, \optword{verbe}) et classe fermée (\optword{adverbe}, \optword{article}, \optword{conjonction}, \optword{interjection},  \optword{préposition}, \optword{pronom}).
\end{itemize}

\end{frame}


\begin{frame}
\frametitle{Introduction au TALN : Les niveaux}
\framesubtitle{Morphologie et syntaxe : Morphologie (Typologie morphologique des langues)}

\begin{itemize}
	\item \optword{Langues isolantes/analytiques} chaque mot est constitué d'un et d'un seul morphème. les modifications morphologiques sont peu nombreuses, voire absentes : \expword{mandarin, vietnamien, thaï, khmer, etc.}. Exemple, \expword{\begin{CJK}{UTF8}{min}四个男孩\end{CJK} /sì ge nánhái/ "quatre garçons" (lit. "quatre [entité de] masculin enfant")}
	\item \optword{Les langues flexionnelles/synthétiques} les mots sont formés d'une \keyword{racine} en plus de morphèmes supplémentaires.
	\begin{itemize}
		\item \optword{Langues agglutinantes} les morphèmes sont toujours clairement différentiables phonétiquement l'un de l'autre : \expword{finnois, turc, japonais, langues berbères, etc.}. Exemple, \expword{\begin{CJK}{UTF8}{min}行く, 行きます\end{CJK}} 
		\item \optword{Langues fusionnelles} il n'est pas toujours aisé de distinguer les morphèmes de la racine, ou les morphèmes les uns des autres : \expword{arabe, anglais, français, etc.}. Exemple, \expword{\RL{kitAb, kutub, 'wa'`.taynAkumuwh}, foot, feet}
	\end{itemize}
\end{itemize}

\end{frame}

\begin{frame}
\frametitle{Introduction au TALN : Les niveaux}
\framesubtitle{Morphologie et syntaxe : Morphologie flexionnelle}

\begin{itemize}
	\item la formation de mots sans changer de catégorie ou créer de nouveaux lexèmes 
	\item modification de la forme des lexèmes afin qu'ils s'adaptent à différents contextes grammaticaux
	\item la flexion peut être une \optword{déclinaison} ou une \optword{conjugaison}
	\item \optword{affixation}
	\begin{itemize}
		\item \optword{préfixe} \expword{\RL{_dhb}} (passé), \expword{\RL{y_dhb}} (présent)
		\item \optword{infixe}
		\item \optword{suffixe} \expword{étudiant} (masculin-singulier), \expword{étudiantes} (féminin-pluriel)
	\end{itemize}
	\item \optword{redoublement} \expword{super-duper, bye-bye, \begin{CJK}{UTF8}{min}きらきら\end{CJK}} 
	
\end{itemize}


\end{frame}

\begin{frame}
\frametitle{Introduction au TALN : Les niveaux}
\framesubtitle{Morphologie et syntaxe : Morphologie flexionnelle (traits grammaticaux)}

\begin{itemize}
	\item \optword{Nombre} : singulier, duel, triel, paucal, pluriel, etc. 
	\item \optword{Personne} : première, deuxième, troisième, etc.
	\item \optword{Genre} : masculin, féminin, neutre, commun 
	\item \optword{Cas} : nominatif, absolutif, accusatif, ergatif, etc.
	\item \optword{Temps} : passé, présent, future
	\item \optword{Aspect} : accompli/inaccompli, progressif, perfectif/imperfectif, itératif, prospectif
	\item \optword{Voix} : active, moyenne, passive, etc.
	\item \optword{Polarité} : affirmative, négative
	\item \optword{Politesse} : informelle, formelle, etc.
\end{itemize}

\end{frame}

\begin{frame}
\frametitle{Introduction au TALN : Les niveaux}
\framesubtitle{Morphologie et syntaxe : Morphologie dérivationnelle}

\begin{itemize}
	\item la formation de mots en changeant de catégorie (\expword{jouer, joueur}) ou en créant de nouveaux lexèmes (\expword{connecter, déconnecter})
	
	\item \optword{affixation} en utilisant des préfixes, infixes et/ou suffixes. 
	Exemple, \expword{happy (ADJ), unhappy (ADJ), unhapyness (N)}; \expword{\RL{jhd} (V), \RL{ijthd} (V)}
	\item \optword{composition} en fusionnant des mots  dans un seul. 
	Exemple, \expword{porter (V) + manteau (N) = portemanteau (N)}; \expword{wind (N), mill (N), windmill (N)}
	
	\item \optword{conversion} le mot change de catégorie grammaticale sans aucune modification. 
	Exemple, \expword{orange (fruit, N), orange (couleur, ADJ)}; \expword{visit-er (V), visite (N)}; \expword{fish (N), to fish (V)}
	
	\item \optword{troncation} le mot est tronqué. 
	Exemple, \expword{bibliographie, biblio}, \expword{information, info}, \expword{fish (N), to fish (V)}
	
	\item \optword{redoublement} Exemple, \expword{\RL{kr} (V), \RL{krkr} (V)} 
	
\end{itemize}

\end{frame}

\begin{frame}
\frametitle{Introduction au TALN : Les niveaux}
\framesubtitle{Morphologie et syntaxe : Syntaxe}

\begin{itemize}
	\item 
\end{itemize}

\end{frame}

\subsection{Sémantique}

\begin{frame}
\frametitle{Introduction au TALN : Les niveaux}
\framesubtitle{Sémantique}

\begin{itemize}
	\item 
\end{itemize}

\end{frame}

\subsection{Pragmatique et discours}

\begin{frame}
\frametitle{Introduction au TALN : Les niveaux}
\framesubtitle{Pragmatique et discours : Pragmatique}

\begin{itemize}
	\item 
\end{itemize}

\end{frame}

\begin{frame}
\frametitle{Introduction au TALN : Les niveaux}
\framesubtitle{Pragmatique et discours : Discours}

\begin{itemize}
	\item 
\end{itemize}

\end{frame}

%===================================================================================
\section{Les applications du TALN}
%===================================================================================

\begin{frame}
\frametitle{Introduction au TALN}
\framesubtitle{Les applications du TALN}

\end{frame}

%===================================================================================
\section{Les défis du TALN}
%===================================================================================

\begin{frame}
\frametitle{Introduction au TALN}
\framesubtitle{Les défis du TALN}

\end{frame}



\insertbibliography{TALN01}{*}

\end{document}

