% !TEX TS-program = pdflatex
% !TeX program = pdflatex
% !TEX encoding = UTF-8
% !TEX spellcheck = fr

\documentclass[xcolor=table]{beamer}


%\usepackage{fullpage}
%\usepackage[left=2.8cm,right=2.2cm,top=2 cm,bottom=2 cm]{geometry}
\setbeamersize{text margin left=10pt,text margin right=10pt}
\usepackage{amsmath,amssymb} 
\usepackage[T1]{fontenc}
\usepackage{textcomp}
\usepackage[utf8]{inputenc}
\usepackage[french]{babel}
\usepackage{arabtex}
\usepackage{txfonts}
\usepackage[]{graphicx}
\usepackage{multirow}
\usepackage{hyperref}
\usepackage{colortbl}
\usepackage{listingsutf8}
\usepackage{wrapfig}
\usepackage{multicol}
\usepackage[export]{adjustbox} %for images in table, also for frame
\usepackage[many]{tcolorbox}

\hypersetup{
	colorlinks,
	urlcolor = blue
}

%\renewcommand{\baselinestretch}{1.5}

\def\supit#1{\raisebox{0.8ex}{\small\it #1}\hspace{0.05em}}

\AtBeginSection{%
	\begin{frame}
		\sectionpage
	\end{frame}
}

\newcommand{\rottext}[2]{%
	\rotatebox{90}{%
	\begin{minipage}{#1}%
		\raggedleft#2%
	\end{minipage}%
	}%
}

\usepackage{longtable}
\usepackage{tabu}


\institute{ %
École  nationale Supérieure d'Informatique (ESI, ex. INI), Algérie
}
\author[ \textbf{\footnotesize\insertframenumber/\inserttotalframenumber} \hspace*{1.5cm} ESI - ARIES Abdelkrime (2020/2021)] %
{ARIES Abdelkrime}
%\titlegraphic{\includegraphics[height=1cm]{../img/esi-logo.png}%\hspace*{4.75cm}~


\date{Année universitaire : 2020/2021} %\today

\usetheme{Warsaw} % Antibes Boadilla Warsaw

\beamertemplatenavigationsymbolsempty

%\setbeamertemplate{headline}{}

\definecolor{lightblue}{HTML}{D0D2FF}
\definecolor{lightyellow}{HTML}{FFFFAA}
\definecolor{darkblue}{HTML}{0000BB}
\definecolor{olivegreen}{HTML}{006600}
\definecolor{violet}{HTML}{6600CC}

\newcommand{\keyword}[1]{\textcolor{red}{\bfseries\itshape #1}}
\newcommand{\expword}[1]{\textcolor{olivegreen}{#1}}
\newcommand{\optword}[1]{\textcolor{violet}{\bfseries #1}}

\makeatletter
\newcommand\mysphere{%
	\parbox[t]{10pt}{\raisebox{0.2pt}{\beamer@usesphere{item projected}{bigsphere}}}}
\makeatother

%\let\oldtabular\tabular
%\let\endoldtabular\endtabular
%\renewenvironment{tabular}{\rowcolors{2}{white}{lightblue}\oldtabular\rowcolor{blue}}{\endoldtabular}


\NoAutoSpacing %french autospacing after ":"

\def\graphpath{}

\newcommand{\changegraphpath}[1]{\def\graphpath{#1}}


\newcommand{\vgraphpage}[2][.84\textheight]{%
%	\begin{center}%
		\includegraphics[height=#1]{\graphpath #2}%
%	\end{center}%
}

\newcommand{\hgraphpage}[2][\textwidth]{%
%	\begin{center}%
		\includegraphics[width=#1]{\graphpath #2}%
%	\end{center}%
}

\newcommand{\graphpage}[2][]{%
	\includegraphics[#1]{\graphpath #2}%
}

\bibliographystyle{acm}

\newcommand{\insertbibliography}[2]{
	\appendix
	\section*{Bibliographie}
	\nocite{#2}
%	\makeatletter % to change template
%	\setbeamertemplate{headline}[default] % not mandatory, but I though it was better to set it blank
%	\def\beamer@entrycode{\vspace*{-\headheight}} % here is the part we are interested in :)
%	\makeatother
	\begin{multicols*}{2}[\frametitle{\insertsection} \usebeamertemplate*{frametitle}]%\usebeamertemplate*{frametitle}\frametitle{Références}
		\tiny
		\bibliography{#1}
	\end{multicols*}
}

\definecolor{my-grey}{RGB}{233, 233, 233}

\newcommand{\insertlicence}{
	\begin{frame}[plain]
	\frametitle{Licence : CC-BY 4.0}
%	\framesubtitle{Licence: CC-BY-NC 4.0}

	\begin{tcolorbox}[colback=cyan,
		colframe=cyan,  
		arc=0pt,outer arc=0pt,
		valign=top, 
		halign=center,
		width=\textwidth]
		
		\includegraphics[width=.5cm]{../img/licence/cc_icon_white_x2.png}
		\includegraphics[width=.5cm]{../img/licence/attribution_icon_white_x2.png}
		
		\color{white}
		\bfseries Attribution 4.0 International (CC BY 4.0) \\
		\tiny \url{https://creativecommons.org/licenses/by/4.0/deed.fr}
		
	\end{tcolorbox}\vspace{-.5cm}
	\begin{tcolorbox}[colback=my-grey,
		colframe=my-grey,  
		center, arc=0pt,outer arc=0pt,
		valign=top, 
		halign=left,
		width=\textwidth]
		
		\tiny
		
		\begin{center}
			\bfseries\Large
			Vous êtes autorisé à :
		\end{center}
		
		\begin{minipage}{0.83\textwidth}
			\begin{itemize}
				\item[] \textbf{Partager} — copier, distribuer et communiquer le matériel par tous moyens et sous tous formats
				\item[] \textbf{Adapter} — remixer, transformer et créer à partir du matériel
				pour toute utilisation, y compris commerciale.
			\end{itemize}
		\end{minipage}
		\begin{minipage}{0.15\textwidth}
			\includegraphics[width=\textwidth]{../img/licence/FreeCulturalWorks_seal_x2.jpg}
		\end{minipage}
	
		
		\begin{center}
			\bfseries\Large
			Selon les conditions suivantes :
		\end{center}
		
		\begin{itemize}
			\item[] \textbf{Attribution} — Vous devez créditer l'Œuvre, intégrer un lien vers la licence et indiquer si des modifications ont été effectuées à l'Oeuvre. Vous devez indiquer ces informations par tous les moyens raisonnables, sans toutefois suggérer que l'Offrant vous soutient ou soutient la façon dont vous avez utilisé son Oeuvre. 
			\item[] \textbf{Pas de restrictions complémentaires} — Vous n'êtes pas autorisé à appliquer des conditions légales ou des mesures techniques qui restreindraient légalement autrui à utiliser l'Oeuvre dans les conditions décrites par la licence.
		\end{itemize}
		
	\end{tcolorbox}
	
%	\begin{center}
%		\bfseries Attribution 4.0 International (CC BY 4.0)
%		\url{https://creativecommons.org/licenses/by/4.0/deed.fr}
%	\end{center}

%	\tiny
%
%	Vous êtes autorisé à : 
%	\begin{itemize}
%		\item \textbf{Partager} — copier, distribuer et communiquer le matériel par tous moyens et sous tous formats
%		\item \textbf{Adapter} — remixer, transformer et créer à partir du matériel
%	\end{itemize}
%	
%	Selon les conditions suivantes : 
%	\begin{itemize}
%		\item \textbf{Attribution} — Vous devez créditer l'Œuvre, intégrer un lien vers la licence et indiquer si des modifications ont été effectuées à l'Oeuvre. Vous devez indiquer ces informations par tous les moyens raisonnables, sans toutefois suggérer que l'Offrant vous soutient ou soutient la façon dont vous avez utilisé son Oeuvre.
%		\item \textbf{Pas d'Utilisation Commerciale} — Vous n'êtes pas autorisé à faire un usage commercial de cette Oeuvre, tout ou partie du matériel la composant. 
%		\item \textbf{Pas de restrictions complémentaires} — Vous n'êtes pas autorisé à appliquer des conditions légales ou des mesures techniques qui restreindraient légalement autrui à utiliser l'Oeuvre dans les conditions décrites par la licence.
%	\end{itemize}

	\end{frame}
}

\settowidth{\leftmargini}{\usebeamertemplate{itemize item}}
\addtolength{\leftmargini}{\labelsep}

\AtBeginDocument{
	\newcolumntype{L}[2]{>{\vbox to #2\bgroup\vfill\flushleft}p{#1}<{\egroup}} 
	
	\begin{frame}[plain]
		\maketitle
	\end{frame}

	\insertlicence
}


% needs etoolbox; to break links after -
\appto\UrlBreaks{\do\-}


\lstdefinelanguage{CSS}{
	keywords={accelerator,azimuth,background,background-attachment,
		background-color,background-image,background-position,
		background-position-x,background-position-y,background-repeat,
		behavior,border,border-bottom,border-bottom-color,
		border-bottom-style,border-bottom-width,border-collapse,
		border-color,border-left,border-left-color,border-left-style,
		border-left-width,border-right,border-right-color,
		border-right-style,border-right-width,border-spacing,
		border-style,border-top,border-top-color,border-top-style,
		border-top-width,border-width,bottom,caption-side,clear,
		clip,color,content,counter-increment,counter-reset,cue,
		cue-after,cue-before,cursor,direction,display,elevation,
		empty-cells,filter,float,font,font-family,font-size,
		font-size-adjust,font-stretch,font-style,font-variant,
		font-weight,height,ime-mode,include-source,
		layer-background-color,layer-background-image,layout-flow,
		layout-grid,layout-grid-char,layout-grid-char-spacing,
		layout-grid-line,layout-grid-mode,layout-grid-type,left,
		letter-spacing,line-break,line-height,list-style,
		list-style-image,list-style-position,list-style-type,margin,
		margin-bottom,margin-left,margin-right,margin-top,
		marker-offset,marks,max-height,max-width,min-height,
		min-width,-moz-binding,-moz-border-radius,
		-moz-border-radius-topleft,-moz-border-radius-topright,
		-moz-border-radius-bottomright,-moz-border-radius-bottomleft,
		-moz-border-top-colors,-moz-border-right-colors,
		-moz-border-bottom-colors,-moz-border-left-colors,-moz-opacity,
		-moz-outline,-moz-outline-color,-moz-outline-style,
		-moz-outline-width,-moz-user-focus,-moz-user-input,
		-moz-user-modify,-moz-user-select,orphans,outline,
		outline-color,outline-style,outline-width,overflow,
		overflow-X,overflow-Y,padding,padding-bottom,padding-left,
		padding-right,padding-top,page,page-break-after,
		page-break-before,page-break-inside,pause,pause-after,
		pause-before,pitch,pitch-range,play-during,position,quotes,
		-replace,richness,right,ruby-align,ruby-overhang,
		ruby-position,-set-link-source,size,speak,speak-header,
		speak-numeral,speak-punctuation,speech-rate,stress,
		scrollbar-arrow-color,scrollbar-base-color,
		scrollbar-dark-shadow-color,scrollbar-face-color,
		scrollbar-highlight-color,scrollbar-shadow-color,
		scrollbar-3d-light-color,scrollbar-track-color,table-layout,
		text-align,text-align-last,text-decoration,text-indent,
		text-justify,text-overflow,text-shadow,text-transform,
		text-autospace,text-kashida-space,text-underline-position,top,
		unicode-bidi,-use-link-source,vertical-align,visibility,
		voice-family,volume,white-space,widows,width,word-break,
		word-spacing,word-wrap,writing-mode,z-index,zoom},  
	sensitive=true,
	morecomment=[l]{//},
	morecomment=[s]{/*}{*/},
	morestring=[b]',
	morestring=[b]",
	alsoletter={:},
	alsodigit={-}
}
\lstdefinelanguage{HTML5}{
	language=html,
	sensitive=true, 
	alsoletter={<>=-},
	otherkeywords={
		% HTML tags
		<, </, >,
		</a, <a, </a>,
		</abbr, <abbr, </abbr>,
		</address, <address, </address>,
		</area, <area, </area>,
		</area, <area, </area>,
		</article, <article, </article>,
		</aside, <aside, </aside>,
		</audio, <audio, </audio>,
		</audio, <audio, </audio>,
		</b, <b, </b>,
		</base, <base, </base>,
		</bdi, <bdi, </bdi>,
		</bdo, <bdo, </bdo>,
		</blockquote, <blockquote, </blockquote>,
		</body, <body, </body>,
		</br, <br, </br>,
		</button, <button, </button>,
		</canvas, <canvas, </canvas>,
		</caption, <caption, </caption>,
		</cite, <cite, </cite>,
		</code, <code, </code>,
		</col, <col, </col>,
		</colgroup, <colgroup, </colgroup>,
		</data, <data, </data>,
		</datalist, <datalist, </datalist>,
		</dd, <dd, </dd>,
		</del, <del, </del>,
		</details, <details, </details>,
		</dfn, <dfn, </dfn>,
		</div, <div, </div>,
		</dl, <dl, </dl>,
		</dt, <dt, </dt>,
		</em, <em, </em>,
		</embed, <embed, </embed>,
		</fieldset, <fieldset, </fieldset>,
		</figcaption, <figcaption, </figcaption>,
		</figure, <figure, </figure>,
		</footer, <footer, </footer>,
		</form, <form, </form>,
		</h1, <h1, </h1>,
		</h2, <h2, </h2>,
		</h3, <h3, </h3>,
		</h4, <h4, </h4>,
		</h5, <h5, </h5>,
		</h6, <h6, </h6>,
		</head, <head, </head>,
		</header, <header, </header>,
		</hr, <hr, </hr>,
		</html, <html, </html>,
		</i, <i, </i>,
		</iframe, <iframe, </iframe>,
		</img, <img, </img>,
		</input, <input, </input>,
		</ins, <ins, </ins>,
		</kbd, <kbd, </kbd>,
		</keygen, <keygen, </keygen>,
		</label, <label, </label>,
		</legend, <legend, </legend>,
		</li, <li, </li>,
		</link, <link, </link>,
		</main, <main, </main>,
		</map, <map, </map>,
		</mark, <mark, </mark>,
		</math, <math, </math>,
		</menu, <menu, </menu>,
		</menuitem, <menuitem, </menuitem>,
		</meta, <meta, </meta>,
		</meter, <meter, </meter>,
		</nav, <nav, </nav>,
		</noscript, <noscript, </noscript>,
		</object, <object, </object>,
		</ol, <ol, </ol>,
		</optgroup, <optgroup, </optgroup>,
		</option, <option, </option>,
		</output, <output, </output>,
		</p, <p, </p>,
		</param, <param, </param>,
		</pre, <pre, </pre>,
		</progress, <progress, </progress>,
		</q, <q, </q>,
		</rp, <rp, </rp>,
		</rt, <rt, </rt>,
		</ruby, <ruby, </ruby>,
		</s, <s, </s>,
		</samp, <samp, </samp>,
		</script, <script, </script>,
		</section, <section, </section>,
		</select, <select, </select>,
		</small, <small, </small>,
		</source, <source, </source>,
		</span, <span, </span>,
		</strong, <strong, </strong>,
		</style, <style, </style>,
		</summary, <summary, </summary>,
		</sup, <sup, </sup>,
		</svg, <svg, </svg>,
		</table, <table, </table>,
		</tbody, <tbody, </tbody>,
		</td, <td, </td>,
		</template, <template, </template>,
		</textarea, <textarea, </textarea>,
		</tfoot, <tfoot, </tfoot>,
		</th, <th, </th>,
		</thead, <thead, </thead>,
		</time, <time, </time>,
		</title, <title, </title>,
		</tr, <tr, </tr>,
		</track, <track, </track>,
		</u, <u, </u>,
		</ul, <ul, </ul>,
		</var, <var, </var>,
		</video, <video, </video>,
		</wbr, <wbr, </wbr>,
		/>, <!
	},  
	ndkeywords={
		% General
		=,
		% HTML attributes
		accept=, accept-charset=, accesskey=, action=, align=, alt=, async=, autocomplete=, autofocus=, autoplay=, autosave=, bgcolor=, border=, buffered=, challenge=, charset=, checked=, cite=, class=, code=, codebase=, color=, cols=, colspan=, content=, contenteditable=, contextmenu=, controls=, coords=, data=, datetime=, default=, defer=, dir=, dirname=, disabled=, download=, draggable=, dropzone=, enctype=, for=, form=, formaction=, headers=, height=, hidden=, high=, href=, hreflang=, http-equiv=, icon=, id=, ismap=, itemprop=, keytype=, kind=, label=, lang=, language=, list=, loop=, low=, manifest=, max=, maxlength=, media=, method=, min=, multiple=, name=, novalidate=, open=, optimum=, pattern=, ping=, placeholder=, poster=, preload=, pubdate=, radiogroup=, readonly=, rel=, required=, reversed=, rows=, rowspan=, sandbox=, scope=, scoped=, seamless=, selected=, shape=, size=, sizes=, span=, spellcheck=, src=, srcdoc=, srclang=, start=, step=, style=, summary=, tabindex=, target=, title=, type=, usemap=, value=, width=, wrap=,
		% CSS properties
		accelerator:,azimuth:,background:,background-attachment:,
		background-color:,background-image:,background-position:,
		background-position-x:,background-position-y:,background-repeat:,
		behavior:,border:,border-bottom:,border-bottom-color:,
		border-bottom-style:,border-bottom-width:,border-collapse:,
		border-color:,border-left:,border-left-color:,border-left-style:,
		border-left-width:,border-right:,border-right-color:,
		border-right-style:,border-right-width:,border-spacing:,
		border-style:,border-top:,border-top-color:,border-top-style:,
		border-top-width:,border-width:,bottom:,caption-side:,clear:,
		clip:,color:,content:,counter-increment:,counter-reset:,cue:,
		cue-after:,cue-before:,cursor:,direction:,display:,elevation:,
		empty-cells:,filter:,float:,font:,font-family:,font-size:,
		font-size-adjust:,font-stretch:,font-style:,font-variant:,
		font-weight:,height:,ime-mode:,include-source:,
		layer-background-color:,layer-background-image:,layout-flow:,
		layout-grid:,layout-grid-char:,layout-grid-char-spacing:,
		layout-grid-line:,layout-grid-mode:,layout-grid-type:,left:,
		letter-spacing:,line-break:,line-height:,list-style:,
		list-style-image:,list-style-position:,list-style-type:,margin:,
		margin-bottom:,margin-left:,margin-right:,margin-top:,
		marker-offset:,marks:,max-height:,max-width:,min-height:,
		min-width:,transition-duration:,transition-property:,
		transition-timing-function:,transform:,
		-moz-transform:,-moz-binding:,-moz-border-radius:,
		-moz-border-radius-topleft:,-moz-border-radius-topright:,
		-moz-border-radius-bottomright:,-moz-border-radius-bottomleft:,
		-moz-border-top-colors:,-moz-border-right-colors:,
		-moz-border-bottom-colors:,-moz-border-left-colors:,-moz-opacity:,
		-moz-outline:,-moz-outline-color:,-moz-outline-style:,
		-moz-outline-width:,-moz-user-focus:,-moz-user-input:,
		-moz-user-modify:,-moz-user-select:,orphans:,outline:,
		outline-color:,outline-style:,outline-width:,overflow:,
		overflow-X:,overflow-Y:,padding:,padding-bottom:,padding-left:,
		padding-right:,padding-top:,page:,page-break-after:,
		page-break-before:,page-break-inside:,pause:,pause-after:,
		pause-before:,pitch:,pitch-range:,play-during:,position:,quotes:,
		-replace:,richness:,right:,ruby-align:,ruby-overhang:,
		ruby-position:,-set-link-source:,size:,speak:,speak-header:,
		speak-numeral:,speak-punctuation:,speech-rate:,stress:,
		scrollbar-arrow-color:,scrollbar-base-color:,
		scrollbar-dark-shadow-color:,scrollbar-face-color:,
		scrollbar-highlight-color:,scrollbar-shadow-color:,
		scrollbar-3d-light-color:,scrollbar-track-color:,table-layout:,
		text-align:,text-align-last:,text-decoration:,text-indent:,
		text-justify:,text-overflow:,text-shadow:,text-transform:,
		text-autospace:,text-kashida-space:,text-underline-position:,top:,
		unicode-bidi:,-use-link-source:,vertical-align:,visibility:,
		voice-family:,volume:,white-space:,widows:,width:,word-break:,
		word-spacing:,word-wrap:,writing-mode:,z-index:,zoom:
	},  
	morecomment=[s]{<!--}{-->},
	tag=[s]
}

%\usepackage{color}
%\definecolor{editorGray}{rgb}{0.95, 0.95, 0.95}
%\definecolor{editorOcher}{rgb}{1, 0.5, 0} % #FF7F00 -> rgb(239, 169, 0)
%\definecolor{editorGreen}{rgb}{0, 0.5, 0} % #007C00 -> rgb(0, 124, 0)
%
%\lstset{%
%	% Basic design
%	backgroundcolor=\color{editorGray},
%	basicstyle={\small\ttfamily},   
%	frame=l,
%	% Line numbers
%	xleftmargin={0.75cm},
%	numbers=left,
%	stepnumber=1,
%	firstnumber=1,
%	numberfirstline=true,
%	% Code design   
%	keywordstyle=\color{blue}\bfseries,
%	commentstyle=\color{darkgray}\ttfamily,
%	ndkeywordstyle=\color{editorGreen}\bfseries,
%	stringstyle=\color{editorOcher},
%	% Code
%	language=HTML5,
%	alsodigit={.:;},
%	tabsize=2,
%	showtabs=false,
%	showspaces=false,
%	showstringspaces=false,
%	extendedchars=true,
%	breaklines=true,        
%}

\lstset{language=CSS,
	basicstyle=\ttfamily,
	keywordstyle=\color{blue}\ttfamily,
	stringstyle=\color{red}\ttfamily,
	commentstyle=\color{green}\ttfamily,
	morecomment=[l][\color{magenta}]{\#}
}

\title[TALN : 10- Quelques applications]%
{Traitement automatique du langage naturel\\Chapitre 10 : Quelques applications} 

\changegraphpath{../img/app/}

\begin{document}
	
\begin{frame}
\frametitle{Traitement automatique du langage naturel}
\framesubtitle{Quelques applications : Introduction}

%\begin{exampleblock}{Exemple d'un texte en français}
%	\begin{center}
%		\Large\bfseries
%		L'hiver est un des quatre saisons de l'année. 
%		Le rapport était bien exposé. 
%		La pluie est une des caractéristiques de cette saison.
%		Le sujet était la saison de l'hiver.
%	\end{center}
%\end{exampleblock}
%
%\begin{itemize}
%	\item Est-ce que vous pouvez comprendre de quoi s'agit-il ?
%	\item S'il y a un problème, lequel ?
%	\item Est-ce qu'on peut améliorer ce texte ?
%\end{itemize}

\end{frame}

%\begin{frame}
%\frametitle{Traitement automatique du langage naturel}
%\framesubtitle{Cohérence du discours :  Un peu d'humour}
%
%\begin{center}
%	\vgraphpage{humour1.jpg}
%\end{center}
%
%\end{frame}

\begin{frame}
\frametitle{Traitement automatique du langage naturel}
\framesubtitle{Quelques applications : Plan}

\begin{multicols}{2}
%	\small
\tableofcontents
\end{multicols}
\end{frame}

%===================================================================================
\section{Transformation}
%===================================================================================

\begin{frame}
	\frametitle{Quelques applications}
	\framesubtitle{Transformation}
	\begin{itemize}
		\item 
	\end{itemize}
\end{frame}

\subsection{Traduction automatique}

\begin{frame}
	\frametitle{Quelques applications : Transformation}
	\framesubtitle{Traduction automatique}
	\begin{itemize}
		\item 
	\end{itemize}
\end{frame}

\subsection{Résumé automatique}

\begin{frame}
	\frametitle{Quelques applications : Transformation}
	\framesubtitle{Résumé automatique : Classification d'un résumé automatique}
	\hgraphpage{sum-classif.pdf}
\end{frame}

\begin{frame}
	\frametitle{Quelques applications : Transformation}
	\framesubtitle{Résumé automatique : Approches du résumé automatique}
	\begin{columns}
		\begin{column}{0.32\textwidth}
			\begin{block}{\scriptsize\bfseries\cite{12-nenkova-mckeown}}
				\begin{itemize}
					\item représentation du sujet 
					:
					mots du sujet,
					fréquences, 
					analyse sémantique latente, 
					modèles de sujets bayésiens,
					clustering
					%			\begin{itemize}
					%				\item mots du sujet 
					%				\item fréquences
					%				\item analyse sémantique latente
					%				\item modèles de sujets bayésiens
					%			\end{itemize}
					\item représentation des indicateurs : 
					par graphes, 
					apprentissage automatique
				\end{itemize}
			\end{block}
		\end{column}
		\begin{column}{0.3\textwidth}
			\begin{block}{\scriptsize\bfseries\cite{12-lloret-palomar}}
				\begin{itemize}
					\item statistique 
					\item par graphes
					\item basée discours
					\item par apprentissage automatique
				\end{itemize}
			\end{block}
		\end{column}
		\begin{column}{0.28\textwidth}
			\begin{block}{\scriptsize\bfseries\cite{19-aries-al}}
				\begin{itemize}
					\item statistique 
					\item par graphes
					\item linguistique 
					\item par apprentissage automatique
				\end{itemize}
			\end{block}
		\end{column}
	\end{columns}
\end{frame}

\begin{frame}
	\frametitle{Quelques applications : Transformation}
	\framesubtitle{Résumé automatique : Approche statistique}
	
	\begin{itemize}
		\item \optword{Fréquence des mots} 
		
		\hspace{.5cm}Ex. $Score_\text{TF-IDF}(s_i) = \sqrt{\sum\limits_{w_{ik} \in s_i} (\text{TF-IDF}(w_{ik}))^2}$
		
		\item \optword{Position des phrases (ou des mots)}
		
		\hspace{.5cm}Ex. $ Score_\text{pos}(s_i) = \max (\frac{1}{i}, \frac{1}{|D| - i + 1}) $
		
		\item \optword{Taille (longueur) des phrases}
		
		\hspace{.5cm}Ex. $ Score_\text{taille}(s_i) = \left\lbrace 
		\begin{array}{lll}
		0 & si & (L_i \geq L_{min}) \\
		\frac{L_i - L_{min}}{L_{min}} & sinon & \\
		\end{array}
		\right. $
		
		\item \optword{Mots du titre et des sous-titres}
		
		\hspace{.5cm}Ex. $ Score_{titre}(s_i) = \frac{\sum_{e \in T \bigcap s_i}{\frac{tf(e)}{tf(e)+1}}}
		{\sum_{e \in T}{\frac{tf(e)}{tf(e)+1}}} $
		
		\item \optword{Centroid}, \optword{Frequent itemsets}, \optword{Analyse sémantique latente}, ...
	\end{itemize}
\end{frame}

\begin{frame}
	\frametitle{Quelques applications : Transformation}
	\framesubtitle{Résumé automatique : Approche statistique (Exemple : \cite{13-aries-al})}
	
	\hgraphpage{tcc-arch.pdf}
	
\end{frame}

\begin{frame}
	\frametitle{Quelques applications : Transformation}
	\framesubtitle{Résumé automatique : Approche statistique (Exemple: \cite{13-aries-al} - suite)}
	
	\begin{itemize}
		\item Un texte peut contenir plusieurs sujets, et une phrase peut discuter plusieurs sujets
		\begin{itemize}
			\item Regroupement de phrases avec similarité et seuil de regroupement ($Th$)
		\end{itemize}
		\item Une phrase est importante si elle peut représenter le maximum des sujets(cluters)
		\[ Score(s_i , c_j , f_k ) = 1 + \sum_{\phi \in s_i} {P(f_k=\phi | s_i \in c_j)} \]
		\[ Score(s_i , \bigcap_{j} c_j , F) =  %\propto 
		\prod_{j} \prod_{k} Score(s_i , c_j , f_k ) \]
		$ s $ : phrase, $ c $ : cluster, $ f $ : caractéristique, $ F $ : ensemble de caractéristiques, $ \phi $: observation de $ f $.
		\item $f$ : TF (Uni, Bi); Pos (intervalle de 10); Len (réel, pré-traité)
	\end{itemize}
	
\end{frame}

\begin{frame}
	\frametitle{Quelques applications : Transformation}
	\framesubtitle{Résumé automatique : Approche statistique (Exemple: \cite{15-oufaida-al})}
	
	\begin{itemize}
		\item Représentation des mots : embeddings pré-entrainés (Polyglot)
		\item Clustering pour extraire les sous-sujets du texte
		\begin{itemize}
			\item On peut trouver le mot $w_j \in S_2$ le plus similaire à un mot $w_i$
			
			$Sim\_Match(w_i | S_2) = \max sim(Rep(w_i), Rep(w_j))$
			
			\item Lien entre deux phrases $S_2$ et $S_2$
			
			$Sim(S_1, S_2) = \frac{\sum_{w_i \in S_1} Sim\_Match(w_i | S_2) + \sum_{w_j \in S_2} Sim\_Match(w_j | S_1)}{|S_1| + |S_2|}$
		\end{itemize}
	
		\item Score des terms en utilisant mRMR
	\end{itemize}
	
\end{frame}

\begin{frame}
	\frametitle{Quelques applications : Transformation}
	\framesubtitle{Résumé automatique : Approche par graphes}
	
	\begin{itemize}
		\item \optword{Propriétés du graphe}
		\begin{itemize}
			\item Bushy paths 
			
			\hspace{.5cm}$Score_{\#arcs}(s_i) = |\{ s_j : a(s_i, s_j) \in A / s_j \in S, s_i \neq s_j \}|$
			
			\item Aggregate Similarity
			
			\hspace{.5cm}$Score_{aggregate}(s_i) = \sum\limits_{(s_i, s_j) \in E} sim(s_i, s_j)$
		\end{itemize}
		\item \optword{Méthodes itératives}
		\begin{itemize}
			\item Mettre à jour les scores des nœuds à base des voisins 
			\item L'arrêt : un état d'équilibre (on peut plus mettre à jour)
			\item Ex. TextRank \cite{04-mihalcea-tarau}
			
			$WS(V_i) = ( 1 - d) + d * \sum\limits_{V_j \in In(V_i)} \frac{w_{ji}}{\sum\limits_{V_k \in Out(V_j)} w_{jk}} WS(V_j)$
			
			$w_{ij} = \frac{|\{w_k \text{ / } w_k \in S_i \text{ and } w_k \in S_j\}|}{\log(|S_i|) + \log(|S_j|)}$
			
			$ d $ : damping factor (en général, environ $ 0.85 $)
		\end{itemize}
	\end{itemize}

\end{frame}

\begin{frame}
	\frametitle{Quelques applications : Transformation}
	\framesubtitle{Résumé automatique : Approche par graphes (Exemple: \cite{21-aries-al})}
	
	\begin{center}
		\vgraphpage{gc-archi.pdf}
	\end{center}
	
\end{frame}

\begin{frame}
	\frametitle{Quelques applications : Transformation}
	\framesubtitle{Résumé automatique : Approche par graphes (Exemple : \cite{21-aries-al} - suite)}
	
	\vspace{-6pt}
	\begin{itemize}
		\item Simplification du graphe
		
		\hspace{.5cm}$noeud\_faible(v_i) = ( \sum_{(v_i, v_j) \in E} w_{ij} < \frac{1}{MImpN(v_i)} )$ 
		
		\hspace{.5cm}$arc\_faible(v_i, v_j) = ( w_{ij} < \frac{Threshold}{MImpN(v_i)})$
		
		\item Score statistique des phrases
		
		\hspace{.5cm}$ Score(s_i/ sim) = sim(s_i, C\backslash s_i) $
		$Score(s_i/ tfisf) = \sqrt{\sum\limits_{w_{ik} \in s_i} (tfisf(w_{ik}))^2}$
		
		\hspace{.5cm}$Score(s_i/ size) = \frac{1}{|s_i|}$
		$Score(s_i/ pos) = \max (\frac{1}{i}, \frac{1}{|D| - i + 1})$
		
		\hspace{.5cm}$SSF(s_i/ F) = \prod_{f_i \in F} score(s_i/f_i)$
		
		\item Score à base de graphe 
		
		\hspace{.5cm}Ex. $GC1(s_i) = SSF(s_i) + \sum\limits_{(s_i, s_j) \in E} sim(s_i, s_j) * SSF(s_j)$
		
		\item Extraction
		
		\hspace{.5cm}Ex. $ suiv_{e4}  =  \arg\min\limits_i (iord\ ssfgc(s_i) + ord\ simil(dernier{e4}, i))$ 
		
		\hspace{3cm}$ \text{ où } (dernier_{e4}, s_i) \in E $
	\end{itemize}
	
\end{frame}

\begin{frame}
	\frametitle{Quelques applications : Transformation}
	\framesubtitle{Résumé automatique : Approche linguistique}
	
	\begin{itemize}
		\item \optword{Mots de sujet} : une liste des mots pertinents au sujet, comme ``significant", ``impossible", etc.
		\begin{itemize}
			\item Ex. \cite{69-edmundson} : Bonus (mots positivement pertinents), Stigma (mots négativement pertinents)
			
			$Score_{cue}(s_i) = \sum_{w \in s_i}{cue(w)}
			\text{ où }
			cue(w) = \left\lbrace 
			\begin{array}{ll}
			b > 0 & \text{si } (w \in Bonus) \\
			\delta < 0 & \text{si } (w \in Stigma) \\
			0 & sinon 
			\end{array} 
			\right. $
		\end{itemize}
		\item \optword{Indicateurs} : Des structures qui impliquent que la phrase les contenant a une chose importante à propos du sujet
		\begin{itemize}
			\item Ex. \expword{the principal aim of this paper is to investigate ...}
		\end{itemize}
		\item \optword{Co-référence} : utilisation des anaphores ou des représentations sémantiques (Ex. Wordnet)
		\item \optword{Structure rhétorique} : utilisation de la structure rhétorique pour noter les phrases ou des syntagmes
	\end{itemize}
	
\end{frame}

\begin{frame}
	\frametitle{Quelques applications : Transformation}
	\framesubtitle{Résumé automatique : Approche linguistique (Exemple : \cite{81-paice})}
	
	\begin{itemize}
		\item Utilisation des patrons (templates) préparés manuellement
		\item\ [x] : x mots peuvent être entre ce mot et le mot précédent
		\item +y : le score est augment par y
		\item ? : ce mot est optionnel
	\end{itemize}

	\begin{figure}[!ht]
		\begin{center}
			\hgraphpage[.7\textwidth]{paice-template.pdf}
			\caption{Un exemple d'un patron simplifié \cite{81-paice}.}
			\label{fig:paice-template}
		\end{center}
	\end{figure}
	
\end{frame}

\begin{frame}
	\frametitle{Quelques applications : Transformation}
	\framesubtitle{Résumé automatique : Approche linguistique (Exemple : \cite{})}
	
	\begin{itemize}
		\item 
	\end{itemize}
	
\end{frame}

\begin{frame}
	\frametitle{Quelques applications : Transformation}
	\framesubtitle{Résumé automatique : Approche par apprentissage automatique}
	
	\begin{itemize}
		\item \optword{Par caractéristiques}
		\begin{itemize}
			\item Réglage : régler des hyper-paramètres comme les poids des caractéristiques pour le score
			\item Classement : décider si une unité (phrase) appartient au résumé ou non
		\end{itemize}
		\item \optword{Bayesian topic models}
		\begin{itemize}
			\item 
		\end{itemize}
		\item \optword{Deep learning}
		\begin{itemize}
			\item Réglage ou Classement
			\item Génération des mots (une forme du classement)
		\end{itemize}
		\item \optword{Reinforcement learning}
		\begin{itemize}
			\item 
		\end{itemize}
	\end{itemize}
	
\end{frame}

%===================================================================================
\section{Interaction}
%===================================================================================

\begin{frame}
	\frametitle{Quelques applications}
	\framesubtitle{Interaction}
	\begin{itemize}
		\item 
	\end{itemize}
\end{frame}

\subsection{Questions-Réponses}

\begin{frame}
	\frametitle{Quelques applications : Interaction}
	\framesubtitle{Questions-Réponses}
	\begin{itemize}
		\item 
	\end{itemize}
\end{frame}

\subsection{Systèmes de dialogue et chatbots}

\begin{frame}
	\frametitle{Quelques applications : Interaction}
	\framesubtitle{Systèmes de dialogue et chatbots}
	\begin{itemize}
		\item 
	\end{itemize}
\end{frame}


%===================================================================================
\section{Classification}
%===================================================================================

\begin{frame}
	\frametitle{Quelques applications}
	\framesubtitle{Classification}
	\begin{itemize}
		\item Filtrage de spams
		\item Identification des langues
		\item Lisibilité 
		\item Analyse des sentiments
		\item Détection de l'humeur
	\end{itemize}
\end{frame}

\subsection{Analyse des sentiments}

\begin{frame}
	\frametitle{Quelques applications : Classification}
	\framesubtitle{Analyse des sentiments}
	\begin{itemize}
		\item 
	\end{itemize}
\end{frame}

\subsection{Lisibilité}

\begin{frame}
	\frametitle{Quelques applications : Classification}
	\framesubtitle{Lisibilité}
	\begin{itemize}
		\item 
	\end{itemize}
\end{frame}


%===================================================================================
\section{Voix}
%===================================================================================

\begin{frame}
	\frametitle{Quelques applications}
	\framesubtitle{Voix}
	\begin{itemize}
		\item 
	\end{itemize}
\end{frame}

\subsection{Reconnaissance de la voix}

\begin{frame}
	\frametitle{Quelques applications : Voix}
	\framesubtitle{Reconnaissance de la voix}
	\begin{itemize}
		\item 
	\end{itemize}
\end{frame}

\subsection{Synthèse de la voix}

\begin{frame}
	\frametitle{Quelques applications : Voix}
	\framesubtitle{Synthèse de la voix}
	\begin{itemize}
		\item 
	\end{itemize}
\end{frame}


\insertbibliography{TALN10}{*}

\end{document}

