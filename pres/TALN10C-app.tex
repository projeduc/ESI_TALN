% !TEX TS-program = pdflatex
% !TeX program = pdflatex
% !TEX encoding = UTF-8
% !TEX spellcheck = fr

\documentclass[xcolor=table]{beamer}


%\usepackage{fullpage}
%\usepackage[left=2.8cm,right=2.2cm,top=2 cm,bottom=2 cm]{geometry}
\setbeamersize{text margin left=10pt,text margin right=10pt}
\usepackage{amsmath,amssymb} 
\usepackage[T1]{fontenc}
\usepackage{textcomp}
\usepackage[utf8]{inputenc}
\usepackage[french]{babel}
\usepackage{arabtex}
\usepackage{txfonts}
\usepackage[]{graphicx}
\usepackage{multirow}
\usepackage{hyperref}
\usepackage{colortbl}
\usepackage{listingsutf8}
\usepackage{wrapfig}
\usepackage{multicol}
\usepackage[export]{adjustbox} %for images in table, also for frame
\usepackage[many]{tcolorbox}

\hypersetup{
	colorlinks,
	urlcolor = blue
}

%\renewcommand{\baselinestretch}{1.5}

\def\supit#1{\raisebox{0.8ex}{\small\it #1}\hspace{0.05em}}

\AtBeginSection{%
	\begin{frame}
		\sectionpage
	\end{frame}
}

\newcommand{\rottext}[2]{%
	\rotatebox{90}{%
	\begin{minipage}{#1}%
		\raggedleft#2%
	\end{minipage}%
	}%
}

\usepackage{longtable}
\usepackage{tabu}


\institute{ %
École  nationale Supérieure d'Informatique (ESI, ex. INI), Algérie
}
\author[ \textbf{\footnotesize\insertframenumber/\inserttotalframenumber} \hspace*{1.5cm} ESI - ARIES Abdelkrime (2020/2021)] %
{ARIES Abdelkrime}
%\titlegraphic{\includegraphics[height=1cm]{../img/esi-logo.png}%\hspace*{4.75cm}~


\date{Année universitaire : 2020/2021} %\today

\usetheme{Warsaw} % Antibes Boadilla Warsaw

\beamertemplatenavigationsymbolsempty

%\setbeamertemplate{headline}{}

\definecolor{lightblue}{HTML}{D0D2FF}
\definecolor{lightyellow}{HTML}{FFFFAA}
\definecolor{darkblue}{HTML}{0000BB}
\definecolor{olivegreen}{HTML}{006600}
\definecolor{violet}{HTML}{6600CC}

\newcommand{\keyword}[1]{\textcolor{red}{\bfseries\itshape #1}}
\newcommand{\expword}[1]{\textcolor{olivegreen}{#1}}
\newcommand{\optword}[1]{\textcolor{violet}{\bfseries #1}}

\makeatletter
\newcommand\mysphere{%
	\parbox[t]{10pt}{\raisebox{0.2pt}{\beamer@usesphere{item projected}{bigsphere}}}}
\makeatother

%\let\oldtabular\tabular
%\let\endoldtabular\endtabular
%\renewenvironment{tabular}{\rowcolors{2}{white}{lightblue}\oldtabular\rowcolor{blue}}{\endoldtabular}


\NoAutoSpacing %french autospacing after ":"

\def\graphpath{}

\newcommand{\changegraphpath}[1]{\def\graphpath{#1}}


\newcommand{\vgraphpage}[2][.84\textheight]{%
%	\begin{center}%
		\includegraphics[height=#1]{\graphpath #2}%
%	\end{center}%
}

\newcommand{\hgraphpage}[2][\textwidth]{%
%	\begin{center}%
		\includegraphics[width=#1]{\graphpath #2}%
%	\end{center}%
}

\newcommand{\graphpage}[2][]{%
	\includegraphics[#1]{\graphpath #2}%
}

\bibliographystyle{acm}

\newcommand{\insertbibliography}[2]{
	\appendix
	\section*{Bibliographie}
	\nocite{#2}
%	\makeatletter % to change template
%	\setbeamertemplate{headline}[default] % not mandatory, but I though it was better to set it blank
%	\def\beamer@entrycode{\vspace*{-\headheight}} % here is the part we are interested in :)
%	\makeatother
	\begin{multicols*}{2}[\frametitle{\insertsection} \usebeamertemplate*{frametitle}]%\usebeamertemplate*{frametitle}\frametitle{Références}
		\tiny
		\bibliography{#1}
	\end{multicols*}
}

\definecolor{my-grey}{RGB}{233, 233, 233}

\newcommand{\insertlicence}{
	\begin{frame}[plain]
	\frametitle{Licence : CC-BY 4.0}
%	\framesubtitle{Licence: CC-BY-NC 4.0}

	\begin{tcolorbox}[colback=cyan,
		colframe=cyan,  
		arc=0pt,outer arc=0pt,
		valign=top, 
		halign=center,
		width=\textwidth]
		
		\includegraphics[width=.5cm]{../img/licence/cc_icon_white_x2.png}
		\includegraphics[width=.5cm]{../img/licence/attribution_icon_white_x2.png}
		
		\color{white}
		\bfseries Attribution 4.0 International (CC BY 4.0) \\
		\tiny \url{https://creativecommons.org/licenses/by/4.0/deed.fr}
		
	\end{tcolorbox}\vspace{-.5cm}
	\begin{tcolorbox}[colback=my-grey,
		colframe=my-grey,  
		center, arc=0pt,outer arc=0pt,
		valign=top, 
		halign=left,
		width=\textwidth]
		
		\tiny
		
		\begin{center}
			\bfseries\Large
			Vous êtes autorisé à :
		\end{center}
		
		\begin{minipage}{0.83\textwidth}
			\begin{itemize}
				\item[] \textbf{Partager} — copier, distribuer et communiquer le matériel par tous moyens et sous tous formats
				\item[] \textbf{Adapter} — remixer, transformer et créer à partir du matériel
				pour toute utilisation, y compris commerciale.
			\end{itemize}
		\end{minipage}
		\begin{minipage}{0.15\textwidth}
			\includegraphics[width=\textwidth]{../img/licence/FreeCulturalWorks_seal_x2.jpg}
		\end{minipage}
	
		
		\begin{center}
			\bfseries\Large
			Selon les conditions suivantes :
		\end{center}
		
		\begin{itemize}
			\item[] \textbf{Attribution} — Vous devez créditer l'Œuvre, intégrer un lien vers la licence et indiquer si des modifications ont été effectuées à l'Oeuvre. Vous devez indiquer ces informations par tous les moyens raisonnables, sans toutefois suggérer que l'Offrant vous soutient ou soutient la façon dont vous avez utilisé son Oeuvre. 
			\item[] \textbf{Pas de restrictions complémentaires} — Vous n'êtes pas autorisé à appliquer des conditions légales ou des mesures techniques qui restreindraient légalement autrui à utiliser l'Oeuvre dans les conditions décrites par la licence.
		\end{itemize}
		
	\end{tcolorbox}
	
%	\begin{center}
%		\bfseries Attribution 4.0 International (CC BY 4.0)
%		\url{https://creativecommons.org/licenses/by/4.0/deed.fr}
%	\end{center}

%	\tiny
%
%	Vous êtes autorisé à : 
%	\begin{itemize}
%		\item \textbf{Partager} — copier, distribuer et communiquer le matériel par tous moyens et sous tous formats
%		\item \textbf{Adapter} — remixer, transformer et créer à partir du matériel
%	\end{itemize}
%	
%	Selon les conditions suivantes : 
%	\begin{itemize}
%		\item \textbf{Attribution} — Vous devez créditer l'Œuvre, intégrer un lien vers la licence et indiquer si des modifications ont été effectuées à l'Oeuvre. Vous devez indiquer ces informations par tous les moyens raisonnables, sans toutefois suggérer que l'Offrant vous soutient ou soutient la façon dont vous avez utilisé son Oeuvre.
%		\item \textbf{Pas d'Utilisation Commerciale} — Vous n'êtes pas autorisé à faire un usage commercial de cette Oeuvre, tout ou partie du matériel la composant. 
%		\item \textbf{Pas de restrictions complémentaires} — Vous n'êtes pas autorisé à appliquer des conditions légales ou des mesures techniques qui restreindraient légalement autrui à utiliser l'Oeuvre dans les conditions décrites par la licence.
%	\end{itemize}

	\end{frame}
}

\settowidth{\leftmargini}{\usebeamertemplate{itemize item}}
\addtolength{\leftmargini}{\labelsep}

\AtBeginDocument{
	\newcolumntype{L}[2]{>{\vbox to #2\bgroup\vfill\flushleft}p{#1}<{\egroup}} 
	
	\begin{frame}[plain]
		\maketitle
	\end{frame}

	\insertlicence
}


% needs etoolbox; to break links after -
\appto\UrlBreaks{\do\-}


\lstdefinelanguage{CSS}{
	keywords={accelerator,azimuth,background,background-attachment,
		background-color,background-image,background-position,
		background-position-x,background-position-y,background-repeat,
		behavior,border,border-bottom,border-bottom-color,
		border-bottom-style,border-bottom-width,border-collapse,
		border-color,border-left,border-left-color,border-left-style,
		border-left-width,border-right,border-right-color,
		border-right-style,border-right-width,border-spacing,
		border-style,border-top,border-top-color,border-top-style,
		border-top-width,border-width,bottom,caption-side,clear,
		clip,color,content,counter-increment,counter-reset,cue,
		cue-after,cue-before,cursor,direction,display,elevation,
		empty-cells,filter,float,font,font-family,font-size,
		font-size-adjust,font-stretch,font-style,font-variant,
		font-weight,height,ime-mode,include-source,
		layer-background-color,layer-background-image,layout-flow,
		layout-grid,layout-grid-char,layout-grid-char-spacing,
		layout-grid-line,layout-grid-mode,layout-grid-type,left,
		letter-spacing,line-break,line-height,list-style,
		list-style-image,list-style-position,list-style-type,margin,
		margin-bottom,margin-left,margin-right,margin-top,
		marker-offset,marks,max-height,max-width,min-height,
		min-width,-moz-binding,-moz-border-radius,
		-moz-border-radius-topleft,-moz-border-radius-topright,
		-moz-border-radius-bottomright,-moz-border-radius-bottomleft,
		-moz-border-top-colors,-moz-border-right-colors,
		-moz-border-bottom-colors,-moz-border-left-colors,-moz-opacity,
		-moz-outline,-moz-outline-color,-moz-outline-style,
		-moz-outline-width,-moz-user-focus,-moz-user-input,
		-moz-user-modify,-moz-user-select,orphans,outline,
		outline-color,outline-style,outline-width,overflow,
		overflow-X,overflow-Y,padding,padding-bottom,padding-left,
		padding-right,padding-top,page,page-break-after,
		page-break-before,page-break-inside,pause,pause-after,
		pause-before,pitch,pitch-range,play-during,position,quotes,
		-replace,richness,right,ruby-align,ruby-overhang,
		ruby-position,-set-link-source,size,speak,speak-header,
		speak-numeral,speak-punctuation,speech-rate,stress,
		scrollbar-arrow-color,scrollbar-base-color,
		scrollbar-dark-shadow-color,scrollbar-face-color,
		scrollbar-highlight-color,scrollbar-shadow-color,
		scrollbar-3d-light-color,scrollbar-track-color,table-layout,
		text-align,text-align-last,text-decoration,text-indent,
		text-justify,text-overflow,text-shadow,text-transform,
		text-autospace,text-kashida-space,text-underline-position,top,
		unicode-bidi,-use-link-source,vertical-align,visibility,
		voice-family,volume,white-space,widows,width,word-break,
		word-spacing,word-wrap,writing-mode,z-index,zoom},  
	sensitive=true,
	morecomment=[l]{//},
	morecomment=[s]{/*}{*/},
	morestring=[b]',
	morestring=[b]",
	alsoletter={:},
	alsodigit={-}
}
\lstdefinelanguage{HTML5}{
	language=html,
	sensitive=true, 
	alsoletter={<>=-},
	otherkeywords={
		% HTML tags
		<, </, >,
		</a, <a, </a>,
		</abbr, <abbr, </abbr>,
		</address, <address, </address>,
		</area, <area, </area>,
		</area, <area, </area>,
		</article, <article, </article>,
		</aside, <aside, </aside>,
		</audio, <audio, </audio>,
		</audio, <audio, </audio>,
		</b, <b, </b>,
		</base, <base, </base>,
		</bdi, <bdi, </bdi>,
		</bdo, <bdo, </bdo>,
		</blockquote, <blockquote, </blockquote>,
		</body, <body, </body>,
		</br, <br, </br>,
		</button, <button, </button>,
		</canvas, <canvas, </canvas>,
		</caption, <caption, </caption>,
		</cite, <cite, </cite>,
		</code, <code, </code>,
		</col, <col, </col>,
		</colgroup, <colgroup, </colgroup>,
		</data, <data, </data>,
		</datalist, <datalist, </datalist>,
		</dd, <dd, </dd>,
		</del, <del, </del>,
		</details, <details, </details>,
		</dfn, <dfn, </dfn>,
		</div, <div, </div>,
		</dl, <dl, </dl>,
		</dt, <dt, </dt>,
		</em, <em, </em>,
		</embed, <embed, </embed>,
		</fieldset, <fieldset, </fieldset>,
		</figcaption, <figcaption, </figcaption>,
		</figure, <figure, </figure>,
		</footer, <footer, </footer>,
		</form, <form, </form>,
		</h1, <h1, </h1>,
		</h2, <h2, </h2>,
		</h3, <h3, </h3>,
		</h4, <h4, </h4>,
		</h5, <h5, </h5>,
		</h6, <h6, </h6>,
		</head, <head, </head>,
		</header, <header, </header>,
		</hr, <hr, </hr>,
		</html, <html, </html>,
		</i, <i, </i>,
		</iframe, <iframe, </iframe>,
		</img, <img, </img>,
		</input, <input, </input>,
		</ins, <ins, </ins>,
		</kbd, <kbd, </kbd>,
		</keygen, <keygen, </keygen>,
		</label, <label, </label>,
		</legend, <legend, </legend>,
		</li, <li, </li>,
		</link, <link, </link>,
		</main, <main, </main>,
		</map, <map, </map>,
		</mark, <mark, </mark>,
		</math, <math, </math>,
		</menu, <menu, </menu>,
		</menuitem, <menuitem, </menuitem>,
		</meta, <meta, </meta>,
		</meter, <meter, </meter>,
		</nav, <nav, </nav>,
		</noscript, <noscript, </noscript>,
		</object, <object, </object>,
		</ol, <ol, </ol>,
		</optgroup, <optgroup, </optgroup>,
		</option, <option, </option>,
		</output, <output, </output>,
		</p, <p, </p>,
		</param, <param, </param>,
		</pre, <pre, </pre>,
		</progress, <progress, </progress>,
		</q, <q, </q>,
		</rp, <rp, </rp>,
		</rt, <rt, </rt>,
		</ruby, <ruby, </ruby>,
		</s, <s, </s>,
		</samp, <samp, </samp>,
		</script, <script, </script>,
		</section, <section, </section>,
		</select, <select, </select>,
		</small, <small, </small>,
		</source, <source, </source>,
		</span, <span, </span>,
		</strong, <strong, </strong>,
		</style, <style, </style>,
		</summary, <summary, </summary>,
		</sup, <sup, </sup>,
		</svg, <svg, </svg>,
		</table, <table, </table>,
		</tbody, <tbody, </tbody>,
		</td, <td, </td>,
		</template, <template, </template>,
		</textarea, <textarea, </textarea>,
		</tfoot, <tfoot, </tfoot>,
		</th, <th, </th>,
		</thead, <thead, </thead>,
		</time, <time, </time>,
		</title, <title, </title>,
		</tr, <tr, </tr>,
		</track, <track, </track>,
		</u, <u, </u>,
		</ul, <ul, </ul>,
		</var, <var, </var>,
		</video, <video, </video>,
		</wbr, <wbr, </wbr>,
		/>, <!
	},  
	ndkeywords={
		% General
		=,
		% HTML attributes
		accept=, accept-charset=, accesskey=, action=, align=, alt=, async=, autocomplete=, autofocus=, autoplay=, autosave=, bgcolor=, border=, buffered=, challenge=, charset=, checked=, cite=, class=, code=, codebase=, color=, cols=, colspan=, content=, contenteditable=, contextmenu=, controls=, coords=, data=, datetime=, default=, defer=, dir=, dirname=, disabled=, download=, draggable=, dropzone=, enctype=, for=, form=, formaction=, headers=, height=, hidden=, high=, href=, hreflang=, http-equiv=, icon=, id=, ismap=, itemprop=, keytype=, kind=, label=, lang=, language=, list=, loop=, low=, manifest=, max=, maxlength=, media=, method=, min=, multiple=, name=, novalidate=, open=, optimum=, pattern=, ping=, placeholder=, poster=, preload=, pubdate=, radiogroup=, readonly=, rel=, required=, reversed=, rows=, rowspan=, sandbox=, scope=, scoped=, seamless=, selected=, shape=, size=, sizes=, span=, spellcheck=, src=, srcdoc=, srclang=, start=, step=, style=, summary=, tabindex=, target=, title=, type=, usemap=, value=, width=, wrap=,
		% CSS properties
		accelerator:,azimuth:,background:,background-attachment:,
		background-color:,background-image:,background-position:,
		background-position-x:,background-position-y:,background-repeat:,
		behavior:,border:,border-bottom:,border-bottom-color:,
		border-bottom-style:,border-bottom-width:,border-collapse:,
		border-color:,border-left:,border-left-color:,border-left-style:,
		border-left-width:,border-right:,border-right-color:,
		border-right-style:,border-right-width:,border-spacing:,
		border-style:,border-top:,border-top-color:,border-top-style:,
		border-top-width:,border-width:,bottom:,caption-side:,clear:,
		clip:,color:,content:,counter-increment:,counter-reset:,cue:,
		cue-after:,cue-before:,cursor:,direction:,display:,elevation:,
		empty-cells:,filter:,float:,font:,font-family:,font-size:,
		font-size-adjust:,font-stretch:,font-style:,font-variant:,
		font-weight:,height:,ime-mode:,include-source:,
		layer-background-color:,layer-background-image:,layout-flow:,
		layout-grid:,layout-grid-char:,layout-grid-char-spacing:,
		layout-grid-line:,layout-grid-mode:,layout-grid-type:,left:,
		letter-spacing:,line-break:,line-height:,list-style:,
		list-style-image:,list-style-position:,list-style-type:,margin:,
		margin-bottom:,margin-left:,margin-right:,margin-top:,
		marker-offset:,marks:,max-height:,max-width:,min-height:,
		min-width:,transition-duration:,transition-property:,
		transition-timing-function:,transform:,
		-moz-transform:,-moz-binding:,-moz-border-radius:,
		-moz-border-radius-topleft:,-moz-border-radius-topright:,
		-moz-border-radius-bottomright:,-moz-border-radius-bottomleft:,
		-moz-border-top-colors:,-moz-border-right-colors:,
		-moz-border-bottom-colors:,-moz-border-left-colors:,-moz-opacity:,
		-moz-outline:,-moz-outline-color:,-moz-outline-style:,
		-moz-outline-width:,-moz-user-focus:,-moz-user-input:,
		-moz-user-modify:,-moz-user-select:,orphans:,outline:,
		outline-color:,outline-style:,outline-width:,overflow:,
		overflow-X:,overflow-Y:,padding:,padding-bottom:,padding-left:,
		padding-right:,padding-top:,page:,page-break-after:,
		page-break-before:,page-break-inside:,pause:,pause-after:,
		pause-before:,pitch:,pitch-range:,play-during:,position:,quotes:,
		-replace:,richness:,right:,ruby-align:,ruby-overhang:,
		ruby-position:,-set-link-source:,size:,speak:,speak-header:,
		speak-numeral:,speak-punctuation:,speech-rate:,stress:,
		scrollbar-arrow-color:,scrollbar-base-color:,
		scrollbar-dark-shadow-color:,scrollbar-face-color:,
		scrollbar-highlight-color:,scrollbar-shadow-color:,
		scrollbar-3d-light-color:,scrollbar-track-color:,table-layout:,
		text-align:,text-align-last:,text-decoration:,text-indent:,
		text-justify:,text-overflow:,text-shadow:,text-transform:,
		text-autospace:,text-kashida-space:,text-underline-position:,top:,
		unicode-bidi:,-use-link-source:,vertical-align:,visibility:,
		voice-family:,volume:,white-space:,widows:,width:,word-break:,
		word-spacing:,word-wrap:,writing-mode:,z-index:,zoom:
	},  
	morecomment=[s]{<!--}{-->},
	tag=[s]
}

%\usepackage{color}
%\definecolor{editorGray}{rgb}{0.95, 0.95, 0.95}
%\definecolor{editorOcher}{rgb}{1, 0.5, 0} % #FF7F00 -> rgb(239, 169, 0)
%\definecolor{editorGreen}{rgb}{0, 0.5, 0} % #007C00 -> rgb(0, 124, 0)
%
%\lstset{%
%	% Basic design
%	backgroundcolor=\color{editorGray},
%	basicstyle={\small\ttfamily},   
%	frame=l,
%	% Line numbers
%	xleftmargin={0.75cm},
%	numbers=left,
%	stepnumber=1,
%	firstnumber=1,
%	numberfirstline=true,
%	% Code design   
%	keywordstyle=\color{blue}\bfseries,
%	commentstyle=\color{darkgray}\ttfamily,
%	ndkeywordstyle=\color{editorGreen}\bfseries,
%	stringstyle=\color{editorOcher},
%	% Code
%	language=HTML5,
%	alsodigit={.:;},
%	tabsize=2,
%	showtabs=false,
%	showspaces=false,
%	showstringspaces=false,
%	extendedchars=true,
%	breaklines=true,        
%}

\lstset{language=CSS,
	basicstyle=\ttfamily,
	keywordstyle=\color{blue}\ttfamily,
	stringstyle=\color{red}\ttfamily,
	commentstyle=\color{green}\ttfamily,
	morecomment=[l][\color{magenta}]{\#}
}

\title[TALN : 10- Quelques applications]%
{Traitement automatique du langage naturel\\Chapitre 10 : Quelques applications} 

\changegraphpath{../img/app/}

\begin{document}
	
\begin{frame}
\frametitle{Traitement automatique du langage naturel}
\framesubtitle{Quelques applications : Introduction}

	\begin{itemize}
		\item On veut présenter quelques applications utilisées dans notre vie
		\item Selon l'entrée/sortie : on peut traiter des paroles ou du texte
		\item Selon l'interactivité : le système interagit avec l'utilisateur ou non
		\item Selon la sortie : un ensemble de classes ou un autre texte généré
		\item Notre structure 
		\begin{itemize}
			\item Transformation
			\item Interaction
			\item Classification
			\item Parole
		\end{itemize}
	\end{itemize}

\end{frame}

%\begin{frame}
%\frametitle{Traitement automatique du langage naturel}
%\framesubtitle{Cohérence du discours :  Un peu d'humour}
%
%\begin{center}
%	\vgraphpage{humour1.jpg}
%\end{center}
%
%\end{frame}

\begin{frame}
\frametitle{Traitement automatique du langage naturel}
\framesubtitle{Quelques applications : Plan}

\begin{multicols}{2}
%	\small
\tableofcontents
\end{multicols}
\end{frame}

%===================================================================================
\section{Transformation}
%===================================================================================

\begin{frame}
	\frametitle{Quelques applications}
	\framesubtitle{Transformation}
	\begin{itemize}
		\item \textbf{Entrée} : un texte 
		\item \textbf{Sortie} : un texte généré à partir de l'entrée
		\item \textbf{Applications} 
		\begin{itemize}
			\item Traduction automatique
			\item Résumé automatique
		\end{itemize}
	\end{itemize}
\end{frame}

\subsection{Traduction automatique}

\begin{frame}
	\frametitle{Quelques applications : Transformation}
	\framesubtitle{Traduction automatique : Approches}
%	\begin{itemize}
%		\item \optword{Approche directe}
%		\item \optword{à base des règles} (RBMT: rule-based machine translation)
%		\begin{itemize} 
%			\item Approche par transfert
%			\item Approche Interlangue
%		\end{itemize}
%	
%		\item \optword{à base du corpus}
%		\begin{itemize}
%			\item \optword{statistique} (SMT: statistical machine translation)
%			\item \optword{neuronale} (NMT: neural machine translation)
%		\end{itemize}
%	
%	\end{itemize}
	\hgraphpage{translation-classif.pdf}
\end{frame}

\begin{frame}
	\frametitle{Quelques applications : Transformation}
	\framesubtitle{Traduction automatique : Approche directe}
	\begin{itemize}
		\item Algorithme
		\begin{itemize}
			\item Le texte source $S$ est traité comme une série de mots 
			\item Chaque mot $S_i$ est remplacé par un mot $T_i$ dans le texte destinataire $T$ en utilisant un dictionnaire bilingue
			\item Les mots sont ordonnés. Ex. \expword{SVO \textrightarrow VSO}, \expword{adj + N \textrightarrow N + Adj}
		\end{itemize}
		\item Préconditions 
		\begin{itemize}
			\item Les langues (source et destinataire) doivent être proches (structures grammaticales proches)
			\item Un dictionnaire bilingue bien conçu
			\item Des outils d'analyse morphologiques 
		\end{itemize}
		\item Niveau morphologique 
		\item Ex. \expword{Systèmes avant 1967 : Météo, Weidner, CULT et Systran (premières versions)}
	\end{itemize}
\end{frame}

\begin{frame}
	\frametitle{Quelques applications : Transformation}
	\framesubtitle{Traduction automatique : Approche par transfert}
	\begin{itemize}
		\item Algorithme
		\begin{itemize}
			\item Trouver l'arbre syntaxique de $S$
			\item Chercher la traduction des mots de la langue source dans la langue destinataire (dictionnaire bilingue)
			\item Appliquer des règles pour transformer l'arbre syntaxique de la langue source vers une arbre syntaxique de la langue destinataire
			\item Générer le texte traduit $T$ à partir de cette arbre syntaxique
		\end{itemize}
		\item Préconditions 
		\begin{itemize}
			\item Grammaires des deux langues (source et destinataire)
			\item Les règles de transfert sont fixées manuellement ou apprises à partir d'un corpus
			\item Un dictionnaire bilingue bien conçu
			\item Des outils d'analyse morphologiques et syntaxique pour la langue source
			\item Un outil de génération du texte à partir de l'arbre syntaxique
		\end{itemize}
		\item Niveau syntaxique 
	\end{itemize}
\end{frame}

\begin{frame}
	\frametitle{Quelques applications : Transformation}
	\framesubtitle{Traduction automatique : Approche par transfert (Exemple de règles de transfert)}
	
	\begin{figure}
		\centering
		\vgraphpage[.7\textheight]{MT-tranfert-exp.pdf}
		\caption{Exemple de règles de transfert syntaxique \cite{06-quah}}
	\end{figure}
\end{frame}

\begin{frame}
	\frametitle{Quelques applications : Transformation}
	\framesubtitle{Traduction automatique : Approche par transfert (Exemple : Apertium \cite{11-forcada-al})}
	
	\begin{itemize}
		\item \url{https://www.apertium.org/}
	\end{itemize}

	\hgraphpage{apertium-arch.pdf}
\end{frame}

\begin{frame}
	\frametitle{Quelques applications : Transformation}
	\framesubtitle{Traduction automatique : Approche interlangue}
	\hgraphpage{MT-Interlingua.pdf}
\end{frame}

\begin{frame}
	\frametitle{Quelques applications : Transformation}
	\framesubtitle{Traduction automatique : Approche interlangue (Description)}
	\begin{itemize}
		\item Algorithme
		\begin{itemize}
			\item Analyser le texte source $S$ pour avoir un arbre syntaxique
			\item Utiliser un dictionnaire entre le langage source et les concepts de l'interlangue 
			\item Transformer l'arbre syntaxique (langue source) vers l'interlingue
			\item Transformer l'interlingue vers une arbre syntaxique (langue destinataire)
			\item Utiliser un dictionnaire entre le langage destinataire et les concepts de l'interlangue 
			\item Générer le texte destinataire $T$
		\end{itemize}
		\item Préconditions 
		\begin{itemize}
			\item Outils d'analyse syntaxique et même sémantique  pour les langues
			\item Un langage de représentation universelle
			\item Dictionnaire sur les concepts de ce langage
		\end{itemize}
		\item Niveau syntaxique/sémantique
	\end{itemize}
\end{frame}


\begin{frame}[fragile]
	\frametitle{Quelques applications : Transformation}
	\framesubtitle{Traduction automatique : Approche interlangue (Exemple KANT \cite{98-czuba-al})}
\begin{columns}
\begin{column}{0.5\textwidth}
\fontsize{4}{5}\selectfont\bfseries
\begin{verbatim}
(*A-REMAIN  ; action rep for ’remain’
   (FORM FINITE)
   (TENSE PAST)
   (MOOD DECLARATIVE)
   (PUNCTUATION PERIOD)
   (IMPERSONAL -) ; passive + expletive subject
   (ARGUMENT-CLASS THEME+PREDICATE) ; predicate argument structure
   (Q-MODIFIER ; PP semrole (generic)
      (*K-DURING ; PP interlingua
         (POSITION FINAL) ; clue for translation
            (OBJECT ; PP object semrole
               (*O-TIME ; object rep for ’time’
                  (UNIT -)
                  (NUMBER SINGULAR)
                  (REFERENCE DEFINITE)
                  (DISTANCE NEAR)
                  (PERSON THIRD)))))
   (THEME ; object semrole
      (*O-DEFAULT-RATE ; object rep for ’default rate’
         (PERSON THIRD)
         (UNIT -)
         (NUMBER SINGULAR)
         (REFERENCE DEFINITE)))
   (PREDICATE ; adjective phrase semrole
      (*P-CLOSE ; property rep for ’closer’
         (DEGREE POSITIVE)
         (Q-MODIFIER
            (*K-TO
               (OBJECT
                  (*O-ZERO
                     (UNIT -)
                     (NUMBER SINGULAR)
                     (REFERENCE NO-REFERENCE)
                     (PERSON THIRD))))))))
\end{verbatim}
\end{column}
\begin{column}{0.5\textwidth}
	\begin{figure}
		\caption{Représentation de la phrase "\textit{The default rate remained close to zero during this time.}" avec KANT interlingua \cite{98-czuba-al}}
	\end{figure}
\end{column}
\end{columns}


\end{frame}

\begin{frame}[fragile]
	\frametitle{Quelques applications : Transformation}
	\framesubtitle{Traduction automatique : Approche interlangue (Exemple système KANTOO \cite{00-nyberg-al})}
	
	\hgraphpage{kantoo-arch.pdf}
	
\end{frame}

\begin{frame}
	\frametitle{Quelques applications : Transformation}
	\framesubtitle{Traduction automatique : Approche statistique}
	\[
	p(T|S) = \frac{p(T) p(S|T)}{p(S)} \propto \underbrace{p(T)}_\text{Cohérence} \underbrace{p(S|T)}_\text{Fidélité}
	\]
	\[\hat{T} = \arg\max_{T} p(T) p(S|T)\]
	\[\]
	\begin{itemize}
		\item $p(T)$ est une modèle de langue entrainé sur la langue destinataire
		\begin{itemize}
			\item Il faut un corpus pour la langue destinataire
			\item $p(T) = \prod_{j=1}^m p(t_j|t_{j-N+1}\ldots t_{j-1})$
		\end{itemize}
		
		\item $p(S|T)$ est le modèle de traduction 
		\begin{itemize}
			\item Il faut un corpus pour la langue destinataire
			\item Le problème de correspondance entre mots (alignement)
			\item Le problème de la différence entre tailles
		\end{itemize}
	\end{itemize}
\end{frame}

\begin{frame}
	\frametitle{Quelques applications : Transformation}
	\framesubtitle{Traduction automatique : Approche statistique (Alignement)}
	\[p(S|T) = \sum_{A} p(S, A | T)\]
%	\[A^* = \arg\max_A p(S, A | T)\]
	\[p(S, A | T) = \prod_{i=1}^{n} p(S_i, A_i | S_1^{i-1}, A_1^{i-1}, T_1^{m})\]
	\[p(S_i, A_i | S_1^{i-1}, A_1^{i-1}, T_1^{m}) = p(A_i | S_1^{i-1}, A_1^{i-1}, T_1^{m}) p(S_i | S_1^{i-1}, A_1^{i}, T_1^{m})\]
	\[ \text{Nombre des alignements possibles } = (m + 1)^n\]
\end{frame}

\begin{frame}
	\frametitle{Quelques applications : Transformation}
	\framesubtitle{Traduction automatique : Approche statistique (Alignement IBM1)}
	\begin{itemize}
		\item Modèle IBM 1 (Distribution uniforme des mots)
	\end{itemize}

	\[p(A_i | S_1^{i-1}, A_1^{i-1}, T_1^{m}) = \frac{1}{m+1}\]
	\[p(S|T) = \frac{1}{(m+1)^n} \sum_{A} \prod_{i=1}^{n} p(S_i | S_1^{i-1}, A_1^{i}, T_1^{m})\]
\end{frame}

\begin{frame}
	\frametitle{Quelques applications : Transformation}
	\framesubtitle{Traduction automatique : Approche statistique (Alignement IBM2)}
	\begin{itemize}
		\item Modèle IBM 2
		\item La probabilité des alignements se base seulement sur la position actuelle $i$, le mot actuel $A_i$, la taille du texte source $n$ et la taille du texte destinataire $m$
		\item Cette probabilité est entrainée en comparant l'alignement juste avec le reste des alignements
	\end{itemize}
	
	\[p(A_i | S_1^{i-1}, A_1^{i-1}, T_1^{m}) = p(A_i | i, n, m)\]
	\[p(S|T) = \sum_{A} \prod_{i=1}^{n} p(A_i | i, n, m) p(S_i | S_1^{i-1}, A_1^{i}, T_1^{m})\]
\end{frame}

\begin{frame}
	\frametitle{Quelques applications : Transformation}
	\framesubtitle{Traduction automatique : Approche statistique (Alignement HMM)}
	\begin{itemize}
		\item Modèle HMM
		\item La probabilité de alignement d'un mot $A_i$ se base sur l'alignement du mot précédent $A_{i-1}$
	\end{itemize}
	
	\[p(A_i | S_1^{i-1}, A_1^{i-1}, T_1^{m}) = p(A_i | A_{i-1}, m)\]
	\[p(S|T) = \sum_{A} \prod_{i=1}^{n} p(A_i | A_{i-1}, m) p(S_i | S_1^{i-1}, A_1^{i}, T_1^{m})\]
\end{frame}

\begin{frame}
	\frametitle{Quelques applications : Transformation}
	\framesubtitle{Traduction automatique : Approche par exemples}
	\begin{itemize}
		\item Comme l'approche statistique
		\item A la place des mots, on peut trouver des segments
		\item On calcule la probabilité d'un segment (ensemble de mots consécutifs) par rapport à un autre
	\end{itemize}
\end{frame}

\begin{frame}
	\frametitle{Quelques applications : Transformation}
	\framesubtitle{Traduction automatique : Approche par exemples (Exemple : Moses \cite{07-koehn-al})}
	\begin{itemize}
		\item \url{http://statmt.org/moses/}
		\item Table de traduction des segments $\phi(S|T)$ : segment $S$, segment $T$ équivalent et une probabilité
		\item Modèle de langage sur la langue destinataire $LM$
		\item Modèle de distorsion $ D(T, S) $ : chaque réorganisation des segment d'une phrase veut un coût 
		\item Pénalité de mots $W(T)$ : pour qu'une traduction ne soit pas longue ou courte
		\item Pour estimer $\hat{T}$ \keyword{Beam search} est utilisé
	\end{itemize}

	\[p(T|S) = \phi(S|T)^{poids_{\phi}} \times LM^{poids_{LM}} \times D(T, S)^{poids_{D}} \times W(T)^{poids_{W}}\]
\end{frame}

\begin{frame}
	\frametitle{Quelques applications : Transformation}
	\framesubtitle{Traduction automatique : Approche neuronale}
	\begin{itemize}
		\item Structure
		\begin{itemize}
			\item Des réseaux de neurones de type sequence-to-sequence (seq2seq)
			\item Modèle : encodeur-décodeur (séquentiel)
			\item Type : Many to Many 
			\item \optword{Encodeur} : encoder une phrase du langage source $S$. Le résultat est une représentation du contexte sous forme d'un vecteur
			\item \optword{Décodeur} : décoder le vecteur du contexte vers une phrase du langage destinataire $T$
		\end{itemize}
	
		\item Modèle formel 
		\[ p(T|S) = p(t_1|S) p(t_2|S, t_1) p(t_3|S, t_1, t_2)\ldots p(t_m|S, t_1\ldots t_{m-1}) \]
		\[\hat{T} = \arg\max_{T} \prod_{i=1}^{m} p(t_i | S, t_1\ldots t_{i-1})\]
	\end{itemize}
\end{frame}

\begin{frame}
	\frametitle{Quelques applications : Transformation}
	\framesubtitle{Traduction automatique : Approche neuronale (Exemple : Google \cite{16-wu-al} )}
	\begin{itemize}
		\item \url{https://translate.google.com/}
	\end{itemize}
	\begin{center}
		\hgraphpage[.9\textwidth]{googlet.pdf}
	\end{center}
\end{frame}

\begin{frame}
	\frametitle{Quelques applications : Transformation}
	\framesubtitle{Traduction automatique : Approche neuronale (Exemple : OpenNMT \cite{17-klein-al})}
	\begin{itemize}
		\item \url{https://opennmt.net/}
	\end{itemize}
	\hgraphpage{opennmt.pdf}
\end{frame}

\begin{frame}
	\frametitle{Quelques applications : Transformation}
	\framesubtitle{Traduction automatique : Un peu d'humour}
	
	\begin{center}
		\hgraphpage[0.65\textwidth]{humour/humour-translation1.jpg}
		\hgraphpage[0.25\textwidth]{humour/humour-translation2.jpeg}
	\end{center}
	
\end{frame}

\subsection{Résumé automatique}

\begin{frame}
	\frametitle{Quelques applications : Transformation}
	\framesubtitle{Résumé automatique : Classification d'un résumé automatique}
	\hgraphpage{sum-classif.pdf}
\end{frame}

\begin{frame}
	\frametitle{Quelques applications : Transformation}
	\framesubtitle{Résumé automatique : Approches du résumé automatique}
	\begin{columns}
		\begin{column}{0.32\textwidth}
			\begin{block}{\scriptsize\bfseries\cite{12-nenkova-mckeown}}
				\begin{itemize}
					\item représentation du sujet 
					:
					mots du sujet,
					fréquences, 
					analyse sémantique latente, 
					modèles de sujets bayésiens,
					clustering
					%			\begin{itemize}
					%				\item mots du sujet 
					%				\item fréquences
					%				\item analyse sémantique latente
					%				\item modèles de sujets bayésiens
					%			\end{itemize}
					\item représentation des indicateurs : 
					par graphes, 
					apprentissage automatique
				\end{itemize}
			\end{block}
		\end{column}
		\begin{column}{0.3\textwidth}
			\begin{block}{\scriptsize\bfseries\cite{12-lloret-palomar}}
				\begin{itemize}
					\item statistique 
					\item par graphes
					\item basée discours
					\item par apprentissage automatique
				\end{itemize}
			\end{block}
		\end{column}
		\begin{column}{0.28\textwidth}
			\begin{block}{\scriptsize\bfseries\cite{19-aries-al}}
				\begin{itemize}
					\item statistique 
					\item par graphes
					\item linguistique 
					\item par apprentissage automatique
				\end{itemize}
			\end{block}
		\end{column}
	\end{columns}
\end{frame}

\begin{frame}
	\frametitle{Quelques applications : Transformation}
	\framesubtitle{Résumé automatique : Approche statistique}
	
	\begin{itemize}
		\item \optword{Fréquence des mots} 
		
		\hspace{.5cm}Ex. $Score_\text{TF-IDF}(s_i) = \sqrt{\sum\limits_{w_{ik} \in s_i} (\text{TF-IDF}(w_{ik}))^2}$
		
		\item \optword{Position des phrases (ou des mots)}
		
		\hspace{.5cm}Ex. $ Score_\text{pos}(s_i) = \max (\frac{1}{i}, \frac{1}{|D| - i + 1}) $
		
		\item \optword{Taille (longueur) des phrases}
		
		\hspace{.5cm}Ex. $ Score_\text{taille}(s_i) = \left\lbrace 
		\begin{array}{lll}
		0 & si & (L_i \geq L_{min}) \\
		\frac{L_i - L_{min}}{L_{min}} & sinon & \\
		\end{array}
		\right. $
		
		\item \optword{Mots du titre et des sous-titres}
		
		\hspace{.5cm}Ex. $ Score_{titre}(s_i) = \frac{\sum_{e \in T \bigcap s_i}{\frac{tf(e)}{tf(e)+1}}}
		{\sum_{e \in T}{\frac{tf(e)}{tf(e)+1}}} $
		
		\item \optword{Centroid}, \optword{Frequent itemsets}, \optword{Analyse sémantique latente}, ...
	\end{itemize}
\end{frame}

\begin{frame}
	\frametitle{Quelques applications : Transformation}
	\framesubtitle{Résumé automatique : Approche statistique (Exemple : \cite{13-aries-al})}
	
	\hgraphpage{tcc-arch.pdf}
	
\end{frame}

\begin{frame}
	\frametitle{Quelques applications : Transformation}
	\framesubtitle{Résumé automatique : Approche statistique (Exemple: \cite{13-aries-al} - suite)}
	
	\begin{itemize}
		\item Un texte peut contenir plusieurs sujets, et une phrase peut discuter plusieurs sujets
		\begin{itemize}
			\item Regroupement de phrases avec similarité et seuil de regroupement ($Th$)
		\end{itemize}
		\item Une phrase est importante si elle peut représenter le maximum des sujets(cluters)
		\[ Score(s_i , c_j , f_k ) = 1 + \sum_{\phi \in s_i} {P(f_k=\phi | s_i \in c_j)} \]
		\[ Score(s_i , \bigcap_{j} c_j , F) =  %\propto 
		\prod_{j} \prod_{k} Score(s_i , c_j , f_k ) \]
		$ s $ : phrase, $ c $ : cluster, $ f $ : caractéristique, $ F $ : ensemble de caractéristiques, $ \phi $: observation de $ f $.
		\item $f$ : TF (Uni, Bi); Pos (intervalle de 10); Len (réel, pré-traité)
	\end{itemize}
	
\end{frame}

\begin{frame}
	\frametitle{Quelques applications : Transformation}
	\framesubtitle{Résumé automatique : Approche statistique (Exemple: \cite{15-oufaida-al})}
	\begin{itemize}
		\item Représentation des mots : embeddings pré-entrainés (Polyglot)
		\item Clustering pour extraire les sous-sujets du texte
		\begin{itemize}
			\item Chercher le mot $w_j \in S_2$ le plus similaire à un mot $w_i$
			
			\[Match(w_i | S_2) = \arg\max_{w_j \in S_2} sim(Rep(w_i), Rep(w_j))\]
			
			\item Similarité entre $S_1$ et $S_2$
			
			\[Sim(S_1, S_2) = \frac{\sum_{w_i \in S_1} Match(w_i | S_2) + \sum_{w_j \in S_2} Match(w_j | S_1)}{|S_1| + |S_2|}\]
			
			\item Utilisation d'un algorithme de clustering sur la matrice de similarités Phrases/Phrases
		\end{itemize}
	
	\end{itemize}
	
\end{frame}

\begin{frame}
	\frametitle{Quelques applications : Transformation}
	\framesubtitle{Résumé automatique : Approche statistique (Exemple: \cite{15-oufaida-al} - suite)}

	\begin{itemize}
		\item Score des termes en utilisant mRMR
		\begin{itemize}
			\item Information mutuelle (X, Y : vecteurs Terme/Phrases) 
			
			\begin{center}
				$I(X, Y) = \sum\limits_{x \in X} \sum\limits_{y \in Y} p(x, y) \log \frac{p(x, y)}{p(x) p(y)}$
			\end{center}
			
			\item Pertinence des termes (H : vecteur Cluster/phrases)
			\begin{center}
				$Pertinence(T_i) = I(T_i, H)$
			\end{center}
			
			\item Redondance des termes (S : toutes les phrases)
			
			\begin{center}
				$Redondance(T_i) = \frac{1}{|S|} \sum\limits_{j \in S} I(T_i, T_j)$
			\end{center}
			
			\item Score final d'un terme : vecteur mRMR des termes
			
			\begin{center}
				$MID \equiv \max_{t \in T} Pertinence(t) - Redondance(t)$
			
			$MIQ \equiv \max_{t \in T} Pertinence(t) / Redondance(t)$
			\end{center}
		\end{itemize}
		
		\item Score d'une phrase est sa similarité avec le vecteur mRMR
		
		\item Lors de la sélection d'une phrase, on décrémente les poids des termes sélectionnés dans mRMR
	\end{itemize}
	
\end{frame}

\begin{frame}
	\frametitle{Quelques applications : Transformation}
	\framesubtitle{Résumé automatique : Approche par graphes}
	
	\begin{itemize}
		\item \optword{Propriétés du graphe}
		\begin{itemize}
			\item Bushy paths 
			
			\hspace{.5cm}$Score_{\#arcs}(s_i) = |\{ s_j : a(s_i, s_j) \in A / s_j \in S, s_i \neq s_j \}|$
			
			\item Aggregate Similarity
			
			\hspace{.5cm}$Score_{aggregate}(s_i) = \sum\limits_{(s_i, s_j) \in E} sim(s_i, s_j)$
		\end{itemize}
		\item \optword{Méthodes itératives}
		\begin{itemize}
			\item Mettre à jour les scores des nœuds à base des voisins 
			\item L'arrêt : un état d'équilibre (on peut plus mettre à jour)
			\item Ex. TextRank \cite{04-mihalcea-tarau}
			
			$WS(V_i) = ( 1 - d) + d * \sum\limits_{V_j \in In(V_i)} \frac{w_{ji}}{\sum\limits_{V_k \in Out(V_j)} w_{jk}} WS(V_j)$
			
			$w_{ij} = \frac{|\{w_k \text{ / } w_k \in S_i \text{ and } w_k \in S_j\}|}{\log(|S_i|) + \log(|S_j|)}$
			
			$ d $ : damping factor (en général, environ $ 0.85 $)
		\end{itemize}
	\end{itemize}

\end{frame}

\begin{frame}
	\frametitle{Quelques applications : Transformation}
	\framesubtitle{Résumé automatique : Approche par graphes (Exemple: \cite{21-aries-al})}
	
	\begin{center}
		\vgraphpage{gc-archi.pdf}
	\end{center}
	
\end{frame}

\begin{frame}
	\frametitle{Quelques applications : Transformation}
	\framesubtitle{Résumé automatique : Approche par graphes (Exemple : \cite{21-aries-al} - suite)}
	
	\vspace{-6pt}
	\begin{itemize}
		\item Simplification du graphe
		
		\hspace{.5cm}$noeud\_faible(v_i) = ( \sum_{(v_i, v_j) \in E} w_{ij} < \frac{1}{MImpN(v_i)} )$ 
		
		\hspace{.5cm}$arc\_faible(v_i, v_j) = ( w_{ij} < \frac{Threshold}{MImpN(v_i)})$
		
		\item Score statistique des phrases
		
		\hspace{.5cm}$ Score(s_i/ sim) = sim(s_i, C\backslash s_i) $
		$Score(s_i/ tfisf) = \sqrt{\sum\limits_{w_{ik} \in s_i} (tfisf(w_{ik}))^2}$
		
		\hspace{.5cm}$Score(s_i/ size) = \frac{1}{|s_i|}$
		$Score(s_i/ pos) = \max (\frac{1}{i}, \frac{1}{|D| - i + 1})$
		
		\hspace{.5cm}$SSF(s_i/ F) = \prod_{f_i \in F} score(s_i/f_i)$
		
		\item Score à base de graphe 
		
		\hspace{.5cm}Ex. $GC1(s_i) = SSF(s_i) + \sum\limits_{(s_i, s_j) \in E} sim(s_i, s_j) * SSF(s_j)$
		
		\item Extraction
		
		\hspace{.5cm}Ex. $ suiv_{e4}  =  \arg\min\limits_i (iord\ gc(s_i) + ord\ sim(dernier_{e4}, i))$ 
		
		\hspace{3cm}$ \text{ où } (dernier_{e4}, s_i) \in E $
	\end{itemize}
	
\end{frame}

\begin{frame}
	\frametitle{Quelques applications : Transformation}
	\framesubtitle{Résumé automatique : Approche linguistique}
	
	\begin{itemize}
		\item \optword{Mots de sujet} : une liste des mots pertinents au sujet, comme ``significant", ``impossible", etc.
		\begin{itemize}
			\item Ex. \cite{69-edmundson} : Bonus (mots positivement pertinents), Stigma (mots négativement pertinents)
			
			$Score_{cue}(s_i) = \sum_{w \in s_i}{cue(w)}
			\text{ où }
			cue(w) = \left\lbrace 
			\begin{array}{ll}
			b > 0 & \text{si } (w \in Bonus) \\
			\delta < 0 & \text{si } (w \in Stigma) \\
			0 & sinon 
			\end{array} 
			\right. $
		\end{itemize}
		\item \optword{Indicateurs} : Des structures qui impliquent que la phrase les contenant a une chose importante à propos du sujet
		\begin{itemize}
			\item Ex. \expword{the principal aim of this paper is to investigate ...}
		\end{itemize}
		\item \optword{Co-référence} : utilisation des anaphores ou des représentations sémantiques (Ex. Wordnet)
		\item \optword{Structure rhétorique} : utilisation de la structure rhétorique pour noter les phrases ou des syntagmes
	\end{itemize}
	
\end{frame}

\begin{frame}
	\frametitle{Quelques applications : Transformation}
	\framesubtitle{Résumé automatique : Approche linguistique (Exemple : \cite{81-paice})}
	
	\begin{itemize}
		\item Utilisation des patrons (templates) préparés manuellement
		\item\ [x] : x mots peuvent être entre ce mot et le mot précédent
		\item +y : le score est augmenté par y
		\item ? : ce mot est optionnel
	\end{itemize}

	\begin{figure}[!ht]
		\begin{center}
			\hgraphpage[.7\textwidth]{paice-template.pdf}
			\caption{Un exemple d'un patron simplifié \cite{81-paice}.}
			\label{fig:paice-template}
		\end{center}
	\end{figure}
	
\end{frame}

%\begin{frame}
%	\frametitle{Quelques applications : Transformation}
%	\framesubtitle{Résumé automatique : Approche linguistique (Exemple : \cite{})}
%	
%	\begin{itemize}
%		\item 
%	\end{itemize}
%	
%\end{frame}

\begin{frame}
	\frametitle{Quelques applications : Transformation}
	\framesubtitle{Résumé automatique : Approche par apprentissage automatique (AA)}
	
	\begin{itemize}
		\item \optword{Par caractéristiques}
		\begin{itemize}
			\item Réglage : régler des hyper-paramètres comme les poids des caractéristiques pour le score
			\item Classement : décider si une unité (phrase) appartient au résumé ou non
		\end{itemize}
		\item \optword{Bayesian topic models}
		\begin{itemize}
			\item Identifier les concepts principaux à partir des documents et les liens entre eux pour avoir une hiérarchie
		\end{itemize}
		\item \optword{Deep learning}
		\begin{itemize}
			\item Réglage ou Classement
			\item Génération des mots (une forme du classement)
		\end{itemize}
		\item \optword{Reinforcement learning}
		\begin{itemize}
			\item Utilisation des actions et des récompenses pour entrainer un système à générer des résumés
		\end{itemize}
	\end{itemize}
	
\end{frame}


\begin{frame}
	\frametitle{Quelques applications : Transformation}
	\framesubtitle{Résumé automatique : Approche par AA (Exemple : \cite{2020-aries})}
	
	\begin{center}
		\vgraphpage{ml2es-archi.pdf}
	\end{center}
	
\end{frame}

\begin{frame}
	\frametitle{Quelques applications : Transformation}
	\framesubtitle{Résumé automatique : Approche par AA (Exemple : \cite{06-daumeiii-marcu})}
	
	\begin{minipage}{.6\textwidth}
		\begin{itemize}
			\item $D$ : un ensemble de $K$ documents
			\item $Q$ : un ensemble de $J$ requêtes
			\item $P^G$ : un modèle de langue général de l'anglais 
			\item $P^Q$ : un modèle de langue des requêtes
			\item $P^D$ : un modèle de langue d'arrière-plan (des documents)
		\end{itemize}
	\end{minipage}
	\begin{minipage}{.38\textwidth}
		\hgraphpage{btm-daumeiii.pdf}
	\end{minipage}
	
\end{frame}

\begin{frame}
	\frametitle{Quelques applications : Transformation}
	\framesubtitle{Résumé automatique : Approche par AA (Exemple : \cite{15-rush-al})}
	
	\begin{center}
		\hgraphpage[.35\textwidth]{2015-rush-al.pdf}
		\hgraphpage[.35\textwidth]{2015-rush-al-exp.pdf}
	\end{center}
	
	\begin{itemize}
		\item[(a)] Un décodeur qui cherche le mot prochain du résumé $y_{i+1}$ sachant la phrase en entrée $x$ et des mots déjà générés pour le résumé $y_c \equiv [y_{i-c+1},\ldots, y_i]$
		\item[(b)] Un encodeur avec attention
	\end{itemize}
	
\end{frame}

\begin{frame}
	\frametitle{Quelques applications : Transformation}
	\framesubtitle{Résumé automatique : Approche par AA (Exemple : \cite{18-narayan-al})}
	
	\hgraphpage{narayan-al.pdf}
	
\end{frame}

\begin{frame}
	\frametitle{Quelques applications : Transformation}
	\framesubtitle{Résumé automatique : Un peu d'umour}
	
	\begin{center}
		\hgraphpage[0.6\textwidth]{humour/humour-summarize1.jpg}
		\hgraphpage[0.3\textwidth]{humour/humour-summarize2.jpg}
	\end{center}
	
\end{frame}

%===================================================================================
\section{Interaction}
%===================================================================================

\begin{frame}
	\frametitle{Quelques applications}
	\framesubtitle{Interaction}
	\begin{itemize}
		\item \textbf{Entrée} : une question (texte)
		\item \textbf{Sortie} : une réponse (texte)
		\item \textbf{Applications} 
		\begin{itemize}
			\item Questions/Réponses
			\item Systèmes de dialogue
		\end{itemize}
	\end{itemize}
\end{frame}

\subsection{Questions-Réponses}

\begin{frame}
	\frametitle{Quelques applications : Interaction}
	\framesubtitle{Questions-Réponses}
%	\begin{itemize}
%		\item 
%	\end{itemize}
	\hgraphpage{qa-classif.pdf}
\end{frame}

\begin{frame}
	\frametitle{Quelques applications : Interaction}
	\framesubtitle{Questions-Réponses : Approche par RI}
	
	\begin{figure}
		\hgraphpage{qa-ri.pdf}
		\caption{Architecture d'un système de questions/réponses \cite{2019-jurafsky-martin}}
	\end{figure}
	
\end{frame}

\begin{frame}
	\frametitle{Quelques applications : Interaction}
	\framesubtitle{Questions-Réponses : Approche par RI (traitement de la question)}
	
	\begin{itemize}
		\item Formulation de la requête
		\begin{itemize}
			\item Séparation des mots
			\item Suppression des mots vides
			\item Radicalisation  
		\end{itemize}
	    \item Détection du type de réponse
	    \begin{itemize}
	    	\item Personne ? Place ? Organisation ? ...
	    	\item En utilisant des taxonomies comme celle de Wordnet  
	    \end{itemize}
	\end{itemize}
	
\end{frame}

\begin{frame}
	\frametitle{Quelques applications : Interaction}
	\framesubtitle{Questions-Réponses : Approche par RI (recherche)}
	
	\begin{itemize}
		\item Recherche des documents
		\begin{itemize}
			\item En utilisant les mots clés et l'index des documents
			\item Retourner les documents avec plus de score
			\item Diviser ces documents en passages (paragraphes ou phrases)
		\end{itemize}
		\item Recherche des passages
		\begin{itemize}
			\item Appliquer la détection des entités nommées sur les passages
			\item Filtrer les passages qui ne contiennent pas le type de la réponse
			\item On peut appliquer un algorithme d'apprentissage automatique
			\item Cette algorithme peut être utilisé pour noter les passages
			\item Caractéristiques : nombre des entités nommées du type recherché, nombre des termes de la question, la séquence la plus longue similaire à la question, le rang du document, etc.
		\end{itemize}
	\end{itemize}
	
\end{frame}

\begin{frame}
	\frametitle{Quelques applications : Interaction}
	\framesubtitle{Questions-Réponses : Approche par RI (extraction de réponse)}
	
	\begin{itemize}
		\item \optword{Apprentissage par caractéristiques}
		\begin{itemize}
			\item Patrons : \expword{\textless REP\textgreater comme \textless QES\textgreater; }
			\item Type de réponse et celui du syntagme
			\item Les mots clés de la question
			\item Nouveauté : au moins un mot n'existe pas dans la question
			\item Ponctuation
			\item ...
		\end{itemize}
		\item \optword{Réseaux de neurones}
		\begin{itemize}
			\item Tâche de lecture/compréhension
			\item Pour chaque mot, calculer la probabilité d'être le début de la réponse et la fin de la réponse en utilisant la question. 
		\end{itemize}
	\end{itemize}
	
\end{frame}

\begin{frame}
	\frametitle{Quelques applications : Interaction}
	\framesubtitle{Questions-Réponses : Approche par RI (extraction par Bi-LSTM \cite{2017-chen-al})}
	
	\begin{figure}
		\hgraphpage{qa-bilstm-exp.pdf}
		\caption{Extraction de réponse dans le système DrQA \cite{2019-jurafsky-martin}}
	\end{figure}
	
\end{frame}

\begin{frame}
	\frametitle{Quelques applications : Interaction}
	\framesubtitle{Questions-Réponses : Approche par RI (extraction par BERT \cite{2018-devlin-al})}
	
	\begin{figure}
		\centering
		\hgraphpage[.5\textwidth]{qa-bert-exp.pdf}
		\caption{Extraction de réponse dans le système DrQA \cite{2019-jurafsky-martin}}
	\end{figure}
	
\end{frame}

\begin{frame}
	\frametitle{Quelques applications : Interaction}
	\framesubtitle{Questions-Réponses : Approche par connaissance}
	
	\begin{itemize}
		\item \optword{Par graphe}
		\begin{itemize}
			\item Annotation sémantique (Entity linking)
			\item Déterminer la relation recherchée ; Ex. \expword{Place\_naissance}
			\item S'il y a plusieurs relations comme réponse : calculer la similarité entre la question et la réponse
			\item Exemple, \expword{en utilisant les embeddings}
		\end{itemize}
		\item \optword{Par analyse sémantique}
		\begin{itemize}
			\item Analyse sémantique de la question
			\item Génération d'une forme structurée de la question : lambda calcul, SQL, SPARQL, etc.
			\item Cette forme est utilisée comme requête
			\item Récupérer la réponse à partir de la base de connaissance
		\end{itemize}
	\end{itemize}
	
\end{frame}

\begin{frame}
	\frametitle{Quelques applications : Interaction}
	\framesubtitle{Questions-Réponses : Approche par connaissance (Exemple)}
	
	\begin{exampleblock}{Exemple des formes logiques des questions \cite{2020-jurafsky-martin}}
		\centering\tiny\bfseries
		\begin{tabular}{p{0.5\textwidth}p{0.45\textwidth}}
			\hline\hline
			Question & Forme logique \\
			\hline
			Quelles sont les wilayas limitrophes de Jijel ? & $\lambda x.wilaya(x) \wedge limitrophe(x, Jijel)$ \\
			
			Quelle est la wilaya la plus grande ? & $\arg\max(\lambda x.wilaya(x), \lambda x.superficie(x))$ \\
			
			Je souhaite réserver un vol de Jijel vers Alger. & {\color{blue}SELECT DISTINCT} f1.flight id
			
			{\color{blue}FROM} vol v1, aeroport  a1,
			
			wilaya w1, service\_aeroport a2, wilaya w2
			
			{\color{blue}WHERE} v1.depart=a1.code\_aeroport
			
			{\color{blue}AND} a1.code\_wilaya=w1.code\_wilaya
			
			{\color{blue}AND} w1.nom\_wilaya= '{\color{red}Jijel}'
			
			{\color{blue}AND} v1.destination=a2.code\_aeroport
			
			{\color{blue}AND} a2.code\_wilaya=w2.code\_wilaya
			
			{\color{blue}AND} w2.nom\_wilaya= '{\color{red}Alger}' \\
			
			
			Combien de personnes ont survécu au naufrage du Titanic ? & (count (!fb:event.disaster.survivors
			
			fb:en.sinking of the titanic))\\
			
			Combien de yards de plus a duré le touché le plus long de Johnson par rapport à son touché le plus court du premier quart-temps ? & 
			ARITHMETIC diff( SELECT num( ARGMAX(
			
			SELECT ) ) SELECT num( ARGMIN( FILTER(
			
			SELECT ) ) ) )\\
			\hline\hline
		\end{tabular}
	\end{exampleblock}
	
	
\end{frame}

\begin{frame}
	\frametitle{Quelques applications : Interaction}
	\framesubtitle{Questions-Réponses : Approche par modèles de langues}
	
	\begin{itemize}
		\item Utiliser un modèle de langue pré-entrainé
		\item Régler le modèle pour répondre aux questions
	\end{itemize}

	\begin{figure}
		\centering
		\hgraphpage[.6\textwidth]{qa-t5.pdf}
		\caption{T5 est pré-entraîné pour remplir le texte absent, ensuite réglé pour répondre aux questions sans saisir d'informations ou de contexte supplémentaires. \cite{2020-roberts-al}}
	\end{figure}
	
\end{frame}

\begin{frame}
	\frametitle{Quelques applications : Interaction}
	\framesubtitle{Questions-Réponses : Un peu d'humour}
	
	\begin{columns}
		\begin{column}{0.3\textwidth}
			\hgraphpage{humour/humour-QR1.jpg}
			
			\hgraphpage{humour/humour-QR3.jpeg}
		\end{column}
		\begin{column}{0.4\textwidth}
			\hgraphpage{humour/humour-QR2.jpeg}
		\end{column}
	\end{columns}

\end{frame}

\subsection{Systèmes de dialogue}

\begin{frame}
	\frametitle{Quelques applications : Interaction}
	\framesubtitle{Systèmes de dialogue}
%	\begin{itemize}
%		\item Systèmes de dialogue ou agents conversationnels
%		\item \textbf{Moyenne} : texte ou parole
%		\item \optword{Systèmes de dialogue orientés tâche}
%		\begin{itemize}
%			\item \textbf{BUT} : aider l'utilisateur à compléter une tâche
%			\item Ex. \expword{Siri, Alexa, Google Now/Home, Cortana, etc.}
%		\end{itemize}
%		\item \optword{Chatbots}
%		\begin{itemize}
%			\item \textbf{BUT} : imiter les conversations entre humains
%			\item \textbf{Méthodes} : à base de règles, à base de RI, à base des encodeurs-décodeurs
%		\end{itemize}
%	\end{itemize}
	\hgraphpage{sd-classif.pdf}
\end{frame}

\begin{frame}
	\frametitle{Quelques applications : Interaction}
	\framesubtitle{Systèmes de dialogue : orienté tâche - Frame-based (Frame)}
	
	\begin{itemize}
		\item Frame (Cadre) : une structure contenant des slots à remplir et des questions prédéfinies pour chaque slot
		\item On ne pose que les questions dont le slot est vide
		\item Il y a la possibilité de remplir d'autres slots dans d'autre cadres. 
		Ex. \expword{Le slot RESERVATION\_DATE dans le cadre HOTEL\_RESERVATION à partir du slot ARRIVAL\_DATE du cadre FLIGHT\_RESERVATION}
		\item Appliquer \keyword{Détection d'intention} pour savoir quel cadre à utiliser
	\end{itemize}

	\begin{exampleblock}{Exemple d'un cadre pour programmer un vol \cite{2020-jurafsky-martin}}
		\centering\tiny\bfseries
		\begin{tabular}{lll}
			\hline\hline
			Slot & Type & Modèle de questions \\
			\hline
			VILLE DEPART & ville & ``De quelle ville partez-vous ?" \\
			VILLE DESTINATION & ville & ``Où allez-vous?" \\
			HEURE DEPART & temps & ``Quand souhaitez-vous partir ?" \\
			DATE DEPART & date & ``Quel jour souhaitez-vous partir ?" \\
			HEURE ARRIVEE & temps & ``Quand voulez-vous arriver ?" \\
			DATE ARRIVEE & date & ``Quel jour souhaitez-vous arriver ?" \\
			\hline\hline
		\end{tabular}
	\end{exampleblock}
	
%	\begin{figure}
%		\centering
%		\hgraphpage[.6\textwidth]{sd-frame-exp.pdf}
%		\caption{Exemple d'un cadre pour programmer un vol. \cite{2020-jurafsky-martin}}
%	\end{figure}
	
\end{frame}

\begin{frame}
	\frametitle{Quelques applications : Interaction}
	\framesubtitle{Systèmes de dialogue : orienté tâche - Frame-based (Réponses multiples)}
	
%	\begin{figure}
		\centering\footnotesize
		\begin{tabular}{lll}
			MONTRER & \textrightarrow & montrez-moi \textbar\ Je veux savoir \textbar\ puis-je voir \\
			HEURE\_DEP & \textrightarrow & (après \textbar\ environ \textbar\ avant) HEURE \textbar HEURE \textbar\\
			&  & matin \textbar\ après-midi \textbar\ soir\\
			DATE\_DEP & \textrightarrow & samedi \textbar\ ... \textbar\ vendredi\\
			HEURE & \textrightarrow & (un \textbar\ deux \textbar\ ... \textbar\ vingt) (am \textbar\ pm) \\
			VOLS & \textrightarrow & (un \textbar\ le) vol \textbar\ (les \textbar\ des) vols \\
			ORIGINE & \textrightarrow & (de \textbar\ à partir de) WILAYA \\
			DESTINATION & \textrightarrow & (à \textbar\ jusqu'à) WILAYA \\
			WILAYA & \textrightarrow & Adrar \textbar\ ... \textbar\ El Meniaa \\
		\end{tabular}
		
		\hgraphpage[.8\textwidth]{sd-frame-parse-exp.pdf}
%		\caption{Exemple d'une grammaire sémantique et un arbre sémantique d'une phrase. \cite{2020-jurafsky-martin}}
%	\end{figure}
	
\end{frame}

\begin{frame}
	\frametitle{Quelques applications : Interaction}
	\framesubtitle{Systèmes de dialogue : orienté tâche - Dialogue-State}
	
	\begin{figure}
		\centering
		\hgraphpage[.7\textwidth]{sd-dialog-arch.pdf}
		\caption{Architecture d'un système utilisant dialogue-state \cite{2016-williams-al}}
	\end{figure}
	
\end{frame}

\begin{frame}
	\frametitle{Quelques applications : Interaction}
	\framesubtitle{Systèmes de dialogue : orienté tâche - Dialogue-State (Actes du dialogue)}
	
	\vspace{-6pt}
	\begin{figure}
		\centering\tiny\bfseries
		\begin{tabular}{llll}
			\hline
			Tag & Sys & User & Description \\
			\hline
			HELLO (a = x, b = y, ...) & \CheckedBox & \CheckedBox & Ouvrir un dialogue et donne l'info a = x, b = y, ... \\
			INFORM(a = x, b = y, ...) & \CheckedBox & \CheckedBox & Donner l'info a = x, b = y, ... \\
			REQUEST(a, b = x, ...) & \CheckedBox & \CheckedBox & Demander une valeur pour une donnée b = x, ... \\
			REQALTS(a = x, ...) & \XBox & \CheckedBox & Demander une alternative avec a = x, ... \\
			CONFIRM(a = x, b = y, ...) & \CheckedBox & \CheckedBox & Confirmer explicitement a = x, b = y, ... \\
			CONFREQ (a = x, ..., d) & \CheckedBox & \XBox & Confirmer implicitement a = x, ... et demander la valeur de d \\
			SELECT(a = x, a = y) & \CheckedBox & \XBox & Confirmer implicitement a = x, ... et demander la valeur de d \\
			AFFIRM(a = x, b = y, ...) & \CheckedBox & \CheckedBox & Confirmer et donner plus d'info a = x, b = y, ... \\
			NEGATE(a = x) & \XBox & \CheckedBox & Annuler et donner la valeur correcte a = x \\
			DENY(a = x) & \XBox & \CheckedBox & Refuser a = x \\
			BYE () & \CheckedBox & \CheckedBox & Fermer une dialogue \\
			\hline
		\end{tabular}
	
		\begin{tabular}{p{0.5\textwidth}p{0.4\textwidth}}
			\hline
			Énoncé & Acte de dialogue \\
			\hline
			U: Salut, je cherche un endroit pour manger.  & hello(task=find, type=restaurant) \\
			S: Vous recherchez un restaurant. Quel type de nourriture aimez-vous?  & confreq(type=restaurant, food) \\
			U: Je voudrais un plat italien quelque part près du musée.  & inform(food=Italian, near=museum)\\
			S: ``Roma" est un bon restaurant italien près du musée.  & inform(name=``Roma", type=restaurant, food=Italian, near=museum) \\
			U: Les prix sont-ils raisonnables ? & confirm(pricerange=moderate) \\
			S: Oui, les prix de ``Roma" sont raisonnables.  & affirm(name=``Roma", pricerange=moderate) \\
			U: Quel est son numéro de téléphone ? & request(phone) \\
			S: Le nombre de ``Roma" est 385456. & inform(name=``Roma", phone=``385456") \\
			U: Ok, merci. Au revoir. & bye() \\
			\hline
		\end{tabular}
		
		\caption{\vspace{-1cm}Recommandation des restaurants HIS \cite{2010-young-al}}
	\end{figure}
	
\end{frame}

\begin{frame}
	\frametitle{Quelques applications : Interaction}
	\framesubtitle{Systèmes de dialogue : orienté tâche - Dialogue-State (Remplissage des slots)}
	
	\begin{itemize}
		\item Classifier la phrase par intention, domaine et slot
		\item Extraire les informations pour remplir les slots. Ex. \expword{En utilisant l'apprentissage automatique pour l'étiquetage des séquences}
	\end{itemize}
	
	\begin{figure}
		\centering
		\hgraphpage[.4\textwidth]{sd-dialog-remp-exp.pdf}
		\caption{Exemple d'une architecture pour le remplissage des slots en utilisant BERT \cite{2020-jurafsky-martin}}
	\end{figure}
	
\end{frame}

\begin{frame}
	\frametitle{Quelques applications : Interaction}
	\framesubtitle{Systèmes de dialogue : orienté tâche - Dialogue-State (Autres composants)}
	
	\begin{itemize}
		\item Traqueur de l'état du dialogue
		\begin{itemize}
			\item Sauvegarder l'état des cadres (slots) et le dernier acte du dialogue
		\end{itemize}
		\item Politique du dialogue
		\begin{itemize}
			\item Déterminer l'action $A_i$ à prendre 
			
			$\hat{A}_i = \arg\max_{A_i \in A} P(A_i | Frame_{i-1}, A_{i-1}, U_{i-1})$
		\end{itemize}
	
		\item Générateur du texte 
		\begin{itemize}
			\item Générer du texte à partir d'un acte de dialogue
			\item On peut entraîner un encodeur/décodeur pour le faire
		\end{itemize}
	\end{itemize}
	
\end{frame}

\begin{frame}
	\frametitle{Quelques applications : Interaction}
	\framesubtitle{Systèmes de dialogue : chatbot - à base des règles (Exemple : ELIZA \cite{1966-Weizenbaum})}
	
	\begin{itemize}
		\item Le plus célèbre chatbot est \keyword{ELIZA} : psychologue
		\item Une liste des patrons/transformations 
		
		\expword{\small [(.*) YOU (.*) ME]\textsubscript{[Patron]} \textrightarrow\ [WHAT MAKES YOU THINK I \$2 YOU?]\textsubscript{[Transformation]}}
		
		\expword{You hate me \textrightarrow\ WHAT MAKES YOU THINK I HATE YOU?}
		
		\item Les patrons sont liés à une liste des mots. Le mot qui score le plus dans la phrase va déclencher plusieurs patrons
		
		\item Parmi les, le patron le plus similaire à la phrase de l'utilisateur est utilisé
	\end{itemize}
	
\end{frame}

\begin{frame}
	\frametitle{Quelques applications : Interaction}
	\framesubtitle{Systèmes de dialogue : chatbot - à base de la RI}
	
	\begin{itemize}
		\item Un corpus des conversations $C$
		\item On prend le texte de l'utilisation comme une requête (question) $q$
		\item On cherche la réponse $r \in C$ qui est plus similaire à la requête $q$
		\[\text{Réponse}(q, C) = \arg\max_{r \in C} \frac{q . r}{|q| |r|}\]
		\item Pour calculer la similarité, on peut utiliser TF-IDF
		\item On peut aussi encoder la requête $q$ et la réponse $r$ en utilisant les embeddings
		\[h_q = BERT_Q(q)[CLS],\; h_r = BERT_R(r)[CLS]\]
		\[\text{Réponse}(q, C) = \arg\max_{r \in C} q . r\]
	\end{itemize}
\end{frame}

\begin{frame}
	\frametitle{Quelques applications : Interaction}
	\framesubtitle{Systèmes de dialogue : chatbot - par génération du texte}
	
	\begin{itemize}
		\item Utiliser un encodeur/décodeur comme dans la traduction automatique
		\[ \hat{r}_i = \arg\max_{w \in V} p(w| q, r_1, \ldots, r_{t-1}) \]
		
		\item Dans l'encodeur, introduire un contexte plus long
		
		\item On peut utiliser un modèle de langue, comme GPT, afin de l'entraîner sur des conversations
	\end{itemize}
	
	\begin{figure}
		\centering
		\hgraphpage[.7\textwidth]{sd-chatbot-encdec-exp.pdf}
		\caption{Exemple d'un chatbot à base d'un encodeur/décodeur \cite{2020-jurafsky-martin}}
	\end{figure}
	
\end{frame}


\begin{frame}
	\frametitle{Quelques applications : Interaction}
	\framesubtitle{Systèmes de dialogue : Un peu d'humour}
	
	\begin{center}
		\hgraphpage[0.5\textwidth]{humour/humour-chatbots1.jpeg}
		\hgraphpage[0.4\textwidth]{humour/humour-chatbots2.jpeg}
	\end{center}
\end{frame}


%===================================================================================
\section{Classification}
%===================================================================================

\begin{frame}
	\frametitle{Quelques applications}
	\framesubtitle{Classification}
	
	\begin{itemize}
		\item \textbf{Entrée} : un texte
		\item \textbf{Sortie} : une catégorie
		\item \textbf{Applications} 
		\begin{itemize}
			\item Filtrage de spams
			\item Identification des langues
			\item Lisibilité 
			\item Analyse des sentiments
			\item Détection de l'humeur
		\end{itemize}
	\end{itemize}

\end{frame}

\subsection{Analyse des sentiments}

\begin{frame}
	\frametitle{Quelques applications : Classification}
	\framesubtitle{Analyse des sentiments}
	\hgraphpage{sentiment-classif.pdf}
\end{frame}

\begin{frame}
	\frametitle{Quelques applications : Classification}
	\framesubtitle{Analyse des sentiments : à base de connaissance}
	
	\begin{itemize}
		\item Identifier les mots indicateurs des sentiments
		\begin{itemize}
			\item En général, les adjectifs et les adverbes sont des bons indicateurs
			\item \optword{à base de dictionnaire} : en utilisant un dictionnaire comme Wordnet pour enrichir la liste des indicateurs (synonymes, antonymes, etc.) 
			\item \optword{à base d'un corpus} : en utilisant la co-occurrence avec les mots  de la liste originale pour l'enrichir 
		\end{itemize}
		\item Assigner des scores à ces mots (ou des étiquettes)
		\item Calculer le score total à  base des mots
		\begin{itemize}
			\item \textbf{granularité} : par phrases ou par documents 
			\item il faut faire attention à  la négation
			\item on peut utiliser la structure; par exemple \keyword{RST}
			\item Ex. \expword{Donner un score de -1 aux mots de polarité négative et +1 aux mots de polarité positive et faire la somme des polarités des mots d'une phrase}
		\end{itemize}
	\end{itemize}
	
\end{frame}

\begin{frame}
	\frametitle{Quelques applications : Classification}
	\framesubtitle{Analyse des sentiments : par apprentissage automatique}
	
	\begin{itemize}
		\item \optword{En utilisant des caractéristiques }
		\begin{itemize}
			\item Présence d'un mot et sa fréquence 
			\item Catégorie grammaticale
			\item Les mots et les syntagmes d'opinion
			\item Négation
		\end{itemize}
		\item \optword{En utilisant des embeddings}
		\begin{itemize}
			\item Utiliser les embeddings des mots pour apprendre la classe de sortie
			\item En utilisant BERT (l'apprentissage par transfert)
			\begin{itemize}
				\item Utiliser la sortie \keyword{[CLS]} pour estimer les classes
				\item Utiliser \keyword{[CLS]} avec un réseaux feedForward pour estimer les classes
				\item Utiliser les derniers états cachés des mots avec des configurations comme CNN, RNN, etc.
			\end{itemize}
		\end{itemize}
	\end{itemize}

\end{frame}

\begin{frame}
	\frametitle{Quelques applications : Classification}
	\framesubtitle{Analyse des sentiments : hybride (exemple \cite{18-bettiche-al}) }
	
	\begin{figure}
		\centering
		\hgraphpage[.6\textwidth]{sent-bettiche-al.pdf}
		\caption{Architecture hybride proposée par \cite{18-bettiche-al} pour  en dialecte algérienne sur les réseaux sociaux}
	\end{figure}
	
\end{frame}

\begin{frame}
	\frametitle{Quelques applications : Classification}
	\framesubtitle{Analyse des sentiments : hybride (exemple \cite{18-bettiche-al} - suite) }
	
	\begin{itemize}
		\item Enrichissement du vocabulaire
		\begin{itemize}
			\item Détection des mots similaires (désignation d'un représentant)
			
			$Ratio = 1 - Levenstein(w_1, w_2)/(|w_1|+|w_2|)$
			
			Ex. \expword{Ratio(kolach, kollach) = 92\%, Ratio(kolach, khlasse) = 38\%.}
			
			\item Calculer l'orientation sémantique d'un mot $w$
			
			
			$SO(w) = \sum_{w_p \in V_p} PMI(w, w_p) - \sum_{w_n \in V_n} PMI(w, w_n)$
			
			$PMI (w_1, w_2) = \log \frac{p(w_1, w_2)}{p(w_1)*p(w_2)}$
		\end{itemize}
		\item Représentation de la polarité
		\begin{itemize}
			\item $polarite(w) = +1 \text{ SI } SO(w) > 0; -1 \text{ SI } SO(w) < 0$
			\item $polarite(w) = SO(w)$
		\end{itemize}
		\item Version avec règles : $polarite(msg) = Signe \sum_{w \in msg} polarite(w)$
		\item Version avec apprentissage automatique
		\begin{itemize}
			\item les polarités des mots comme caractéristiques
			\item ou la corrélation $PMI$ pour sélection des mots d'entrée
			\item plusieurs algorithmes sont utilisés : NB, arbres de décision, SVM, etc.
		\end{itemize}
	\end{itemize}
	
\end{frame}

\begin{frame}
	\frametitle{Quelques applications : Classification}
	\framesubtitle{Analyse des sentiments : hybride (exemple \cite{18-guellil-al}) }
	
	\begin{figure}
		\centering
		\hgraphpage[.45\textwidth]{sent-guellil-al.pdf}
		\caption{Architecture hybride proposée par \cite{18-guellil-al} pour l'analyse des sentiments des messages en arabizi sur les réseaux sociaux}
	\end{figure}
	
\end{frame}

\begin{frame}
	\frametitle{Quelques applications : Classification}
	\framesubtitle{Analyse des sentiments : Un peu d'humour}
	
	\begin{center}
		\vgraphpage{humour/humour-sentiment.jpg}
	\end{center}
	
\end{frame}

\subsection{Lisibilité}

\begin{frame}
	\frametitle{Quelques applications : Classification}
	\framesubtitle{Lisibilité}
	\hgraphpage{lisibilite-classif.pdf}
\end{frame}

\begin{frame}
	\frametitle{Quelques applications : Classification}
	\framesubtitle{Lisibilité : Formule (Flesch-Kincaid Grade Level)}
	\[
	206.835 - 1.015 (\frac{\text{\slshape mots totaux}}{\text{\slshape phrases totales}})
	- 84.6 (\frac{\text{\slshape syllabes totales}}{\text{\slshape mots totaux}})
	\]
	
	\begin{center}
			\rowcolors{2}{lightblue}{lightyellow}\footnotesize
		\begin{tabular}{p{.15\textwidth}lp{.5\textwidth}}
			\rowcolor{darkblue}
			\bfseries\textcolor{white}{Score} && \bfseries\textcolor{white}{Difficulté}\\
			90-100 && Très facile à lire (Élève de 11 ans). \\
			80-90 && Facile à lire. \\
			70-80 && Plutôt facile à lire.\\
			60-70 && En clair (Élève de 13 ou 15 ans). \\
			50-60 && Plutôt difficile à lire. \\
			30-50 && Difficile à lire (Université). \\
			0-30 && Très difficile à lire (Diplôme universitaire). \\
		\end{tabular}
	\end{center}
	
\end{frame}

\begin{frame}
	\frametitle{Quelques applications : Classification}
	\framesubtitle{Lisibilité : Formule (Dale–Chall readability formula)}
	\[
	0.1579 (\frac{\text{\slshape mots difficiles}}{\text{\slshape mots totaux}})
	+ 0.0496 (\frac{\text{\slshape mots totaux}}{\text{\slshape phrases totales}})
	\]
	
	\begin{center}
		\rowcolors{2}{lightblue}{lightyellow}\footnotesize
		\begin{tabular}{p{.15\textwidth}lp{.5\textwidth}}
			\rowcolor{darkblue}
			\bfseries\textcolor{white}{Score} && \bfseries\textcolor{white}{Difficulté}\\
			\textless= 4.9 && Étudiant du 4ième. \\
			5-5.9 && Étudiant du 5ième et 6ième. \\
			6-6.9 && Étudiant du 7ième et 8ième.\\
			7-7.9 && Étudiant du 9ième et 10ième. \\
			8-8.9 && Étudiant du 11ième et 12ième. \\
			9-9.9 && Étudiant du 13ième et 15ième (collège). \\
		\end{tabular}
	\end{center}
	
\end{frame}

\begin{frame}
	\frametitle{Quelques applications : Classification}
	\framesubtitle{Lisibilité : Apprentissage}
	
	\begin{figure}
		\centering
		\hgraphpage[0.8\textwidth]{lisibilite-ML.pdf}
		\caption{Pipeline d'estimation de difficulté de lecture avec apprentissage automatique \cite{2014-collins}}
	\end{figure}
	
\end{frame}

\begin{frame}
	\frametitle{Quelques applications : Classification}
	\framesubtitle{Analyse des sentiments : Un peu d'humour}
	
	\begin{center}
		\vgraphpage{humour/humour-readability1.jpg}
	\end{center}
	
\end{frame}

%===================================================================================
\section{Parole}
%===================================================================================

\begin{frame}
	\frametitle{Quelques applications}
	\framesubtitle{Parole}
	\begin{itemize}
		\item Reconnaissance de la parole
		\begin{itemize}
			\item \textbf{Entrée} : parole
			\item \textbf{Sortie} : texte
			\item \textbf{Anglais} : speech recognition
			\item \textbf{Anglais} : speech to text
		\end{itemize}
		\item Synthèse de la parole
		\begin{itemize}
			\item \textbf{Entrée} : texte
			\item \textbf{Sortie} : parole
			\item \textbf{Anglais} : speech synthesis
			\item \textbf{Anglais} : text to speech (TTS)
		\end{itemize}
	\end{itemize}
\end{frame}

\subsection{Reconnaissance de la parole}

\begin{frame}
	\frametitle{Quelques applications : Parole}
	\framesubtitle{Reconnaissance vocale : Classification}
	\hgraphpage{asr-classif.pdf}
\end{frame}


%\begin{frame}
%	\frametitle{Quelques applications : Parole}
%	\framesubtitle{Reconnaissance vocale : Architecture}
%	\begin{figure}
%		\centering
%		\hgraphpage[.45\textwidth]{asr-arch_.pdf}
%		\caption{Architecture d'un système de reconnaissance de paroles \cite{18-haridas}}
%	\end{figure}
%\end{frame}


%\begin{frame}
%	\frametitle{Quelques applications : Parole}
%	\framesubtitle{Reconnaissance vocale : Approches}
%	\begin{itemize}
%		\item \optword{Acoustique-Phonétique}
%		\begin{itemize}
%			\item Détecter les phonèmes et les transcrire
%		\end{itemize}
%		\item \optword{Reconnaissance des formes}
%		\begin{itemize}
%			\item Apprendre à détecter les différentes formes
%			\item \optword{Par modèles} 
%			\item \optword{Stochastique}
%		\end{itemize}
%		\item \optword{Intelligence artificielle}
%		\begin{itemize}
%			\item Fusion entre les deux approches précédentes
%		\end{itemize}
%	\end{itemize}
%\end{frame}


\begin{frame}
	\frametitle{Quelques applications : Parole}
	\framesubtitle{Reconnaissance vocale : Extraction des caractéristiques (Échantillonnage)}
	\begin{itemize}
		\item Transformer les signaux sonores vers une séquence des vecteurs de caractéristiques acoustiques
		\item Chaque vecteur représente l'information du signal encodée dans une petite fenêtre de temps
		\item \optword{Échantillonnage}
		\begin{itemize}
			\item Il faut prendre 2 échantillons par cycle (pour capturer les parties positive et négative du signal)
			\item La fréquence des paroles sont inférieures à 10 KHz (sampling rate = 20 KHz)
			\item Dans la téléphonie, la fréquence est 4KHz (sampling rate = 8 KHz)
		\end{itemize}
	\end{itemize}
\end{frame}

\begin{frame}
	\frametitle{Quelques applications : Parole}
	\framesubtitle{Reconnaissance vocale : Extraction des caractéristiques (Quantification)}
	\begin{itemize}
		\item \optword{Quantification}
		\begin{itemize}
			\item Les amplitudes des échantillons sont stockées sous forme des entiers
			\item Soit 8 bits (-128 à 127) ou 16 bits (-32768 à 32767)
			\item La valeur d'un échantillon dans un temps $n$ est représentée comme $x[n]$
		\end{itemize}
	\end{itemize}
\end{frame}

\begin{frame}
	\frametitle{Quelques applications : Parole}
	\framesubtitle{Reconnaissance vocale : Extraction des caractéristiques (Fenêtrage)}
	\begin{itemize}
		\item \optword{Fenêtrage}
		\begin{itemize}
			\item Utiliser une fenêtre pour capturer une partie d'un phonème 
			\item Cette partie est appelée Cadre (frame)
			\item L'opération a deux paramètres : taille de la fenêtre (Window size) et décalage (Frame stride, shift, offset)
		\end{itemize}
	\end{itemize}
	\begin{figure}
		\centering
		\hgraphpage[.5\textwidth]{ASR-windowing-exp.pdf}
		\caption{Exemple d'un fenêtrage \cite{2020-jurafsky-martin}}
	\end{figure}
\end{frame}

\begin{frame}
	\frametitle{Quelques applications : Parole}
	\framesubtitle{Reconnaissance vocale : Extraction des caractéristiques (Fenêtrage - suite)}
	\begin{itemize}
		\item \optword{Fenêtrage} (suite)
		\begin{itemize}
			\item Pour extraire un cadre $y[n]$, on utilise un signal de la fenêtre $w[n]$ sur le signal initial $s[n]$
			\[y[n] = w[n] s[n]\]
			\item La plus simple est la fenêtre rectangulaire 
			
			$w[n] = \begin{cases}
			1 & \text{si } 0 \le n \le L-1 \\
			0 & \text{sinon }\\
			\end{cases}$
			\item Ceci va créer des problèmes avec l'analyse de Fourier 
			\item La solution est la fenêtre de Hamming
			
			$w[n] = \begin{cases}
			0.54 - 0.46 \cos (\frac{2\pi n}{L}) & \text{si } 0 \le n \le L-1 \\
			0 & \text{sinon }\\
			\end{cases}$
		\end{itemize}
	\end{itemize}
\end{frame}

\begin{frame}
	\frametitle{Quelques applications : Parole}
	\framesubtitle{Reconnaissance vocale : Extraction des caractéristiques (Fenêtrage - exemple)}
	
	\begin{figure}
		\centering
		\hgraphpage[.6\textwidth]{ASR-windowing2-exp.pdf}
		\caption{Exemple d'un fenêtrage \cite{2020-jurafsky-martin}}
	\end{figure}
	
\end{frame}

\begin{frame}
	\frametitle{Quelques applications : Parole}
	\framesubtitle{Reconnaissance vocale : Extraction des caractéristiques (DFT)}
	\begin{itemize}
		\item \optword{Discrete Fourier Transform}
		\begin{itemize}
			\item Combien d'énergie un signal contient dans les différentes bandes de fréquence
			\[X[k] = \sum\limits_{n=0}^{N-1} x[n] e^{-j\frac{2\pi}{N} k n}\]
		\end{itemize}
	\end{itemize}
	\begin{figure}
		\centering
		\hgraphpage[.5\textwidth]{ASR-DFT-exp.pdf}
		\caption{(a) une portion du signal du voyelle [iy] (b) son DFT \cite{2020-jurafsky-martin}}
	\end{figure}
\end{frame}

\begin{frame}
	\frametitle{Quelques applications : Parole}
	\framesubtitle{Reconnaissance vocale : Extraction des caractéristiques (Filtre de Mel)}
	\begin{itemize}
		\item \optword{Filtre de Mel}
		\begin{itemize}
			\item L'audition humaine n'est pas également sensible à toutes les bandes de fréquences
			\item Collecter les énergies à chaque fréquence en se basant sur l'échelle de Mel
			\[mel(f) = 1127 \ln (1 + \frac{f}{700})\]
		\end{itemize}
	\end{itemize}
	\begin{figure}
		\centering
		\hgraphpage[.5\textwidth]{ASR-mel-exp.pdf}
		\caption{Un exemple du filtre de Mel \cite{2020-jurafsky-martin}}
	\end{figure}
\end{frame}

\begin{frame}
	\frametitle{Quelques applications : Parole}
	\framesubtitle{Reconnaissance vocale : Reconnaissance}
	\begin{figure}
		\centering
		\hgraphpage[.75\textwidth]{ASR-rec-exp.pdf}
		\caption{Exemple de reconnaissance en utilisant un encodeur-décodeur \cite{2020-jurafsky-martin}}
	\end{figure}
\end{frame}

\begin{frame}
	\frametitle{Quelques applications : Parole}
	\framesubtitle{Reconnaissance vocale : Reconnaissance - suite}
	\begin{itemize}
		\item $p(y_1, \ldots, y_n) = \prod\limits_{i=1}^n p(y_i| y_1, \ldots, y_{i-1}, X)$
		
		\item $\hat{y}_i = \arg\max_{c \in V} p(c| y_1, \ldots, y_{i-1}, X)$
		
		\item clairement, ceci est un modèle de langue
		\item mais, il se peut que les données ne sont pas suffisants pour apprendre un bon modèle de langue
		\item Solution : intégrer un modèle de langue séparé
		
		\item $score(Y|X) = \frac{1}{|Y|_{car}} \log p(Y|X) + \lambda \log p_{LM}(Y)$
	\end{itemize}
\end{frame}

\begin{frame}
	\frametitle{Quelques applications : Parole}
	\framesubtitle{Reconnaissance vocale : Un peu d'humour}
		\begin{center}
			\vgraphpage{humour/humour-ASR.jpg}
		\end{center}
\end{frame}

\subsection{Synthèse de la parole}

\begin{frame}
	\frametitle{Quelques applications : Parole}
	\framesubtitle{Synthèse vocale}
	\hgraphpage{tts-classif.pdf}
\end{frame}

\begin{frame}
	\frametitle{Quelques applications : Parole}
	\framesubtitle{Synthèse vocale : Architecture}
	\begin{figure}
		\centering
		\hgraphpage[.5\textwidth]{tts-arch.pdf}
		\caption{Architecture d'un système de synthèse vocale \cite{2017-Hinterleitner}}
	\end{figure}
\end{frame}


\begin{frame}
	\frametitle{Quelques applications : Parole}
	\framesubtitle{Synthèse vocale : Un peu d'humour}
	\begin{center}
		\vgraphpage{humour/humour-TTS.jpg}
	\end{center}
\end{frame}


\insertbibliography{TALN10}{*}

\end{document}

