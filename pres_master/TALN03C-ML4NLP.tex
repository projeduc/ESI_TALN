% !TEX TS-program = pdflatex
% !TeX program = pdflatex
% !TEX encoding = UTF-8
% !TEX spellcheck = fr

\documentclass[xcolor=table]{beamer}


%\usepackage{fullpage}
%\usepackage[left=2.8cm,right=2.2cm,top=2 cm,bottom=2 cm]{geometry}
\setbeamersize{text margin left=10pt,text margin right=10pt}
\usepackage{amsmath,amssymb} 
\usepackage[T1]{fontenc}
\usepackage{textcomp}
\usepackage[utf8]{inputenc}
\usepackage[french]{babel}
\usepackage{arabtex}
\usepackage{txfonts}
\usepackage[]{graphicx}
\usepackage{multirow}
\usepackage{hyperref}
\usepackage{colortbl}
\usepackage{listingsutf8}
\usepackage{wrapfig}
\usepackage{multicol}
\usepackage[export]{adjustbox} %for images in table, also for frame
\usepackage[many]{tcolorbox}

\hypersetup{
	colorlinks,
	urlcolor = blue
}

%\renewcommand{\baselinestretch}{1.5}

\def\supit#1{\raisebox{0.8ex}{\small\it #1}\hspace{0.05em}}

\AtBeginSection{%
	\begin{frame}
		\sectionpage
	\end{frame}
}

\newcommand{\rottext}[2]{%
	\rotatebox{90}{%
	\begin{minipage}{#1}%
		\raggedleft#2%
	\end{minipage}%
	}%
}

\usepackage{longtable}
\usepackage{tabu}


\institute{ %
École  nationale Supérieure d'Informatique (ESI, ex. INI), Algérie
}
\author[ \textbf{\footnotesize\insertframenumber/\inserttotalframenumber} \hspace*{1.5cm} ESI - ARIES Abdelkrime (2020/2021)] %
{ARIES Abdelkrime}
%\titlegraphic{\includegraphics[height=1cm]{../img/esi-logo.png}%\hspace*{4.75cm}~


\date{Année universitaire : 2020/2021} %\today

\usetheme{Warsaw} % Antibes Boadilla Warsaw

\beamertemplatenavigationsymbolsempty

%\setbeamertemplate{headline}{}

\definecolor{lightblue}{HTML}{D0D2FF}
\definecolor{lightyellow}{HTML}{FFFFAA}
\definecolor{darkblue}{HTML}{0000BB}
\definecolor{olivegreen}{HTML}{006600}
\definecolor{violet}{HTML}{6600CC}

\newcommand{\keyword}[1]{\textcolor{red}{\bfseries\itshape #1}}
\newcommand{\expword}[1]{\textcolor{olivegreen}{#1}}
\newcommand{\optword}[1]{\textcolor{violet}{\bfseries #1}}

\makeatletter
\newcommand\mysphere{%
	\parbox[t]{10pt}{\raisebox{0.2pt}{\beamer@usesphere{item projected}{bigsphere}}}}
\makeatother

%\let\oldtabular\tabular
%\let\endoldtabular\endtabular
%\renewenvironment{tabular}{\rowcolors{2}{white}{lightblue}\oldtabular\rowcolor{blue}}{\endoldtabular}


\NoAutoSpacing %french autospacing after ":"

\def\graphpath{}

\newcommand{\changegraphpath}[1]{\def\graphpath{#1}}


\newcommand{\vgraphpage}[2][.84\textheight]{%
%	\begin{center}%
		\includegraphics[height=#1]{\graphpath #2}%
%	\end{center}%
}

\newcommand{\hgraphpage}[2][\textwidth]{%
%	\begin{center}%
		\includegraphics[width=#1]{\graphpath #2}%
%	\end{center}%
}

\newcommand{\graphpage}[2][]{%
	\includegraphics[#1]{\graphpath #2}%
}

\bibliographystyle{acm}

\newcommand{\insertbibliography}[2]{
	\appendix
	\section*{Bibliographie}
	\nocite{#2}
%	\makeatletter % to change template
%	\setbeamertemplate{headline}[default] % not mandatory, but I though it was better to set it blank
%	\def\beamer@entrycode{\vspace*{-\headheight}} % here is the part we are interested in :)
%	\makeatother
	\begin{multicols*}{2}[\frametitle{\insertsection} \usebeamertemplate*{frametitle}]%\usebeamertemplate*{frametitle}\frametitle{Références}
		\tiny
		\bibliography{#1}
	\end{multicols*}
}

\definecolor{my-grey}{RGB}{233, 233, 233}

\newcommand{\insertlicence}{
	\begin{frame}[plain]
	\frametitle{Licence : CC-BY 4.0}
%	\framesubtitle{Licence: CC-BY-NC 4.0}

	\begin{tcolorbox}[colback=cyan,
		colframe=cyan,  
		arc=0pt,outer arc=0pt,
		valign=top, 
		halign=center,
		width=\textwidth]
		
		\includegraphics[width=.5cm]{../img/licence/cc_icon_white_x2.png}
		\includegraphics[width=.5cm]{../img/licence/attribution_icon_white_x2.png}
		
		\color{white}
		\bfseries Attribution 4.0 International (CC BY 4.0) \\
		\tiny \url{https://creativecommons.org/licenses/by/4.0/deed.fr}
		
	\end{tcolorbox}\vspace{-.5cm}
	\begin{tcolorbox}[colback=my-grey,
		colframe=my-grey,  
		center, arc=0pt,outer arc=0pt,
		valign=top, 
		halign=left,
		width=\textwidth]
		
		\tiny
		
		\begin{center}
			\bfseries\Large
			Vous êtes autorisé à :
		\end{center}
		
		\begin{minipage}{0.83\textwidth}
			\begin{itemize}
				\item[] \textbf{Partager} — copier, distribuer et communiquer le matériel par tous moyens et sous tous formats
				\item[] \textbf{Adapter} — remixer, transformer et créer à partir du matériel
				pour toute utilisation, y compris commerciale.
			\end{itemize}
		\end{minipage}
		\begin{minipage}{0.15\textwidth}
			\includegraphics[width=\textwidth]{../img/licence/FreeCulturalWorks_seal_x2.jpg}
		\end{minipage}
	
		
		\begin{center}
			\bfseries\Large
			Selon les conditions suivantes :
		\end{center}
		
		\begin{itemize}
			\item[] \textbf{Attribution} — Vous devez créditer l'Œuvre, intégrer un lien vers la licence et indiquer si des modifications ont été effectuées à l'Oeuvre. Vous devez indiquer ces informations par tous les moyens raisonnables, sans toutefois suggérer que l'Offrant vous soutient ou soutient la façon dont vous avez utilisé son Oeuvre. 
			\item[] \textbf{Pas de restrictions complémentaires} — Vous n'êtes pas autorisé à appliquer des conditions légales ou des mesures techniques qui restreindraient légalement autrui à utiliser l'Oeuvre dans les conditions décrites par la licence.
		\end{itemize}
		
	\end{tcolorbox}
	
%	\begin{center}
%		\bfseries Attribution 4.0 International (CC BY 4.0)
%		\url{https://creativecommons.org/licenses/by/4.0/deed.fr}
%	\end{center}

%	\tiny
%
%	Vous êtes autorisé à : 
%	\begin{itemize}
%		\item \textbf{Partager} — copier, distribuer et communiquer le matériel par tous moyens et sous tous formats
%		\item \textbf{Adapter} — remixer, transformer et créer à partir du matériel
%	\end{itemize}
%	
%	Selon les conditions suivantes : 
%	\begin{itemize}
%		\item \textbf{Attribution} — Vous devez créditer l'Œuvre, intégrer un lien vers la licence et indiquer si des modifications ont été effectuées à l'Oeuvre. Vous devez indiquer ces informations par tous les moyens raisonnables, sans toutefois suggérer que l'Offrant vous soutient ou soutient la façon dont vous avez utilisé son Oeuvre.
%		\item \textbf{Pas d'Utilisation Commerciale} — Vous n'êtes pas autorisé à faire un usage commercial de cette Oeuvre, tout ou partie du matériel la composant. 
%		\item \textbf{Pas de restrictions complémentaires} — Vous n'êtes pas autorisé à appliquer des conditions légales ou des mesures techniques qui restreindraient légalement autrui à utiliser l'Oeuvre dans les conditions décrites par la licence.
%	\end{itemize}

	\end{frame}
}

\settowidth{\leftmargini}{\usebeamertemplate{itemize item}}
\addtolength{\leftmargini}{\labelsep}

\AtBeginDocument{
	\newcolumntype{L}[2]{>{\vbox to #2\bgroup\vfill\flushleft}p{#1}<{\egroup}} 
	
	\begin{frame}[plain]
		\maketitle
	\end{frame}

	\insertlicence
}


% needs etoolbox; to break links after -
\appto\UrlBreaks{\do\-}


\lstdefinelanguage{CSS}{
	keywords={accelerator,azimuth,background,background-attachment,
		background-color,background-image,background-position,
		background-position-x,background-position-y,background-repeat,
		behavior,border,border-bottom,border-bottom-color,
		border-bottom-style,border-bottom-width,border-collapse,
		border-color,border-left,border-left-color,border-left-style,
		border-left-width,border-right,border-right-color,
		border-right-style,border-right-width,border-spacing,
		border-style,border-top,border-top-color,border-top-style,
		border-top-width,border-width,bottom,caption-side,clear,
		clip,color,content,counter-increment,counter-reset,cue,
		cue-after,cue-before,cursor,direction,display,elevation,
		empty-cells,filter,float,font,font-family,font-size,
		font-size-adjust,font-stretch,font-style,font-variant,
		font-weight,height,ime-mode,include-source,
		layer-background-color,layer-background-image,layout-flow,
		layout-grid,layout-grid-char,layout-grid-char-spacing,
		layout-grid-line,layout-grid-mode,layout-grid-type,left,
		letter-spacing,line-break,line-height,list-style,
		list-style-image,list-style-position,list-style-type,margin,
		margin-bottom,margin-left,margin-right,margin-top,
		marker-offset,marks,max-height,max-width,min-height,
		min-width,-moz-binding,-moz-border-radius,
		-moz-border-radius-topleft,-moz-border-radius-topright,
		-moz-border-radius-bottomright,-moz-border-radius-bottomleft,
		-moz-border-top-colors,-moz-border-right-colors,
		-moz-border-bottom-colors,-moz-border-left-colors,-moz-opacity,
		-moz-outline,-moz-outline-color,-moz-outline-style,
		-moz-outline-width,-moz-user-focus,-moz-user-input,
		-moz-user-modify,-moz-user-select,orphans,outline,
		outline-color,outline-style,outline-width,overflow,
		overflow-X,overflow-Y,padding,padding-bottom,padding-left,
		padding-right,padding-top,page,page-break-after,
		page-break-before,page-break-inside,pause,pause-after,
		pause-before,pitch,pitch-range,play-during,position,quotes,
		-replace,richness,right,ruby-align,ruby-overhang,
		ruby-position,-set-link-source,size,speak,speak-header,
		speak-numeral,speak-punctuation,speech-rate,stress,
		scrollbar-arrow-color,scrollbar-base-color,
		scrollbar-dark-shadow-color,scrollbar-face-color,
		scrollbar-highlight-color,scrollbar-shadow-color,
		scrollbar-3d-light-color,scrollbar-track-color,table-layout,
		text-align,text-align-last,text-decoration,text-indent,
		text-justify,text-overflow,text-shadow,text-transform,
		text-autospace,text-kashida-space,text-underline-position,top,
		unicode-bidi,-use-link-source,vertical-align,visibility,
		voice-family,volume,white-space,widows,width,word-break,
		word-spacing,word-wrap,writing-mode,z-index,zoom},  
	sensitive=true,
	morecomment=[l]{//},
	morecomment=[s]{/*}{*/},
	morestring=[b]',
	morestring=[b]",
	alsoletter={:},
	alsodigit={-}
}
\lstdefinelanguage{HTML5}{
	language=html,
	sensitive=true, 
	alsoletter={<>=-},
	otherkeywords={
		% HTML tags
		<, </, >,
		</a, <a, </a>,
		</abbr, <abbr, </abbr>,
		</address, <address, </address>,
		</area, <area, </area>,
		</area, <area, </area>,
		</article, <article, </article>,
		</aside, <aside, </aside>,
		</audio, <audio, </audio>,
		</audio, <audio, </audio>,
		</b, <b, </b>,
		</base, <base, </base>,
		</bdi, <bdi, </bdi>,
		</bdo, <bdo, </bdo>,
		</blockquote, <blockquote, </blockquote>,
		</body, <body, </body>,
		</br, <br, </br>,
		</button, <button, </button>,
		</canvas, <canvas, </canvas>,
		</caption, <caption, </caption>,
		</cite, <cite, </cite>,
		</code, <code, </code>,
		</col, <col, </col>,
		</colgroup, <colgroup, </colgroup>,
		</data, <data, </data>,
		</datalist, <datalist, </datalist>,
		</dd, <dd, </dd>,
		</del, <del, </del>,
		</details, <details, </details>,
		</dfn, <dfn, </dfn>,
		</div, <div, </div>,
		</dl, <dl, </dl>,
		</dt, <dt, </dt>,
		</em, <em, </em>,
		</embed, <embed, </embed>,
		</fieldset, <fieldset, </fieldset>,
		</figcaption, <figcaption, </figcaption>,
		</figure, <figure, </figure>,
		</footer, <footer, </footer>,
		</form, <form, </form>,
		</h1, <h1, </h1>,
		</h2, <h2, </h2>,
		</h3, <h3, </h3>,
		</h4, <h4, </h4>,
		</h5, <h5, </h5>,
		</h6, <h6, </h6>,
		</head, <head, </head>,
		</header, <header, </header>,
		</hr, <hr, </hr>,
		</html, <html, </html>,
		</i, <i, </i>,
		</iframe, <iframe, </iframe>,
		</img, <img, </img>,
		</input, <input, </input>,
		</ins, <ins, </ins>,
		</kbd, <kbd, </kbd>,
		</keygen, <keygen, </keygen>,
		</label, <label, </label>,
		</legend, <legend, </legend>,
		</li, <li, </li>,
		</link, <link, </link>,
		</main, <main, </main>,
		</map, <map, </map>,
		</mark, <mark, </mark>,
		</math, <math, </math>,
		</menu, <menu, </menu>,
		</menuitem, <menuitem, </menuitem>,
		</meta, <meta, </meta>,
		</meter, <meter, </meter>,
		</nav, <nav, </nav>,
		</noscript, <noscript, </noscript>,
		</object, <object, </object>,
		</ol, <ol, </ol>,
		</optgroup, <optgroup, </optgroup>,
		</option, <option, </option>,
		</output, <output, </output>,
		</p, <p, </p>,
		</param, <param, </param>,
		</pre, <pre, </pre>,
		</progress, <progress, </progress>,
		</q, <q, </q>,
		</rp, <rp, </rp>,
		</rt, <rt, </rt>,
		</ruby, <ruby, </ruby>,
		</s, <s, </s>,
		</samp, <samp, </samp>,
		</script, <script, </script>,
		</section, <section, </section>,
		</select, <select, </select>,
		</small, <small, </small>,
		</source, <source, </source>,
		</span, <span, </span>,
		</strong, <strong, </strong>,
		</style, <style, </style>,
		</summary, <summary, </summary>,
		</sup, <sup, </sup>,
		</svg, <svg, </svg>,
		</table, <table, </table>,
		</tbody, <tbody, </tbody>,
		</td, <td, </td>,
		</template, <template, </template>,
		</textarea, <textarea, </textarea>,
		</tfoot, <tfoot, </tfoot>,
		</th, <th, </th>,
		</thead, <thead, </thead>,
		</time, <time, </time>,
		</title, <title, </title>,
		</tr, <tr, </tr>,
		</track, <track, </track>,
		</u, <u, </u>,
		</ul, <ul, </ul>,
		</var, <var, </var>,
		</video, <video, </video>,
		</wbr, <wbr, </wbr>,
		/>, <!
	},  
	ndkeywords={
		% General
		=,
		% HTML attributes
		accept=, accept-charset=, accesskey=, action=, align=, alt=, async=, autocomplete=, autofocus=, autoplay=, autosave=, bgcolor=, border=, buffered=, challenge=, charset=, checked=, cite=, class=, code=, codebase=, color=, cols=, colspan=, content=, contenteditable=, contextmenu=, controls=, coords=, data=, datetime=, default=, defer=, dir=, dirname=, disabled=, download=, draggable=, dropzone=, enctype=, for=, form=, formaction=, headers=, height=, hidden=, high=, href=, hreflang=, http-equiv=, icon=, id=, ismap=, itemprop=, keytype=, kind=, label=, lang=, language=, list=, loop=, low=, manifest=, max=, maxlength=, media=, method=, min=, multiple=, name=, novalidate=, open=, optimum=, pattern=, ping=, placeholder=, poster=, preload=, pubdate=, radiogroup=, readonly=, rel=, required=, reversed=, rows=, rowspan=, sandbox=, scope=, scoped=, seamless=, selected=, shape=, size=, sizes=, span=, spellcheck=, src=, srcdoc=, srclang=, start=, step=, style=, summary=, tabindex=, target=, title=, type=, usemap=, value=, width=, wrap=,
		% CSS properties
		accelerator:,azimuth:,background:,background-attachment:,
		background-color:,background-image:,background-position:,
		background-position-x:,background-position-y:,background-repeat:,
		behavior:,border:,border-bottom:,border-bottom-color:,
		border-bottom-style:,border-bottom-width:,border-collapse:,
		border-color:,border-left:,border-left-color:,border-left-style:,
		border-left-width:,border-right:,border-right-color:,
		border-right-style:,border-right-width:,border-spacing:,
		border-style:,border-top:,border-top-color:,border-top-style:,
		border-top-width:,border-width:,bottom:,caption-side:,clear:,
		clip:,color:,content:,counter-increment:,counter-reset:,cue:,
		cue-after:,cue-before:,cursor:,direction:,display:,elevation:,
		empty-cells:,filter:,float:,font:,font-family:,font-size:,
		font-size-adjust:,font-stretch:,font-style:,font-variant:,
		font-weight:,height:,ime-mode:,include-source:,
		layer-background-color:,layer-background-image:,layout-flow:,
		layout-grid:,layout-grid-char:,layout-grid-char-spacing:,
		layout-grid-line:,layout-grid-mode:,layout-grid-type:,left:,
		letter-spacing:,line-break:,line-height:,list-style:,
		list-style-image:,list-style-position:,list-style-type:,margin:,
		margin-bottom:,margin-left:,margin-right:,margin-top:,
		marker-offset:,marks:,max-height:,max-width:,min-height:,
		min-width:,transition-duration:,transition-property:,
		transition-timing-function:,transform:,
		-moz-transform:,-moz-binding:,-moz-border-radius:,
		-moz-border-radius-topleft:,-moz-border-radius-topright:,
		-moz-border-radius-bottomright:,-moz-border-radius-bottomleft:,
		-moz-border-top-colors:,-moz-border-right-colors:,
		-moz-border-bottom-colors:,-moz-border-left-colors:,-moz-opacity:,
		-moz-outline:,-moz-outline-color:,-moz-outline-style:,
		-moz-outline-width:,-moz-user-focus:,-moz-user-input:,
		-moz-user-modify:,-moz-user-select:,orphans:,outline:,
		outline-color:,outline-style:,outline-width:,overflow:,
		overflow-X:,overflow-Y:,padding:,padding-bottom:,padding-left:,
		padding-right:,padding-top:,page:,page-break-after:,
		page-break-before:,page-break-inside:,pause:,pause-after:,
		pause-before:,pitch:,pitch-range:,play-during:,position:,quotes:,
		-replace:,richness:,right:,ruby-align:,ruby-overhang:,
		ruby-position:,-set-link-source:,size:,speak:,speak-header:,
		speak-numeral:,speak-punctuation:,speech-rate:,stress:,
		scrollbar-arrow-color:,scrollbar-base-color:,
		scrollbar-dark-shadow-color:,scrollbar-face-color:,
		scrollbar-highlight-color:,scrollbar-shadow-color:,
		scrollbar-3d-light-color:,scrollbar-track-color:,table-layout:,
		text-align:,text-align-last:,text-decoration:,text-indent:,
		text-justify:,text-overflow:,text-shadow:,text-transform:,
		text-autospace:,text-kashida-space:,text-underline-position:,top:,
		unicode-bidi:,-use-link-source:,vertical-align:,visibility:,
		voice-family:,volume:,white-space:,widows:,width:,word-break:,
		word-spacing:,word-wrap:,writing-mode:,z-index:,zoom:
	},  
	morecomment=[s]{<!--}{-->},
	tag=[s]
}

%\usepackage{color}
%\definecolor{editorGray}{rgb}{0.95, 0.95, 0.95}
%\definecolor{editorOcher}{rgb}{1, 0.5, 0} % #FF7F00 -> rgb(239, 169, 0)
%\definecolor{editorGreen}{rgb}{0, 0.5, 0} % #007C00 -> rgb(0, 124, 0)
%
%\lstset{%
%	% Basic design
%	backgroundcolor=\color{editorGray},
%	basicstyle={\small\ttfamily},   
%	frame=l,
%	% Line numbers
%	xleftmargin={0.75cm},
%	numbers=left,
%	stepnumber=1,
%	firstnumber=1,
%	numberfirstline=true,
%	% Code design   
%	keywordstyle=\color{blue}\bfseries,
%	commentstyle=\color{darkgray}\ttfamily,
%	ndkeywordstyle=\color{editorGreen}\bfseries,
%	stringstyle=\color{editorOcher},
%	% Code
%	language=HTML5,
%	alsodigit={.:;},
%	tabsize=2,
%	showtabs=false,
%	showspaces=false,
%	showstringspaces=false,
%	extendedchars=true,
%	breaklines=true,        
%}

\lstset{language=CSS,
	basicstyle=\ttfamily,
	keywordstyle=\color{blue}\ttfamily,
	stringstyle=\color{red}\ttfamily,
	commentstyle=\color{green}\ttfamily,
	morecomment=[l][\color{magenta}]{\#}
}

\title[TALN : 03- ML pour TALN]%
{Traitement automatique du langage naturel\\Chapitre 03 : Apprentissage automatique pour TALN} 

\changegraphpath{../img/ml4nlp/}

\begin{document}
	
\begin{frame}
\frametitle{Traitement automatique du langage naturel}
\framesubtitle{ML pour TALN : Introduction}


\end{frame}

\begin{frame}
	\frametitle{Traitement automatique du langage naturel}
	\framesubtitle{ML pour TALN : Algorithmes}
	
	\begin{itemize}
		\item \optword{NB} (Naive Bayes)
		\item \optword{LR} (Logistic regression) : Régression logique
		\item \optword{SVM} (Support-Vector Machine) : Machines à vecteurs de support 
		\item \optword{NN} (Neural Network) : Réseau de neurones
		\item \optword{FFN} (Feed Forward Network) : Réseau à propagation avant
		\begin{itemize}
			\item \optword{MLP} (Multi-Layer Perceptron) : Perceptron multicouche
			\item \optword{CNN} (Convolutional Neural Network) : Réseau de neurones convolutif
		\end{itemize}
		\item \optword{RNN} (Recurrent Neural Network) : Réseau de neurones récurrents
		\item \optword{HMM} (Hidden Markov Model) : Modèle de Markov caché
		\item \optword{MEMM} (Maximum-Entropy Markov Model) : Modèle de Markov à entropie maximale
		\item \optword{CRF} (Conditional Random Field) : Champ aléatoire conditionnel
	\end{itemize}
	
\end{frame}

\begin{frame}
	\frametitle{Traitement automatique du langage naturel}
	\framesubtitle{ML pour TALN : Types des modèles}
	
	\scriptsize
	\begin{tblr}{
			colspec = {p{.12\textwidth}lp{.37\textwidth}lp{.38\textwidth}},
			row{odd} = {lightblue},
			row{1} = {darkblue, fg=white, font=\bfseries, valign=m, halign=c},
			column{2,4}={white},
			column{1}={bg=darkblue, fg=white, font=\bfseries, valign=m},
			row{even} = {white},
			cell{1}{1}={white},
			colsep=3pt,
			rowsep=1pt,
		}
		
		&& Génératif && Discriminatif \\
		
		&&&&\\
		
		Texte && NB && LR, SVM, MLP, RNN, CNN \\
		
		&&&&\\
		
		Séquence && HMM  && MEMM, CRF, MLP \\
		
	\end{tblr}

	\vfill
	
	\begin{itemize}
		\item Modèle génératif : un modèle qui apprend à générer des caractéristiques étant donnée la classe
		
		$\hat{Y} = \arg\max_k P(Y_k) P(X | Y_k)$
		
		\item Modèle discriminatif : un modèle qui apprend à estimer la classe directement 
		
		$\hat{Y} = \arg\max_k P(Y_k | X)$
		
		
	\end{itemize}
	
\end{frame}


\begin{frame}
\frametitle{Traitement automatique du langage naturel}
\framesubtitle{ML pour TALN : Plan}

\begin{multicols}{2}
%	\small
\tableofcontents
\end{multicols}
\end{frame}

%===================================================================================
\section{Apprentissage automatique}
%===================================================================================

\begin{frame}
\frametitle{ML pour TALN}
\framesubtitle{\insertsection}

\end{frame}

\subsection{ML traditionnel}

\begin{frame}
\frametitle{ML pour TALN : \insertsection}
\framesubtitle{\insertsubsection\ : Naive Bayes}

Classement 
\[\hat{Y} = \arg\max_{Y_k} \log P(Y_k) + \sum_{j=1}^{N} \log P(X_j|Y_k)\]

Apprentissage
\[P(Y_k) = \frac{|\{y / y \in Y \wedge y = k\}|}{M}\]

\begin{itemize}
	\item $|\{y / y \in Y \wedge y = k\}|$ est le nombre des échantillons appartenant à la classe $k$ dans le dataset d'entraînement
	\item $M$ est le nombre des échantillons dans le dataset d'entraînement
	\item $P(X_j|Y_k)$ est calculée selon une des trois lois : lois multinomiale, lois normale et lois de Bernoulli
\end{itemize}

\end{frame}

\begin{frame}
	\frametitle{ML pour TALN : \insertsection}
	\framesubtitle{\insertsubsection\ : Régression logistique}
	
	Classement 
	\[\hat{Y} = \arg\max_{k} Softmax(\sum_{j=1}^{N} \theta_j X_j + \theta_0)\]
	
	Apprentissage
	\[J_\theta = \frac{-1}{M} \sum\limits_{i=1}^{M} \sum_{k=1}^{L} Y^{(i)}_k \log(H^{(i)}_k)\]
	\[\theta_j = \theta_j - \alpha \frac{\partial J_\theta}{\partial \theta_j}\]
	
	\begin{itemize}
		\item $\theta$ est mis à jours avec un algorithme d'optimisation comme la \keyword{descente du gradient}
		\item $\theta[N, K]$ est une matrice de $N$ caractéristiques et $K$ classes
	\end{itemize}
	
\end{frame}

\begin{frame}
	\frametitle{ML pour TALN : \insertsection}
	\framesubtitle{\insertsubsection\ : Machine à vecteur de support}
	
	Classement 
	\[\hat{Y_t} = \arg\max_{k} \sum^M_{i=1} \alpha^k_i y^{(i)} K(x^{(i)}, x_t) - b^k\]
	
	Apprentissage
	\[J_\alpha = \sum\limits_{i=1}^{M} \alpha_i - \frac{1}{2} \sum\limits_{i=1}^{M} \sum\limits_{j=1}^{M} \alpha_i \alpha_j y^{(i)} y^{(j)} K(x^{(i)}, x^{(j)}) \]
	
	\begin{itemize}
		\item $\alpha$ est mis à jours avec un algorithme d'optimisation comme la \keyword{Sequential minimal optimization}
	\end{itemize}
	
\end{frame}

\subsection{MLP}

\begin{frame}
	\frametitle{ML pour TALN : \insertsection}
	\framesubtitle{\insertsubsection}
	
	\vgraphpage{RN.pdf}
	
\end{frame}

\begin{frame}
	\frametitle{ML pour TALN : \insertsection}
	\framesubtitle{\insertsubsection\ : Architecture}
	
	\vgraphpage{RNPA.pdf}
	
\end{frame}


\subsection{CNN}

\begin{frame}
	\frametitle{ML pour TALN : \insertsection}
	\framesubtitle{\insertsubsection : traitement d'image traditionnel}
	
	\begin{center}
		\hgraphpage[\textwidth]{img-learn.pdf}
	\end{center}
	
\end{frame}

\begin{frame}
	\frametitle{ML pour TALN : \insertsection}
	\framesubtitle{\insertsubsection\ : convolution (principe)}
	
	\begin{itemize}
		\item Une image est une matrice des pixels
		\item Convolution : modifier la valeur d'un pixel par rapport à ces voisins
		\item deux paramètres : \optword{padding} (entourer l'image avec des 0 afin de préserver la taille originale), \optword{stride} (le pas de défilement du noyau/masque)
	\end{itemize}
	
	\begin{center}
		\hgraphpage[.8\textwidth]{conv.pdf}
	\end{center}
	
\end{frame}

\begin{frame}
	\frametitle{ML pour TALN : \insertsection}
	\framesubtitle{\insertsubsection\ : Conv2D}
	
	\begin{minipage}{0.60\textwidth} 
		\begin{itemize}
			\item On préserve la structure spatiale des images
			\item La couche apprend un ou plusieurs noyaux
			\item $ w' = \frac{w - w_f + 2P}{S} + 1$,  $ h' = \frac{h - h_f + 2P}{S} + 1$
			\item On peut spécifier le nombres des filtres $k$ à générer 
			\item Le nombre des paramètres sera $w_f * h_f * c * k$ plus $k$ biais
			\item Exemple, \expword{image : 32x32x3; noyau : 5x5; s : 1; p: 0; k : 6.} Dans ce cas, on aura \expword{81} paramètres à entraîner
		\end{itemize}
	\end{minipage}
	%
	\begin{minipage}{0.39\textwidth}
		\hgraphpage[\textwidth]{conv2d.pdf}
	\end{minipage}
	
\end{frame}

\begin{frame}
	\frametitle{ML pour TALN : \insertsection}
	\framesubtitle{\insertsubsection\ : Pooling}
	
	\begin{minipage}{0.60\textwidth} 
		\begin{itemize}
			\item Rendre la représentation plus petite et plus gérable
			\item $ w' = \frac{w - w_f + 2P}{S} + 1$,  $ h' = \frac{h - h_f + 2P}{S} + 1$
			\item Pas de paramètres 
			\item \optword{Max Pool} : le gradient est passé seulement à la cellule gagnante (qui a le max) 
			\item \optword{Average Pool} : le gradient passé aux cellules participantes à la moyennes est $\frac{\text{gradient}}{w_f * h_f}$ 
		\end{itemize}
	\end{minipage}
	%
	\begin{minipage}{0.39\textwidth}
		\hgraphpage[\textwidth]{maxpool.pdf}
	\end{minipage}
	
\end{frame}

\subsection{RNN}

\begin{frame}
	\frametitle{ML pour TALN : \insertsection}
	\framesubtitle{\insertsubsection\ : Architecture}
	
	\begin{minipage}{0.49\textwidth} 
		\begin{itemize}
			\item \optword{Elman network}
			\begin{align*}
				h_t = f(w_x x_t + w_h \textcolor{red}{h_{t-1}} + b_h) \\
				\hat{y}_t = g(w_y h_t + b_y)
			\end{align*}
			\item \optword{Jordan network}
			\begin{align*}
				h_t = f(w_x x_t + w_h \textcolor{red}{\hat{y}_{t-1}} + b_h) \\
				\hat{y}_t = g(w_y h_t + b_y)
			\end{align*}
		\end{itemize}
	\end{minipage}
	%
	\begin{minipage}{0.5\textwidth}
		\hgraphpage[\textwidth]{RNN.pdf}
	\end{minipage}
	
\end{frame}

\begin{frame}
	\frametitle{ML pour TALN : \insertsection}
	\framesubtitle{\insertsubsection\ : Applications}
	
	\begin{tabular}{p{.32\textwidth}p{.15\textwidth}p{.4\textwidth}}
		\hline\hline
		Type & Illustration & Exemple \\
		\hline
		Plusieurs à plusieurs & 
		\vgraphpage[1.5cm, valign=c]{RNNpp1.pdf} & 
		Détection d'entités nommées \\
		
		\hline
		Plusieurs à plusieurs & 
		\vgraphpage[1.5cm, valign=c]{RNNpp2.pdf} & 
		Traduction automatique \\
		
		\hline
		Plusieurs à un & 
		\vgraphpage[1.5cm, valign=c]{RNNp1.pdf} & 
		Classification de sentiments \\
		
		\hline
		Un à plusieurs & 
		\vgraphpage[1.5cm, valign=c]{RNN1p.pdf} & 
		Sous-titrage d'images \\
		
		\hline\hline
		
	\end{tabular}
	
\end{frame}

%\begin{frame}
%	\frametitle{ML pour TALN : \insertsection}
%	\framesubtitle{\insertsubsection : Un peu d'humour}
%	\hgraphpage{humour/humour-ngram.png}
%\end{frame}

%===================================================================================
\section{Classement des textes}
%===================================================================================

\begin{frame}
	\frametitle{ML pour TALN}
	\framesubtitle{\insertsection}
	
	\begin{itemize}
		\item Classer un texte entier 
		\begin{itemize}
			\item \optword{Analyse des sentiments} : classer un texte (comme les tweets) comme positif, négatif ou neutre
			\item \optword{Détection de spam} : classer un message comme spam ou non spam
			\item \optword{Détection de langage figuratif} : classer un texte comme métaphore, ironie, etc.
		\end{itemize}
		\item Classer une phrase du texte
		\begin{itemize}
			\item \optword{Résumé automatique} : classer une phrase du texte comme phrase du résumé ou non
			\item \optword{Détection de spam} : classer un message comme spam ou non spam
			\item \optword{Implication textuelle} : vérifier si la première phrase implique la deuxième
		\end{itemize}
	\end{itemize}
	
\end{frame}

\subsection{ML traditionnel et MLP}

\begin{frame}
	\frametitle{ML pour TALN : \insertsection}
	\framesubtitle{\insertsubsection}
	
	\begin{itemize}
		\item On doit trouver une représentation globale du texte.
		\item Cette représentation est introduite à l'algorithme pour le classement.
		\item Exemples :
		\begin{itemize}
			\item \expword{Nombre d'occurence de chaque mot du vocabulaire}
			\item \expword{Nombre des adjectifs dans le texte}
			\item \expword{Position de la phrase dans le texte}
			\item \expword{Trouver les représentations vectorielles des mots et calculer le centre pour représenter le texte }
		\end{itemize}
	\end{itemize}
	
\end{frame}

\begin{frame}
	\frametitle{ML pour TALN : \insertsection}
	\framesubtitle{\insertsubsection\ : Exemples}
	
	\vgraphpage{text_class_trad_exp.pdf}
	
\end{frame}


\subsection{CNN}

\begin{frame}
	\frametitle{ML pour TALN : \insertsection}
	\framesubtitle{\insertsubsection}
	
	\begin{itemize}
		\item On doit trouver une représentation vectorielle de chaque mot du texte (vecteur de \keyword{N} éléments)
		\item Les représentations sont concaténées verticalement (\keyword{M} mots)
		\item On aura un matrice de \keyword{M X N}
		\item On applique une convolution à une dimension (dimension = \keyword{K}) ; c.à.d. un filtre de dimension \keyword{K X N}
		\item On peut utiliser plusieurs filtres à la fois pour apprendre plusieurs représentations
		\item On peut utiliser le Pooling
		\item On utilise une couche \keyword{MLP} (ou plusieurs) à la fin pour estimer la classe
		\item \textbf{Problème} : Techniquement, on ne peut pas définir un réseau de neurones avec un nombre variable des mots 
		\item \textbf{Solution} : \textcolor{yellow!50}{On définit un nombre maximal ; si le nombre de mots est inférieur, on ajoute des paddings ; sinon, on supprime les mots en plus}
	\end{itemize}
	
\end{frame}

\begin{frame}
	\frametitle{ML pour TALN : \insertsection}
	\framesubtitle{\insertsubsection\ : Exemples}
	
	\vgraphpage{text_class_CNN_exp.pdf}
	
\end{frame}


\subsection{RNN}

\begin{frame}
	\frametitle{ML pour TALN : \insertsection}
	\framesubtitle{\insertsubsection}
	
	\begin{itemize}
		\item On doit trouver une représentation vectorielle de chaque mot du texte (vecteur de \keyword{N} éléments)
		\item Le réseau RNN crée M cellules correspondantes aux \keyword{M} mots
		\item Chaque cellule calcule un état intermédiaire et le passe au cellule suivante
		\item La dernière cellule génère une représentation du texte (la probabilité d'occurene de ce mot sachant les mots passés)
		\item Cette représentation est introduite à un \keyword{MLP} pour estimer la classe du texte
		\item \textbf{Problème} : Techniquement, lors de l'entrainement, on ne peut pas traiter les mots de tailles différentes avec une seule opération matricielle
		\item \textbf{Solution} : \textcolor{yellow!50}{On définit un nombre maximal ; si le nombre de mots est inférieur, on ajoute des paddings ; sinon, on supprime les mots en plus}
	\end{itemize}
	
\end{frame}

\begin{frame}
	\frametitle{ML pour TALN : \insertsection}
	\framesubtitle{\insertsubsection\ : Exemples}
	
	\vgraphpage{text_class_RNN_exp.pdf}
	
\end{frame}

%===================================================================================
\section{Classement des séquences}
%===================================================================================

\begin{frame}
	\frametitle{ML pour TALN}
	\framesubtitle{\insertsection}
	
\end{frame}

\subsection{HMM}

\begin{frame}
	\frametitle{ML pour TALN : \insertsection}
	\framesubtitle{\insertsubsection}
	
\end{frame}

\subsection{MEMM}

\begin{frame}
	\frametitle{ML pour TALN : \insertsection}
	\framesubtitle{\insertsubsection}
	
\end{frame}

\subsection{CRF}

\begin{frame}
	\frametitle{ML pour TALN : \insertsection}
	\framesubtitle{\insertsubsection}
	
\end{frame}

\subsection{RNN}

\begin{frame}
	\frametitle{ML pour TALN : \insertsection}
	\framesubtitle{\insertsubsection}
	
\end{frame}
%===================================================================================
\section{Attention}
%===================================================================================

\begin{frame}
	\frametitle{ML pour TALN}
	\framesubtitle{\insertsection}
	
\end{frame}

\subsection{Seq2Seq}

\begin{frame}
	\frametitle{ML pour TALN : \insertsection}
	\framesubtitle{\insertsubsection}
	
\end{frame}

\subsection{Seq2Seq avec attention}

\begin{frame}
	\frametitle{ML pour TALN : \insertsection}
	\framesubtitle{\insertsubsection}
	
\end{frame}

\subsection{Types d'attention}

\begin{frame}
	\frametitle{ML pour TALN : \insertsection}
	\framesubtitle{\insertsubsection}
	
\end{frame}


\subsection{Trasformers}

\begin{frame}
	\frametitle{ML pour TALN : \insertsection}
	\framesubtitle{\insertsubsection}
	
\end{frame}

\insertbibliography{TALN03}{*}

\end{document}

