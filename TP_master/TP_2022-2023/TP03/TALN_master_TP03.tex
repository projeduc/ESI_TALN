% !TEX TS-program = pdflatex
% !TeX program = pdflatex
% !TEX encoding = UTF-8
% !TEX spellcheck = fr

\documentclass[11pt, a4paper]{article}
%\usepackage{fullpage}
\usepackage[left=1cm,right=1cm,top=1cm,bottom=2cm]{geometry}
\usepackage[fleqn]{amsmath}
\usepackage{amssymb}
%\usepackage{indentfirst}
\usepackage[T1]{fontenc}
\usepackage[utf8]{inputenc}
\usepackage[french,english]{babel}
\usepackage{txfonts} 
\usepackage[]{graphicx}
\usepackage{multirow}
\usepackage{hyperref}
\usepackage{parskip}
\usepackage{multicol}
\usepackage{wrapfig}

\usepackage{turnstile}%Induction symbole

\usepackage{tikz}
\usetikzlibrary{arrows, automata}
\usetikzlibrary{decorations.pathmorphing}

\renewcommand{\baselinestretch}{1}

\setlength{\parindent}{24pt}


\begin{document}

\selectlanguage {french}
%\pagestyle{empty} 

\noindent
\begin{tabular}{ll}
\multirow{3}{*}{\includegraphics[width=2cm]{../../../img/esi-logo.png}} & \'Ecole national Supérieure d'Informatique\\
& Master (2022-2023) \\
& Traitement automatique du langage naturel
\end{tabular}\\[.25cm]
\noindent\rule{\textwidth}{1pt}\\%[-0.25cm]
\begin{center}
{\LARGE \textbf{TP03 : Rédaction d'un article sur une application TALN}}
\begin{flushright}
	ARIES Abdelkrime
\end{flushright}
\end{center}
\noindent\rule{\textwidth}{1pt}

Nous avons un article rédigé sur la méthode proposée en TP01.
Dans ce TP, vous devez modifier cet article afin de décrire votre méthode conçue en TP02.


\section{Améliorations possibles}

Il faut 
\begin{itemize}
	\item changer le titre
	\item changer l'auteur
	\item changer quelques phrases dans le résumé
	\item changer des parties de la description de la méthode 
	\item changer des parties des expérimentations et de la conclusion
\end{itemize}

\section{Recommandations}

\begin{itemize}
	\item Vous pouvez utiliser Overleaf; envoyer un fichier zip contenant le projet
	\item Il faut rendre le pdf généré et pas le code \LaTeX
\end{itemize}


\end{document}
