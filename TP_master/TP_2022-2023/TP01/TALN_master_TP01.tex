% !TEX TS-program = pdflatex
% !TeX program = pdflatex
% !TEX encoding = UTF-8
% !TEX spellcheck = fr

\documentclass[11pt, a4paper]{article}
%\usepackage{fullpage}
\usepackage[left=1cm,right=1cm,top=1cm,bottom=2cm]{geometry}
\usepackage[fleqn]{amsmath}
\usepackage{amssymb}
%\usepackage{indentfirst}
\usepackage[T1]{fontenc}
\usepackage[utf8]{inputenc}
\usepackage[french,english]{babel}
\usepackage{txfonts} 
\usepackage[]{graphicx}
\usepackage{multirow}
\usepackage{hyperref}
\usepackage{parskip}
\usepackage{multicol}
\usepackage{wrapfig}

\usepackage{turnstile}%Induction symbole

\usepackage{tikz}
\usetikzlibrary{arrows, automata}
\usetikzlibrary{decorations.pathmorphing}

\renewcommand{\baselinestretch}{1}

\setlength{\parindent}{24pt}


\begin{document}

\selectlanguage {french}
%\pagestyle{empty} 

\noindent
\begin{tabular}{ll}
\multirow{3}{*}{\includegraphics[width=2cm]{../../../img/esi-logo.png}} & \'Ecole national Supérieure d'Informatique\\
& Master (2022-2023) \\
& Traitement automatique du langage naturel
\end{tabular}\\[.25cm]
\noindent\rule{\textwidth}{1pt}\\%[-0.25cm]
\begin{center}
{\LARGE \textbf{TP01 : Découverte de quelques outils TALN}}
\begin{flushright}
	ARIES Abdelkrime
\end{flushright}
\end{center}
\noindent\rule{\textwidth}{1pt}

Dans ce TP guidé, nous allons développer quelques modèles pour l'analyse de sentiments (classement de textes).
Le dataset utilisé est un dataset d'analyse de sentiments dans le contexte de la finance (\url{https://www.kaggle.com/datasets/sbhatti/financial-sentiment-analysis}).
Nous allons utiliser \textbf{pandas}, \textbf{scikit-learn} et \textbf{nltk} :

\begin{verbatim}
import nltk
import numpy  as np
import pandas as pd
from sklearn.metrics                 import classification_report
from sklearn.preprocessing           import LabelBinarizer
from sklearn.neural_network          import MLPClassifier
from sklearn.model_selection         import train_test_split
from sklearn.feature_extraction.text import CountVectorizer

import re
token_pattern = re.compile(r'(?u)\b\w\w+\b')
tokenizer = token_pattern.findall
\end{verbatim}

\section{Préparation de données}

\begin{itemize}
	\item Importer le dataset CSV fourni en utilisant \textbf{pandas} et afficher les 10 premiers échantillons.
	\item Encoder les sentiments en utilisant \textbf{LabelBinarizer} pour avoir un encodage OneHot.
	\item Diviser le dataset en entrainement et test, en utilisant \textbf{train\_test\_split}, la taille du test est 30\% et random\_state=0.
	\item Entrainer un modèle de vectorisation \textbf{TF} sur le texte d'entrainement et vectoriser le. 
	\item En utilisant le même modèle, vectoriser le dataset de test.
	\item Définir une fonction \textbf{tokenstem} qui prend un texte et qui utilise \textbf{tokenizer} pour avoir une liste des tokens ensuite elle génère une liste des tokens radicalisés en utilisant \textbf{nltk.stem.porter.PorterStemmer} et elle ne considère pas les tokens appartenant à \textbf{nltk.corpus.stopwords.words('english')}
	\item Refaire les deux étapes précédentes, mais en limitant la taille du vecteur \textbf{max\_features = 3000} et en utilisant  \textbf{tokenstem} comme \textbf{tokenizer}.
	\item Refaire la même chose, mais sans limiter la taille du vecteur.
	
\end{itemize}

\section{Entrainement}

\begin{itemize}
	\item Créer trois modèles de type \textbf{MLPClassifier} avec une couche cachée et un maximum d'iérations de 50.
	\item Entrainer ces trois modèles sur les trois représentations vectorielles.
	\item Prédire les classes en appliquant les trois modèles sur les trois représentations d'entrainement.
	\item Afficher le rapport de classification en précisant \textbf{target\_names} à partir du modèle \textbf{LabelBinarizer} et en spécifiant \textbf{zero\_division=0}.
	
\end{itemize}

\section{Test}

\begin{itemize}
	\item Prédire les classes en appliquant les trois modèles sur les trois représentations de test.
	\item En utilisant \textbf{timeit}, calculer le temps de prédiction.
	\item Afficher le rapport de classification en précisant \textbf{target\_names} à partir du modèle \textbf{LabelBinarizer} et en spécifiant \textbf{zero\_division=0}.
	
\end{itemize}


\end{document}
